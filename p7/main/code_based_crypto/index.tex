\chapter{Code Based Cryptography}

The following introduction of the McElice public key cryptosystem (abreviated McElice PKCS), is based upon \cite{r6}. Before we can introduce the McElice PKCS, we will need to introduce some basic concepts from error correcting codes.

\begin{definition}
  Let $\mathcal{C}_1$ and $\mathcal{C}_2$ be $[n, k]_{q}$ codes, with generator matrices $G_1$ and $G_2$ respectively then:
  \begin{enumerate}
    \item If there exists a permutation matrix $P \in  \mathbb{F}_{q}^{k \times k}$ such that $G_1 = G_{2}P$, then $\mathcal{C}_1$ and $\mathcal{C}_{2}$ are called \textit{permutation equivalent}.
    \item The codes $\mathcal{C}_1$ and $\mathcal{C}_2$ are called \textit{equivalent}, denoted $\mathcal{C}_1 \sim \mathcal{C}_{2}$, if there exists a matrix $S \in GL_n(\mathbb{F}_q)$ and permutation matrix $P \in \mathbb{F}_q^{n \times n}$ such that $G_1 = S G_2P$.
  \end{enumerate}
%  then if there exists a permutation matrix $P \in \mathbb{F}_q^{k \times k}$ and a non-singular matrix $S \in \mathbb{F}_q^{n \times n}$ such that $G_1 = SG_2P$, then $\mathcal{C}_1$ and $\mathcal{C}_2$ are called \textit{equivalent}, written $\mathcal{C}_1 \sim \mathcal{C}_{2}$. Additionally if there exits a permutation matrix $Q \in \mathbb{F}_{q}^{k \times k}$ such that $G_1 = QG_2$, then $\mathcal{C}_1$ is called \textit{permutation equivalent} to $\mathcal{C}_{2}$.
  %, then $\mathcal{C}_1$ and $\mathcal{C}_2$ are called \textit{permutation equivalent} if there exists a permutation matrix $P \in \mathbb{F}_q^{k \times k}$ such that $G_1 = G_{2}P$ alternatively if there exists a non-singular matrix $S \in \mathbb{F}_q^{n \times n}$ such that  $G_1 = SG_2P$ then $\mathcal{C}_1$ and $\mathcal{C}_2$ are called \textbf{equivalent}.
\end{definition}
\begin{remark}
  The relation $\sim$, is indeed an equivalence relation:
  \begin{enumerate}
\item If $G$ is a generator matrix, then $G = G$ so $\sim$ is reflective.
\item Additionally if $G_1 = SG_2P$  then $S^{-1}G_1P^{-1} = G_2$ so $\sim$ is symmetric.
\item Finally, to show that $\sim$ is transitive, let $G_1 = S_1G_2P_1$ and $G_2 = S_2G_{3}P_2$. Then $G_1 = (S_1S_2)G_3(P_2P_1)$.
  \end{enumerate}
\end{remark}

Two equivalent codes, clearly share the same dimension, after all their generator matricies have the same number of rows. However it is not immediately obvious that they share the same minimum distance.
\begin{proposition}\label{prop:equvilant_codes_share_parameters}
  Let $\mathcal{C}_1$ and $\mathcal{C}_2$ be equivalent $[n, k]_q$ codes, with generator matricies $G_1$ and $G_2$ respectively, such that $G_1 = SG_2P$, for some $S \in GL_n(\mathbb{F}_q)$ and permutation matrix $P \in \mathbb{F}_q^{k \times k}$: Then:
  \begin{enumerate}
    \item If $H_2 \in \mathbb{F}_q^{n \times k}$ is a parity check matrix of $\mathcal{C}_2$ then the matrix $H_2 P$ is a parity check matrix of $\mathcal{C}_{1}$. \label{prop:equvilant_codes_share_parameters1}
    \item Additionally $d(\mathcal{C}_1) = d(\mathcal{C}_2)$. \label{prop:equvilant_codes_share_parameters2}
  \end{enumerate}
\end{proposition}
\begin{proof}
  We start by proving Assertion \ref{prop:equvilant_codes_share_parameters1}. We do this by showing that $(H_2P)G_1^T = 0_{k \times k}$ meaning $\mathcal{C}_1 \subseteq \Null(H_2P)$ and that $\dim_{\mathbb{F}_q}(\Null(H_2P)) = k$.
  The fact that $\dim_{\mathbb{F}_q}(\Null(H_2P)) = k$ follows from the fact that $P$ is a permutation matrix and hence:
  \begin{equation*}
    \dim_{\mathbb{F}_q}(\Null(H_2P)) = \dim_{\mathbb{F}_q}(\Null(H_{2})) \dim_{\mathbb{F}_q} (\mathcal{C_{2}}) = k
  \end{equation*}
  Next using the fact that $G_1 = SG_2P$ we see that:
  \begin{equation*}
    H_2PG_1^{T} = H_2P(P^TG_2^TS^T) \overset{(a)}{=} H_2G_2^TS^T \overset{(b)}{=} 0_{n \times n}
    % NOTE: har fjernet en (S^T)^{-1}
  \end{equation*}
  where equality $(a)$ follows as $P$ is a permutation matrix and hence orthogonal, and equality $(b)$ as $H_2G_{2}^{T} = 0_{n \times n}$.

  Continuing Assertion \ref{prop:equvilant_codes_share_parameters2} follows by combining Assertion \ref{prop:equvilant_codes_share_parameters1} with the fact that the minimum distance of a code with parity check matrix $H$, corresponds with the minimum number of linearly dependent columns of $H$. Finally we conclude the proof by noting that $H_2P$ has the same columns as $H_2$.
  %We note that $\mathcal{C}_1 \sim \mathcal{C}_2$ implies that there exists a $S \in GL_{n}(\mathbb{F}_{q})$ and a permutation matrix $P \in \mathbb{F}_q^{k \times k}$ such that $G_1 = SG_{2}P$. Hence as $S$ and $P$ are non-singular we know that $rank(G_1) = rank(G_{2})$.
  \label{}
\end{proof}
\begin{definition}\label{def:decoder}
Let $\mathcal{C}$ be a $[n, k, d]_{q}$ code and let $t \leq \floor{\frac{d - 1}{2}}$. A \textit{$t$-error correcting decoder} for $\mathcal{C}$ is a mapping $Dec_{\mathcal{C}}: \mathbb{F}_q^n \to \mathcal{C} \cup \left\{?\right\}$ which satisfies the condition that $Dec_{\mathcal{C}}(y) = c$ whenever $y = c + e$, with $c \in \mathcal{C}$ and $e \in \mathbb{F}_q^n$ such that $\wt(e) \leq t$, and $Dec_{\mathcal{C}}(y) = \; ?$ otherwise. An algorithm which implements a $t$-error correcting decoder for $\mathcal{C}$ is called a \textit{$t$-error correcting decoding algorithm} for $\mathcal{C}$.
\end{definition}
\begin{remark}
  We often simply refer to the $t$-error correcting decoder $Dec_{\mathcal{C}}$ as a \textit{decoder} and any algorithm which implements a decoder for $\mathcal{C}$ as an \textit{decoding algorithm} for $\mathcal{C}$.
\end{remark}

%Let $\mathcal{C}$ be a $[n, k, d]_q$ code with generator matrix $G \in \mathbb{F}_q^{n \times k}$, such that we have an efficient algorithm $dec_{\mathcal{C}}$ for decoding codewords up to $t$ errors. Furthermore we let $S \in \mathbb{F}_q^{k \times k}$ be a random non-singular matrix and $P \in \mathbb{F}_q^{n \times n}$ be a random permutation matrix.
%The secret key will be the tripe $(Dec_{\mathcal{C}}, S, P)$ and the public key will be $G' = S \cdot G \cdot P$. To encrypt a $k$-length message $m$, we will compute $y = m G' + e$, such that $e \in \mathbb{F}_q^{n}$ is a random vector with $\wt(e) \leq t$. The decryption of the message is slightly more complicated, the algorithm is given in Algorithm \ref{alg:McElice_decrypt}.

We have now covered all of the necessary tools needed to introduce the McElice PKCS:
Let $\mathcal{C}$ be an $[n, k, d]_q$ code with generator matrix $G$ and an efficient $t$-error correcting decoding algorithm $Dec_{\mathcal{C}}$, furthermore let $S \in GL_{k \times k}(\mathbb{F}_q)$ and $P \in \mathbb{F}_q^{n \times n}$ be a random permutation matrix. An overview of the McElice PKCS is given below in Algorithm \ref{alg:McElice}.
\begin{algorithm}
\caption{McElice PKC}\label{alg:McElice}
\begin{algorithmic}
  \State \textbf{private key} $(Dec_{\mathcal{C}}, S, P)$
  \State \textbf{public key} $(G' = S \cdot G \cdot P, t)$
  \\
  \Procedure{McElice Encryption} {$m$: plain message}
    \State $c \gets m^{T} G'$
    \State Choose $e \in \mathbb{F}_q^n$ uniformly, such that $\wt(e) = t$.
    \State \Return $c + e$
  \EndProcedure \\
  \Procedure{McElice Decryption} {$y$: received chipher-text}
    \State $y' \gets y P^{-1}$
    \State $m' \gets Dec_{\mathcal{C}}(y')$
    \State \Return $m' S^{-1}$
  \EndProcedure
\end{algorithmic}
\end{algorithm}

\begin{proposition}\label{prop:decryption_algorithm_produces_correct_message}
  Given some $y = m^{T}G' + e$, with $e \in \mathbb{F}_q^{n}$ such that $\wt(e) = t$, the decryption algorithm described in Algorithm \ref{alg:McElice} yiels the correct message $m^{T}$.
\end{proposition}
\begin{proof}
  Sticking to the notation used in Algorithm \ref{alg:McElice} we have:
  \begin{equation*}
    y' := (m^{T} G' + e) P^{-1} = m^{T}SG + e P^{-1}
  \end{equation*}
  However as $P^{-1} = P^{T}$ is also permutation matrix, we see that $\wt(eP^{-1}) = t$. Hence we may apply our $t$-error correcting decoder $Dec_{\mathcal{C}}$ to $y'$ and get $m' := m^{T}S$ now multiplying $m'$ by $S^{-1}$ we obtain $m^{T}$.
\end{proof}

\begin{remark}
  The matrix $G'$ will be the generator matrix of another $[n, k]_{q}$ code $\mathcal{C}'$. By Proposition \ref{prop:equvilant_codes_share_parameters} $d(\mathcal{C}') = d(\mathcal{C})$. Hence the decoding of $m^T G' + e$ also makes sense. In fact the decoding $m^T G' + e$ is one of the ways to attack the McElice PKCS. We will discuss this in more detail in Sections \ref{sec:syndrome_decoding} and \ref{sec:information_set_decoding}.
\end{remark}

We mentioned earlier that we would study public key cryptosystem which where bases on the general decoding problem, See Problem \ref{prob:general_decoding}. However the McElice PKCS is actually based on the following ``weaker'' problem, in the sense that an efficient algorithm for solving this problem doesn't necessarily yield an efficient solution to Problem \ref{prob:general_decoding}. However if $\mathbf{NP} = \mathbf{P}$, meaning that Problem \ref{prob:general_decoding} could be solved in polynomial time, then the McElice PKCS would be venerable to attack.
\begin{problem}[McElice Problem]\label{prob:McElice}
  Given $(G', t)$ and a ciphertext $y$ find the unique $m \in \mathbb{F}_q^{k}$ such that $\wt(m^{T}G' - c) = t$.
\end{problem}
The main difference between Problem \ref{prob:McElice} and Problem \ref{prob:general_decoding} is that we are provided with a basis of $\mathcal{C}'$ in Problem \ref{prob:McElice}, since we know $G'$. However the idea is to have as little structure of the underlying code revealed by $G'$ as possible, to make the problem closer to Problem \ref{prob:general_decoding}.

In general we have two kinds of attacks on the McElice PKCS
\begin{enumerate}
    \item \textit{Structural Attacks}: Where we try to extract information about the underlying code, using $G'$. If we obtain enough information we may be able to implement our own efficient $t$-error correcting decoding algorithm.
    \item \textit{Generic Attacks}: Where we try to construct efficient $t$-error correcting decoding algorithms given $(G', t)$, without concerning our selves with the underlying structure of $\mathcal{C}$, we will refer to such decoding algorithms as being \textit{generic}.
\end{enumerate}

Two of the main goals of this project is to investigate these attacks. Understanding of structural attacks allows us to gain insights into which families of codes constitutes ``good'' candidates for use in the McElice PKCS. While generic attacks allows us to measure the security of the McElice PKCS constructed on the codes resilient to structural attacks.

\section{Syndrome Decoding} \label{sec:syndrome_decoding}
Before we get to the syndrome we introduce a way to partition a linear code, which will the syndrome decoding algorithm will be built upon.
\begin{definition}
  Let $\mathcal{C}$ be a $[n, k]_q$ code and let $y \in \mathbb{F}_q^n$, then the \textit{coset} of $\mathcal{C}$ determined by $y$ is defined as:
  \begin{equation*}
    \mathcal{C} + y := \left\{c + y : c \in \mathcal{C}\right\} =: y + \mathcal{C}
  \end{equation*}
\end{definition}
We note that these cosets are nothing more, than the cosets found in group theory as $(\mathcal{C}, +)$ forms a subgroup of the abelian group $(\mathbb{F}_q^n, +)$, hence the results of the following proposition should come as no suprice, never the less, we will provide a proof.
\begin{proposition}\label{prop:basic_properties_of_cosets}
  Let $\mathcal{C}$ be a $[n, k]_q$ code, then the following holds for all $y, y' \in \mathbb{F}_q^{n}$:
  \begin{enumerate}
    \item If $y' \in \mathcal{C} + y$ then $\mathcal{C} + y = \mathcal{C} + y'$.\label{prop:basic_properties_of_coset1}
    \item If $\mathcal{C} + y \neq \mathcal{C} + y'$ then $(\mathcal{C} + y) \cap (\mathcal{C} + y') = \emptyset$. \label{prop:basic_properties_of_coset2}
    \item There are $q^{n - k}$ disjoint cosets of $\mathcal{C}$. \label{prop:basic_properties_of_coset3}
  \end{enumerate}
\end{proposition}
\begin{proof}
  We start with Assertion \ref{prop:basic_properties_of_coset1}:
  If $y' \in \mathcal{C} + y$, then $y' = c + y$ for some $c \in \mathcal{C}$,
  If $y' \in \mathcal{C} + y$, then $y' = c + y$ for some $c \in \mathcal{C}$, hence $y' + (-c) = y$, now since $y' + (-c)$ and $\mathcal{C}$ is a vector space we see that
  \begin{equation*}
   \underset{=y}{\underbrace{y' + (-c)}} + c' \in \mathcal{C} + y \text{ for all } c' \in \mathcal{C}
  \end{equation*}
  so $y' + \mathcal{C} \subseteq y + \mathcal{C}$. A similar argument can be made in the other direction, hence $\mathcal{C} + y = \mathcal{C} + y'$.

  Continuing with Assertion \ref{prop:basic_properties_of_coset2}: Assume for the sake of contradiction that $\mathcal{C} + y \neq \mathcal{C} + y'$ and $(\mathcal{C} + y) \cap (\mathcal{C} + y') \neq \emptyset$, then pick $x \in (\mathcal{C} + y) \cap (\mathcal{C} + y')$, now by Assertion \ref{prop:basic_properties_of_coset1} we have $\mathcal{C} + y) = x + \mathcal{C} = (\mathcal{C} + y')$ which is clearly a contradiction.

  Finally Assertion \ref{prop:basic_properties_of_coset3} follows by the Lagrange index theorem as:
  \begin{equation*}
    q^{n} = \abs{\mathbb{F}_q^n} = \abs{\mathbb{F}_q^{n} / \mathcal{C}}\abs{\mathcal{C}} = \abs{\mathbb{F}_q^n / \mathcal{C}} q^{k}
  \end{equation*}
  implies that $\abs{\mathbb{F}_q^n / \mathcal{C}} = q^{n - k}$. The fact that the cosets are disjoint follows directly from Assertion \ref{prop:basic_properties_of_coset2}.
\end{proof}

\begin{definition}
  Let $\mathcal{C}$ be a $[n, k]$ code with parity check matrix $H \in \mathbb{F}_q^{(n - k) \times n}$, then for any  $y \in \mathbb{F}_q^n$ we define its \textit{syndrome}, with respect to $H$, to be $S_H(y) = H y$.
\end{definition}

\begin{remark}\label{rem:basic_properties}
  Clearly $S_H(y) = 0$ if and only if $y \in \mathcal{C}$, this is exactly the definition of a parity check matrix. In addition we have $S_H(y + y') = S_H(y) + S_H(y')$ since matrix vector multiplication is distributive.
\end{remark}

\begin{lemma}\label{lem:syndrome_is_the_same_iff_they_are_in_the_same_coset}
  Let $\mathcal{C}$ be a $[n, k]_{q}$ code with parity check matrix $H$ and $y, y' \in \mathbb{F}_q^{n}$, then $S_H(y)=S_H(y')$ if and only if $y$ and $y'$ are in the same coset of $\mathcal{C}$.
\end{lemma}

\begin{proof}
  Firstly we note that for all $y, y' \in \mathbb{F}_q^n$ there exists a $x \in \mathbb{F}_q^n$ such that $y = y' + x$ and hence $S_H(y) = S_H(y') + S_H(x)$ per Remark \ref{rem:basic_properties}. Now if $S_H(y) = S_H(y')$ we must have $S_H(x) = 0$ meaning $x \in \mathcal{C}$, again by Remark \ref{rem:basic_properties}. On the other hand if $y$ and $y'$ are in the same coset, then there exists a codeword $c \in \mathcal{C}$ such that $y = y' + c$ and hence $S_H(y) = S_H(y') + S_H(c) = S_H(y')$ by Remark \ref{rem:basic_properties}.
  %First we note that we may write $y = c + e$ and $y' = c' + e'$ with $c, c' \in \mathbb{F}_q$ and $e, e' \in \mathbb{F}_q^{n}$. We see that $S_H(y) = S_H(c) + S_H(e) = S_H(e)$ by Remark \ref{rem:basic_properties_of_syndromes}, and similarly that $S_H(y') = S_H(e')$. Hence $S_H(y) = S_H(y')$ if and only if $S_H(e) = S_H(e')$, however this happends if and only if $e = e' + c^{*}$ for some $c^{*} \in \mathcal{C}$, by Remark \ref{rem:basic_properties_of_syndromes}.
\end{proof}

\begin{definition}\label{def:syndrome_lookup_table}
  If $\mathcal{C}$ is a $[n, k]_q$ code, with parity check matrix $H \in \mathbb{F}_q^{(n - k) \times k}$, then we define the \textit{syndrome lookup table} (SLT) $S^{*}_H: \mathbb{F}_q^n \to \mathbb{F}_q^n$ as
  \begin{equation*}
    S_{H}^{*}(y) = \underset{e \in y + \mathcal{C}}{\arg \min} \; \wt(e)
  \end{equation*}
  Furthermore the vector $S_H^{*}(y)$ is called the \textit{coset leader} of $y + \mathcal{C}$.
\end{definition}

We are now finally able to describe the algorithm used for decoding recived words with syndrome decoding. Syndrome decoding will require such a lookup table, alternatively we could when decoding a received word $y$ go through the elements of $y + \mathcal{C}$ to find the coset leader, however this is impractical and we instead use a syndrome lookup table to avoid excess computation. Hence we will first introduce an algorithm for constructing such a lookup table.

\begin{algorithm}[H]
\caption{Syndrome Lookup Table Construction and Syndrome Decoding}\label{alg:syndrome_decoding}
\begin{algorithmic}
  \Procedure{SLT Construction} {$\mathcal{C}$: a $[n,k]_q$ code, $H$: parity check matrix of $\mathcal{C}$}
  \State $S_H^{*} \gets \emptyset$
  \For{$(y  + \mathcal{C}) \in \mathbb{F}_q^n / \mathcal{C}$}
     \State $\hat{e} \gets \underset{e \in y + \mathcal{C}}{\arg \min} \; \wt(e)$
     \State $S_H^{*} \gets S_H^{*} \cup \{(S_H(\hat{e}), \hat{e})\}$
  \EndFor
  \State \Return $S_H^{*}$ \Comment{Vied as a mapping.}
  \EndProcedure
  \newline

  \Procedure{Syndrome Decoding} {$y$: received word, $S_{H}^{*}$: syndrome lookup table}
    \State $\hat{e} \gets S_H^{*}(Hy)$
    \State \Return $y - \hat{e}$
  \EndProcedure
\end{algorithmic}
\end{algorithm}
\begin{remark}
The $S^*_H$ constructed in Algorithm \ref{alg:syndrome_decoding}, differs from the $S_H^{*}$ defined in Definition \ref{def:syndrome_lookup_table}, in that in the fact that it takes the syndrome of $y$ and returns the coset leader. However the function $S_H^{*} \circ S_H$ do confine to Definition \ref{def:syndrome_lookup_table}.
\end{remark}
\begin{remark}
  Since $\hat{e}$ is the coset leader of $y + \mathcal{C}$, we know that $y - \hat{e}$ is our nearest neighbor estimate for $c$, since $\wt(\hat{e})$ is minimal.
\end{remark}
We also note that since every coset contains a finite number of words, we may simply iterate through them to find the one which minimizes the hamming weight. By now the diligent reader, might start to question how does all this relate to solving the McEliece problem, since we simply know a generator matrix? This might seem like a problem, however it is pretty trivial to solve, we start by proving the following theorem:

\begin{theorem}\label{thm:from_generator_matrix_to_parity_check_matrix}
  Let $\mathcal{C}$ be a $[n, k]_q$ code and $A \in \mathbb{F}_q^{(n - k) \times k}$. Then $G = \begin{bmatrix} I_k & A \end{bmatrix}$ is a generator matrix for $\mathcal{C}$ if and only if $H = \begin{bmatrix}-A^T & I_{n - k}\end{bmatrix}$ is a parity check matrix for $\mathcal{C}$.
%Let $\mathcal{C}$ be a $[n, k]_q$ code. If $\mathcal{C}$ has a generator matrix $G$ which is in standard form, that is $G = \begin{bmatrix}
%I_k & A
%                                                                                                                   \end{bmatrix}$ for some $A \in \mathbb{F}_q^{(n  - k) \times k}$, then $H = \begin{bmatrix}
%-A^T & I_{n - k}
%                                                                                                                                                                                           \end{bmatrix}$ is a parity check matrix for $\mathcal{C}$.
\end{theorem}
\begin{proof}
  We start by noting that $HG^T = -A^T + A^T = 0_{k \times k}$ hence all rows of $G$ are in $\Null(H)$. Hence if $G$ is a generator matrix of $\mathcal{C}$, then $H$ is a parity check matrix of $\mathcal{C}$. On the otherhand if $H$ is a parity check matrix for $\mathcal{C}$, then $G$ is a generator matrix for $\mathcal{C}$ since $\dim_{\mathbb{F}_q}(\Null(H)) = n - \rank(H) = n - (n - k) = k$ and $\rank(G) = k$. Hence the rows of $G$ must form a $\mathbb{F}_q$-basis of $\Null(H) = \mathcal{C}$.
\end{proof}

Hence if we receive a generator matrix for a $[n, k]_q$ code $C$, we can simply apply Gaussian elimination algorithm to get a generator matrix $G$ which is in reduced echelon form, note that $G$ isn't neccecarily standard form, but by multiplying our generator matrix $G$ by a permutation matrix $P$, we may obtain a generator matrix $G'$, which is in standard form, since $G$ has $k$ pivots as $\dim(\mathcal{C}) = k$. We note that $G'$ is not nessecarily a generator matrix for $\mathcal{C}$ but rather a generator matrix for a code $\mathcal{C}'$ permutation equvilent to $\mathcal{C}$. Applying Theorem \ref{thm:from_generator_matrix_to_parity_check_matrix} we obtain a parity check matrix $H'$ for $\mathcal{C}'$, which we transform into a parity check matrix $H$ of $\mathcal{C}$ by setting $H = P^{-1}H'$

Finally we consider the time and space complexity of Algorithm \ref{alg:syndrome_decoding}. We will consider the two procedures described seperately, since we only need to create one syndrome lookup table once, to decode
a given code.

Since $\textsc{SLT Construction}$ loops over each coset of $\mathcal{C}$, of which there is $q^{n - k}$, by \ref{prop:basic_properties_of_cosets}\ref{prop:basic_properties_of_coset3}. To find the coset leader we can assume that we simply iterate over the words in $y + \mathcal{C}$, of which there is $q^k$, hence we see that $\textsc{SLT Construction}$ has a time complexity of $O(q^{n})$. In addition since each syndrome and coset leader pair needs to be stored, we see that $\textsc{SLT Construction}$ has a space complexity of $O(q^{n - k})$.

Continuing with the $\textsc{Syndrome Decoding}$ procedure, we once again get a space complexity of $O(q^{n - k})$, since we have to store $S_H^{*}$, while the time complexity depends on the underlying datastructure chosen to represent $S_H^{*}$, however it will be atleast $O((n - k)n)$ since we have to compute $S_H(y) = Hy$ and $H$ is a $(n - k) \times n$ matrix

\section{The Niederreiter Public Key Cryptosystem}

Suppose $\mathcal{C}$ is a $[n, k, d]_q$ code with parity check matrix $H$ and an efficient $t$-error correcting decoding algorithm $dec_{\mathcal{C}}$. Furthermore let $S \in GL_{(n - k) \times (n - k)}(\mathbb{F}_q)$ and $P \in \mathbb{F}_q^{n \times n}$ be a random permutation matrix. The Neiderreiter PCKS (based on $\mathcal{C}$), which allows for encryption of a plain message $e \in \mathbb{F}_q^n$ with $\wt(e) = t$, is described below in Algorithm \ref{alg:Neiderreiter}.

\begin{algorithm}
\caption{The Neiderreiter PKCS}\label{alg:Neiderreiter}
\begin{algorithmic}
  \State \textbf{private key} $(Dec_{\mathcal{C}}, S, P)$
  \State \textbf{public key} $(H' = S \cdot H \cdot P, t)$
  \\
  \Procedure{Neiderreiter Encryption} {$e$: plain message with $\wt(e) = t$}
    \State \Return $H'e$
  \EndProcedure \\
  \Procedure{Neiderreiter Decryption} {$y$: received ciphertext}
    \State Find $z \in \mathbb{F}_q^n$ such that $Hz = S^{-1}y$ using linear algebra.
    \State $c \gets Dec_{\mathcal{C}}(z)$
    \State \Return $c P^{-1}$
  \EndProcedure
\end{algorithmic}
\end{algorithm}

Finding a $z \in \mathbb{F}_q^{n}$ such that $Hz = S^{-1}y$ is simply a matter of solving a linear system. Additionally since $S^{-1}y = S^{-1}SHPe = HPe$ we see that $c := Dec_{\mathcal{C}}(z)$ is the closest codeword to $HPe$ so $e = z - c P^{-1}$.

\subsection{The Equivalence Between the Neiderreiter and the McElice PCKS}%

Consider the $[n, k, d]_{q}$ code $\mathcal{C}$ with generator matrix $G$ and parity check matrix $H$. We demonstrate that the McElice and Neither cryptosystems based on $\mathcal{C}$ have an equivalent level of security. That is if a there exists an efficient attack on the McElice PCKS (based on $\mathcal{C}$), then there exists an efficient attack on the Neither PCKS (based on $\mathcal{C}$).

Hence we let $G'$ and $H'$ be the public keys of the McElice and Neither PCKS respectively.

Suppose we have a message $m \in \mathbb{F}_q^k$, we encrypt the message a obtain $y \in \mathbb{F}_q^{n}$, using Algorithm \ref{alg:McElice}. That is:
\begin{equation*}
  y = m^T G' + e
\end{equation*}
where $e \in \mathbb{F}_q^n$ with $\wt(e) = t$. Given $G'$ we may obtain a parity check matrix $H'$ for the code generated confer Theorem \ref{thm:from_generator_matrix_to_parity_check_matrix}. Multiplying by $(H')^T$ we get:
\begin{equation}\label{eq:Neiderreiter_and_McElice}
  y(H')^T = mG'(H')^T + e (H')^T = e (H')^T
\end{equation}
where the last equality follows as $G'(H')^T = 0$. Additionally since $y$ and $H'$ are public, the righthand side of Equation \eqref{eq:Neiderreiter_and_McElice} can easily be computed. Furthermore since $\wt(e) = t$, we see that we may compute the error $e$ efficiently provided we have an efficient attack on the Neiderreiter PCKS. Hence an efficient attack on the Neiderreiter PCKS would lead to an efficient attack on the corresponding McElice PCKS with very little overhead.

Conversely assume that we have a message $e \in \mathbb{F}_q^{n}$ such that $\wt(e) = t$. If we encrypt the message to obtain $y \in \mathbb{F}_q^{(n - k)}$, using Algorithm \ref{alg:Neiderreiter}. That is:
\begin{equation*}
  y = H'e
\end{equation*}
Again we may obtain generator matrix $G'$ of the code $\Null(H')$, since the null space of a matrix is invariant under row operations, thus we may apply Theorem \ref{thm:from_generator_matrix_to_parity_check_matrix}. Using basic linear algebra one my find a vector $z \in \mathbb{F}_q^n$ with $\wt(z) \geq t$ such that $y = H'z$ after all $d$ is the minimum number of linearly independent columns of $H'$ and $t \leq \floor{\frac{d - 1}{2}}$. Hence:
\begin{equation*}
  y = H'z \text{ and } z = yG' + e
\end{equation*}
Hence $e$ could be extracted efficiently provided that there exists an efficient algorithm for breaking the McElice PCKS.


\section{Advantages and Drawbacks of McElice and Neiderreiter}%
In this final subsection we will briefly discuss some of the advantages and drawbacks of using each system. Both compared to each other and compared to traditional public key cryptography systems. The results are summarized in Table \ref{tab:pros_and_cons}. The encryption procedure of the McElice PCKS is very efficient additionally there exists no well known quantum algorithm for breaking the McElice PCKS. The primary con of the McElice PCKS is the large key size, for example the original proposal by McElice, used a $[1024, 524]_2$ classical Goppa code, see Definition \ref{def:classical}. Due to the dimensions of the code the public key was roughly $524 \cdot 1024 = 536576$ bits or about $67.1$ KB \textcolor{blue}{while yielding a security level of about $65$ bits (remember to do the actual calculation). Meaning an attacker would have to perform $2^{65}$ operations to break the encryption.}. In comparison using RSA a public key size of only $3072$ bits is sufficient to yield a security level of $128$ bits, refer \cite{nist_recomendations_for_key_management}[Table 2].

For this reason much research has been done with the objective of lowering the public key size. However even though many of these proposals have succeeded in lowering the public key size, they often come with security issues.

The Neiderreiter PCKS has similar drawbacks as the McElice PCKS, in that the key size is very large compared to traditional public key cryptography systems, like RSA. However it has an additional drawback compared to the McElice PCKS, namely that the message has to have weight $t$ while having length $n$. Where as the only restriction on the message in the McElice PCKS is that it should have length $n$. On the otherhand one of the advantages of using the Neiderreiter PCKS is that it offers a smaller key size. Since the parity check matrix $H'$ may be published in systematic form, that is $H' = [A | I_{(n - k) \times (n - k)}]$, while keeping the security level the same, we prove this below in Proposition \ref{prop:H_can_be_in_standard_form}. We will however first need a small lemma:

\begin{lemma}\label{lem:equal_syndrome_and_equal_weight}
  Let $H$ be parity check matrix for the $[n, k, d]_q$ code $\mathcal{C}$, $t \leq \floor{\frac{d - 1}{2}}$ and $x, x' \in \mathbb{F}_q^n$. Then $\wt(x) = \wt(x') = t$ and $S_H(x) = S_H(x')$ implies $x = x'$.
\end{lemma}
\begin{proof}
  By Lemma \ref{lem:syndrome_is_the_same_iff_they_are_in_the_same_coset} we have $x = x' + c$ for some $c \in \mathcal{C}$. Thus $c = x - x'$ which implies:
  \begin{equation*}
    \wt(c) = \wt(x - x') \leq 2t < d
  \end{equation*}
  since $\wt(x) = \wt(x') = t$. Thus $c$ must equal zero.
\end{proof}

\begin{proposition}\label{prop:H_can_be_in_standard_form}
  Let $H$ be a parity check matrix for the $[n, k, d]_q$ code $\mathcal{C}$. Furthermore let $H' := UH$ be a systematic parity check matrix of $\mathcal{C}$, then any attack able to break the scheme using $H'$ is able to break a scheme using $H$.
\end{proposition}
\begin{proof}
  Let $W_t = \left\{x \in \mathbb{F}_q^n | \wt(x) = t\right\}$ with $t \leq \floor{\frac{d - 1}{2}}$. Then $S_H$ and $S_{H'}$, restricted to $W_{t}$, are injective by Lemma \ref{lem:equal_syndrome_and_equal_weight}. Assume that $\phi$ breaks the system using $H'$, that is $x = \phi(y) = S_{H'}^{-1}(y)$ for all $y \in S_{H'}(W_t)$. \\
  Next suppose $y \in S_{H}(W_t)$, then $y = S_{H}(x) = Hx$ for some $x \in W_t$. Hence:
  \begin{equation*}
    Uy = U S_H(x) = U H x = S_{H'}(x) \in S_{H'}(W_t)
  \end{equation*}
   and $\phi(Uy) = x$. That is $\phi$ can be used to break the system using the parity check matrix $H$, we note that $U$ can be obtained by performing elementary row operations on $H$.
\end{proof}

Since $H$ is allowed to be in standard form, may simply publish the first $k$ columns of $H$ assuming $H$ is in systematic form, allowing for a much smaller public key size at least when $k$ is relatively large compared to $n$. As an example consider McElices original proposal which used a $[1024, 524]_2$ classical Goppa code, meaning the public key of the equivalent neither system, could be published using $(1024-524) \cdot 524 = 26200$ or about $32.8$ KB, which is about half the size of the public key in the equivalent McElice PCKS. Finally the contents of the discussion is summarized in Table \ref{tab:pros_and_cons}.
\begin{table}[H]
    \centering
    \begin{tabular} {||c|c|c||}
        \hline
        \textbf{System} & \textbf{Advantages} & \textbf{Drawbacks} \\
        \hline
        McElice & \makecell{Fast encryption\\No known efficient quantum attacks}& Large key size (versus e.g. RSA) \\
        \hline

        Neiderreiter & \makecell{Smaller key size (versus McElice) \\ No known efficient quantum attacks} & \makecell{Large key size (versus e.g. RSA)\\ Messages must be of weight $t$}\\
        \hline
    \end{tabular}
    \caption{Advantages and drawbacks of the McElice and Neiderreiter PCKS.}
    \label{tab:pros_and_cons}
\end{table}

\chapter{Error Correcting Codes}\label{chap:error_correcting_codes}
In this section we will fix a finite field $\mathbb{F}_{q}$. We will apply the theory of algebraic geometry, to the affine and projective plane.
However since $\mathbb{F}_{q}$ is not algebraically closed, confer Proposition \ref{prop:finite_fields_arent_algebraicly_closed}, we will let $\cF_{q}$ be the algebraic closure of $\mathbb{F}_{q}$, as in Proposition \ref{prop:algebraic_closure_of_finite_field}.

\begin{definition}\label{def:affine_plane_curve}
  Let $F \in \F_{q}[X, Y]$, then the zero set $V(F)$ is called an \textit{affine plane curve} over $\cF_{q}$. The equation $F = 0$ is called the \textit{defining equation} of $V(F)$. The points $P \in \mathbb{F}_{q}^{n} \cap V(F)$ are called \textit{$\mathbb{F}_q$-rational points} of $V(F)$. The \textit{degree} of $V(F)$ is the degree of $F$, a curve of degree $1$ is called a \textit{line}.
\end{definition}
In general the defining equation of an affine plane curve, is not unique, for instance the equations $X = 0$ and $aX = 0$ describe the same curve for all $a \in \F_{q}^{*}$.
Additionally it is worth highlighting that many of the properties of the curve depend on the particular ground field.

\begin{example}\label{exmp:affine_line}
  Every line $L$ over $\mathbb{F}_{q}$ in the affine plane has $q$ $\mathbb{F}_{q}$-rational points. Consider the general defining equation $aX + bY + c = 0$ of a line where at least one of $a, b$ are non-zero.
  Then assuming $a \neq 0$ we get that $X = - a^{-1}(bY + c)$, so picking $y \in \F_{q}$ yields an element $x \in \F_{q}$ such that $(x, y)$ is a $\mathbb{F}_{q}$-rational point of $L$.
  Conversely if $a = 0$, we see that $bY + c = 0$. Hence we see that $Y = -b^{-1}c$, so $(x, -b^{-1}c)$ where $x \in \F_{q}$ are the only $\mathbb{F}_{q}$-rational points on $L$.
\end{example}
\begin{definition}
  Let $F \in \F_{q}[X, Y]$ and $P \in \mathbb{A}^{n}(\cF_{q})$ be a point on the affine plane curve $V(F)$, then $P$ is called \textit{singular} if $F_{X}(P) = F_{Y}(P) = 0$, the point $P$ is called \textit{regular} otherwise. If all points $P \in V(F)$ are singular or regular then the curve it self is called singular or regular respectively.
  Let $P = (a, b)$ be a regular point of the curve $V(F)$, then we define the \textit{tangent line} at $P$ as the affine curve $V \left(F_{X}(P)(X - a) + F_{Y}(P)(Y - b)\right)$.
\end{definition}


This is one of the properties, which depends strongly on the characteristic of the ground field, we will illustrate this in Example \ref{exmp:fermat_curve_derivatives}.

In corollary \ref{cor:irr_gives_affine_variety} we saw that the affine zero set of irreducible polynomials were affine varieties, we would like to extend the notion of irreducibility to polynomials over $\F_{q}$, however as we consider the affine zero set over the ground field $\cF_{q}$, we are primarily interested in if $F \in \F_{q}[X_1, X_2, \ldots, X_{n}]$ is also irreducible when viewed as a polynomial in $\cF_{q}[X_1, X_2, \ldots, X_{n}]$.
\begin{definition}
  Let $F \in \F_{q}[X_1, X_2, \ldots, X_{n}]$, then $F$ is called \textit{absolutely irreducible} if $F$ is irreducible in $\cF_{q}[X_1, X_2, \ldots, X_{n}]$. That is there exists no $G \in \cF_{q}[X_1, X_2, \ldots, X_{n}]$ with $0 < \deg(G) < \deg(F)$, such that $F = G H$ for some $H \in \cF_{q}[X_1, X_2, \ldots, X_{n}]$.
\end{definition}

We will call $V(F)$ an \textit{absolutely irreducible affine plane curve} if $F \in \mathbb{F}_{q}(X, Y)$ is absolutely irreducible, by Corollary \ref{cor:irr_gives_affine_variety} we see that every absolutely irreducible affine plane curve is an affine variety.

%By Corollary \ref{cor:irr_gives_affine_variety} we have that $\mathcal{X} := V(F)$ is an affine variety if $F \in \F_{q}[X, Y]$ is absolutely irreducible. We will call $Z(F)$ an \textit{absolutely irreducible curve} if $F$ is absolutely irreducible.

\begin{definition}
  If $F \in \cF_{q}[X, Y, Z]$ is a homogeneous polynomial, then $V_{\mathbb{P}}(F)$ is called a \textit{projective plane curve}. A point $P \in V_{\mathbb{P}}(F)$ is called \textit{rational} if there exists a representation $P = [a : b : c]$, where $a, b, c \in \F_{q}$.
\end{definition}
The \textit{defining equation} and \textit{degree} of a projective plane curve are defined similarly to the affine case. Hence we will omit them.
\begin{definition}
  Let $F \in \cF_{q}[X, Y, Z]$ be a homogeneous polynomial and $P \in V_{\mathbb{P}}(F)$. Then $P$ is called a \textit{singular} if $F_{X}(P) = F_{Y}(P) = F_{Z}(P) = 0$, otherwise $P$ is called \textit{regular}
\end{definition}
\begin{remark}
  Since the polynomial $F$ is homogeneous we see that $F_X, F_Y$ and $F_Z$ are homogeneous as well.
  Combining this with the fact that the notion of a root of a homogeneous polynomial at a projective point, is well defined, we see that the notion of a singular and regular point on a projective line is well defined. \\
  However since the value of a homogeneous polynomial at a projective point is not generally well defined, we can not define a tangent line of a projective plane curve.
\end{remark}


Let $F \in \F_q[X, Y]$, then $V(F)$ is an affine plane curve and $V(F^{*})$ is a projective plane curve, furthermore we say that $V(F^{*})$ \textit{corresponds to} $V(F)$. \\ Similarly to an affine curve, a projective plane curve is a projective variety if the defining polynomial $F$ is absolutely irreducible, by Corollary \ref{cor:irr_gives_projective_variety}. \\

\begin{example}\label{exmp:fermat_curve_derivatives}
  Let $F = X^m + Y^m + Z^{m}$, then the \textit{Fermat curve} of degree $m$ denoted $\mathcal{F}_m$ is the projective plane curve with defining equation $F = 0$. The partial derivatives $F$ was found in Example \ref{exmp:fermat_curve_derivatives}, as $mX^{m - 1}, mY^{m - 1}, mZ^{m - 1}$. Assuming $m \geq 2$ and that $m$ is coprime with $q$ then this curve is regular. However if $m$ is not coprime with $q$, then all points are singular.
\end{example}

\begin{example}\label{exmp:hermetian_curve}
   Let $p$ be a prime, and $\cF_{p^{2}}$ be the algebraic closure of $\mathbb{F}_{p^{2}}$, as noted Remark \ref{rem:existence_of_alg_closure} this algebraic closure exists. Let $F :=Y^{p}Z + YZ^{p} - X^{p + 1}$, then the \textit{Hermitian curve} $\mathcal{H}_{p}$ over $\cF_{p^{2}}$ is the projective curve with defining equation $F = 0$. The polynomial $F$ has the following partial derivatives: $F_{X} = -(p + 1)X^{p} = -X^{p}$, since $\ch({\cF_{p^{2}}}) = p$, $F_{Y} = Z^{p}$ and $F_{Z} = Y^{p}$. Hence the Hermitian curve is another example of a smooth curve.
\end{example}

\begin{lemma}\label{lem:maximal_ideal_of_plane_curves_are_prinicpal}
  Let $\mathcal{X}$ be a regular affine plane curve and $P \in \mathcal{X}$, then the unique maximal ideal $\mathfrak{m}_{P} \subseteq \mathcal{O}_{P}(\mathcal{X})$ is a principal ideal.
\end{lemma}

\begin{proof}
  Let $F = 0$ be the defining equation of $\mathcal{X}$. We may assume that $P = (0, 0)$, and that the tangent of $\mathcal{X}$ at $P$ has defining equation $Y = 0$\footnote{If this is not the case, then there exists an affine change of coordinates such that this is the case.}.
  Furthermore we have that $\mathfrak{m}_{P} = \gen{x, y}$, by Hilbert Nullstellensatz (Theorem \ref{thm:hilbert_nullstellensatz}).
  It follows that all monomials of $F$ have degree greater than or equal to $2$, as $\mathcal{X}$ is regular and the tangent line has defining equation $Y = 0$. Hence one can write
\begin{equation}\label{eq:m_p_is_pid}
  F(X, Y) = Y + Y G(Y) + X H(X, Y)
\end{equation}
where $G(0) = 0$ and $H(0, 0) = 0$. Now $F(x, y) = 0$ implies that $y = -x H(x,y) (1 + G(y))^{-1}$, by Equation \eqref{eq:m_p_is_pid}. From this it follows that $H(x, y) \in \mathcal{O}_{P}(\mathcal{X})$ as $G(0) = 0$. Therefore $y \in \gen{x}$ and hence $\mathfrak{m}_{P} = \gen{x, y} = \gen{x}$.
\end{proof}

\begin{remark}\label{rem:projective_is_also_dvr}
   If $\mathcal{X}$ is a projective plane curve we likewise have that $\mathcal{O}_{P}(\mathcal{X})$ is a prinicipal ideal, as $\mathcal{O}_{P}(\mathcal{X})$ is isomorphic to $\mathcal{O}_{P}(\mathcal{X}_{*})$, confer Remark \ref{rem:function_fields_are_iso}.
\end{remark}

%The next proposition is stated without its proof which can be found in \cite{Fulton}[Section 3.2]. The overall outline of the proof is quite similar to the proof of Lemma \ref{lem:maximal_ideal_of_plane_curves_are_prinicpal}. However will omit it as the proposition is primarily be used in examples.
%
%\begin{proposition}\label{prop:uniformizing_parameter}
%  Let $\mathcal{X}$ be a regular affine plane curve and $p \in \mathcal{X}$, if the line $\mathcal{L}$ with defining equation $F = 0$ is not a tangent to $\mathcal{X}$ at $p$, then $f$ is a uniformizing parameter of $\mathcal{O}_{p}(\mathcal{X})$.
%\end{proposition}

In many examples we will need a way to compute the uniformizing parameter at a point of a projective plane curve. Hence we state the next proposition from \cite{notes_on_alg_geom_codes}[Chapter 3] without proof as we will exclusively be using it in examples.
\begin{proposition}\label{prop:uniformizing_parameter}
  Let $\mathcal{X}$ be a smooth projective plane curve, and $P = [a : b : c] \in \mathcal{X}$ such that $c \neq 0$. Then $f = L_{1} / L_{2} \in \mathfrak{m}_{P}$ is a uniformizing parameter at $P$ if $\deg(L_{1}) = \deg(L_{2}) = 1$, $L_{2}(P) \neq 0$ and $L_1$ is not a constant multiple of $F_X(P)X + F_Y(P)Y + F_Z(P)Z$.
\end{proposition}
\begin{remark}
  It is actually sufficient that one of $a, b, c$ is non zero.\footnote{As we can apply a projective change of cordinates otherwise, for instance if $b \neq 0$, then we may swap $Y$ and $Z$.}
\end{remark}

Suppose that $\mathcal{X}$ is an affine or projective regular plane curve then, by Lemma \ref{lem:maximal_ideal_of_plane_curves_are_prinicpal} or Remark \ref{rem:projective_is_also_dvr}, there exists $t \in \mathcal{O}_{P}(\mathcal{X})$ such that $\mathfrak{m}_{P}(\mathcal{X}) = \gen{t}$. By Theorem \ref{thm:local_ring_is_a_DVR} We see that $\mathcal{O}_{P}(\mathcal{X})$ is a discrete valuation ring and that $t$ is a uniformizing parameter. We will denote the order of $f \in \mathcal{O}_{P}(\mathcal{X})$ as $ord_{P}(f)$.

\begin{definition}\label{def:vp}
  Let $\mathcal{X}$ be an regular plane curve, the function $v_{P}: \mathcal{O}_{P}(\mathcal{X}) \to \mathbb{N} \cup \left\{\infty\right\}$, defined as:
  \begin{equation*}
    v_{P}(f) = \begin{cases} \infty & \text{if } f = 0 \\ ord_{P}(f) & \text{otherwise.} \end{cases}
  \end{equation*}
  Let $f \in \mathcal{O}_{P}(\mathcal{X})$, then $P$ is called a \textit{zero} of order $v_P(f)$ if $v_{P}(f) > 0$.
\end{definition}
\begin{remark}\label{rem:extension}
The function $v_{P}$, can be extended to the domain $\cF_{q}(\mathcal{X})$ and codomain $\mathbb{Z} \cup \left\{\infty\right\}$, since $\mathcal{O}_{P}(\mathcal{X})$ is a subring of $\cF_{q}(\mathcal{X})$, by setting $v_{P}(f) = v_{P}(g) - v_{P}(h)$, where $f = g/h \in \cF_{q}(\mathcal{X})$. If $v_{P}(f) < 0$, then $f$ is said to have a \textit{pole} at $P$ of order $-v_{P}(f)$.
\end{remark}

We will now show that the map $v_{P}: \mathcal{O}_P(\mathcal{X}) \to \mathbb{N} \cup \left\{\infty\right\}$, that is the non extended version, is a \textit{discete valuation}, meaning that it satisfies properties \ref{thm:vp_is_dv:1}-\ref{thm:vp_is_dv:5} in the theorem below. However the map $v_{P}: \cF_{q}(\mathcal{X}) \to \mathbb{Z} \cup \left\{\infty\right\}$, have properties similar to \ref{thm:vp_is_dv:3} and \ref{thm:vp_is_dv:4} albeit for $f, g \in \cF_{q}(\mathcal{X})$.
\begin{theorem}\label{thm:vp_is_dv}
  The function $v_{P}: \mathcal{O}_P(\mathcal{X}) \to \mathbb{N} \cup \left\{\infty\right\}$, satisfies the following properties hold for all $f, g \in \mathcal{O}_{P}(\mathcal{X})$:
  \begin{enumerate}
    \item $v_P(f) = \infty$ if and only if $f = 0$. \label{thm:vp_is_dv:1}
    \item $v_P(\lambda f) = v_P(f)$ for all $\lambda \in \cF_{q}^{*}$.\label{thm:vp_is_dv:2}
    \item $v_{P}(f + g) \geq \min \left\{v_P(f), v_P(g)\right\}$.\label{thm:vp_is_dv:3}
    \item $v_P(fg) = v_P(f) + v_P(g)$.\label{thm:vp_is_dv:4}
    \item If $v_{P}(f) = v_P(g)$, where $f, g \in \mathcal{O}_{P}(\mathcal{X}) \setminus \left\{0\right\}$ then there exists a unit $\lambda \in \cF_{q}$ such that $v_P(f - \lambda g) > v_{P}(g)$ \label{thm:vp_is_dv:5}
  \end{enumerate}
\end{theorem}
\begin{proof}
  Assertions \ref{thm:vp_is_dv:1} and \ref{thm:vp_is_dv:2} follow directly from Definitions \ref{def:dvr} and \ref{def:vp}. Next we will show that $v_P$ satisfies Assertion \ref{thm:vp_is_dv:3}:
  Let $t \in \mathcal{O}_{P}(\mathcal{X})$ be the uniformizing parameter. Suppose $f = 0$ this implies that $v_{P}(f + g) = v_{P}(g) = \min \left\{v_{P}(f), v_{P}(g)\right\}$ as $v_{P}(f) = \infty$. A similar argument holds when $g = 0$.
  Hence we may assume $f \neq 0$ and $g \neq 0$, furthermore we can assume without loss of generality that $v_{P}(g) \geq v_{P}(f)$. Since $f \neq 0$ and $g \neq 0$ there exists units $u, v$ such that $f = u t^{v_{P}(f)}$ and $g = v t^{v_{P}(g)}$, then:
  \begin{equation*}
    f + g = (u + v t^{v_{P}(g) - v_{P}(f)}) t^{v_P(f)}.
  \end{equation*}
  Hence $v_{P}(f + g) \geq v_{P}(f) = \min \left\{v_{P}(f) v_P(g)\right\}$ as $(u + v t^{v_P(g) - v_P(f)})$ is a unit in $\mathcal{O}_P(\mathcal{X})$ since
  \begin{equation*}
    (u + v t^{v_P(g) - v_P(f)})(P) = u(P) + v(P) t^{v_{P}(g) - v_P(f)}(P) = u(P) \neq 0
  \end{equation*}
  because $u \in \mathcal{O}_{P}(\mathcal{X})^{*}$. Continuing with Assertion \ref{thm:vp_is_dv:4}, we are once again able to assume that $f \neq 0$ and $g \neq 0$, then $fg = u t^{v_{P}(f)} \cdot v t^{v_{P}(g)} = u v \cdot t^{v_{P}(f) + v_{P}(g)}$, and hence $v_{P}(fg) = v_{P}(f) + v_{P}(g)$. Finishing the proof with Assertion \ref{thm:vp_is_dv:5}, we consider the polynomial $f - \lambda g \in \mathcal{O}_{P}(\mathcal{X})[\lambda]$, then:
  \begin{equation*}
    f - \lambda g = u t^{v_{P}(f)} + \lambda vt^{v_P(g)} = (u + \lambda v) t^{v_{P}(f)}
  \end{equation*}
  However when $\lambda = -uv^{-1}$ then $(u + \lambda v) = 0$, and hence $v_{P}(f - \lambda g) = \infty$.
\end{proof}
\newpage

\section{Transmission and Nearest Neighbour Decoding}\label{sec:transmission_and_nearest_neighbour_decoding}
We will now explain how one can use a $[n,k,d]_{q}$ code $\mathcal{C}$ when transmitting data through a noisy channel. There are many different assumptions one can impose on the distribution of this noise, we will however not consider this in this project. \\
Given some $x \in \mathbb{F}_{q}^{k}$, which we shall refer to as the \textit{message}, we \textit{encode} said message by computing the corresponding codeword $c := x^{T}G$, please note the linear transformation $T: \F^k_q \to \mathcal{C}$, defined as $x \mapsto x^{T}G$, is injective since the rows of $G$ forms a $\mathbb{F}_{q}$-basis of $\mathcal{C}$. Since $c \in \mathbb{F}_q^{n}$, with $n \geq k$, more data has to be transmitted. However we have added redundant information, which will hopefully help correct any errors induced by the noise in the channel.
\\
\\
Since the codeword $c \in \mathcal{C}$ is subjected to random noise, during transmission, we receive some $y \in \mathbb{F}_q^n$, which is not necessarily equal to $c$.
Now the main question is: How does one get back the original $c$ or at least a good estimate of $c$ say $\hat{c}$?
Multiple of such estimates exists, and the best choice depends highly on the distribution of the random noise. However, we will only consider the perhaps most intuitive, named \textit{nearest neighbour decoding}, where we chose the estimate:
\begin{equation*}
   \hat{c}_{\d} = \underset{c \; \in \; \mathcal{C}}{\arg\min} \; \d(c, y).
\end{equation*}
Please note that this estimate might not be unique. Finally, we can obtain an estimate $\hat{x}$ of $x$, by \textit{decoding} $\hat{c}$. The decoding of $\hat{c}$ is often highly dependent on the specific code, and we will not discuss it in this general setting.

Continuing our disposition on nearest neighbor decoding, we prove the following theorem:
\begin{theorem}\label{thm:disjoint_closed_balls}
  Let $\mathcal{C}$ be a $[n, k, d]_{q}$ code, $r = \floor{\frac{d-1}{2}}$, and $c, c' \in \mathcal{C}$ such that $c \neq c'$ then
  \begin{equation*}
     \overline{B_{r}}(c) \cap \overline{B_r}(c') = \emptyset
  \end{equation*}
  where $\overline{B_{r}}(x) = \left\{y \in \mathbb{F}_q^n \middle| d(x, y) \leq r\right\}$.
\end{theorem}
\begin{proof}
  Assume for the sake of contradiction that $\overline{B_{r}}(c) \cap \overline{B_r}(c') \neq \emptyset$, then pick $x \in \overline{B_{r}}(c) \cap \overline{B_r}(c')$. Now since $\d$ complies with the triangle inequality confer Proposition \ref{prop:hamming_metric}\ref{prop:hamming_metric_triangle_inequality}, we have:
  \begin{equation*}
    \d(c, c') \leq \d(c, x) + \d(x,c') \leq 2r
  \end{equation*}
  but $2r = 2 \floor{\frac{d-1}{2}} \leq d - 1$, which contradicts the assumption that $d$ is the minimum distance of $\mathcal{C}$.
\end{proof}

\begin{remark}\label{rem:maximum_number_of_correctable_errors}
A natural implication of Theorem \ref{thm:disjoint_closed_balls} is that if a maximum of $r = \floor{\frac{d-1}{2}}$ entries in $c$ are corrupted during transmission, and we receive $y \in \mathbb{F}^k_q$ then $\d(y, c) \leq r$. Therefore, $\hat{c}_{\d} = c$ since $\overline{B_{r}}(c) \cap \overline{B_r}(c') = \emptyset$ for all $c' \in \mathcal{C} \setminus \left\{c\right\}$. From this it follows that a $[n, k, d]_{q}$ code $\mathcal{C}$ can correct $\floor{\frac{d - 1}{2}}$ errors. However we can detect $d-1$ errors, since the received vector, will not be a codeword in this case.
\end{remark}

The contents of Remark \ref{rem:maximum_number_of_correctable_errors} indicate that when we are interested in finding a $[n, k]_{q}$ code $\mathcal{C}$, with the best error correcting capabilities then we need to find one with a high minimum distance.
However, upper bounds on the minimum distance $d$ given $n$ and $k$ do exist, a one of which will be presented in the following section.

\section{Bounds on the Parameters of Linear Codes}\label{sec:bounds}
In this section we will introduce a few select bounds for the parameters of codes, namely the Singleton, Gilbert and asymptotic Gilbert-Varshamov bound. We will start by stating and proving the following lemma, its proof will be based upon the ideas presented in the proof of Theorem 2 found in \cite{the_singleton_bound_and_rs_code}. % FIXME: Find en anden kilde

\begin{proposition}\label{prop:d-1_distinct_columns_are_linearly_independent}
  Let $\mathcal{C}$ be a $[n, k, d]_q$ code, with parity check matrix $H$ then any $d-1$ columns of $H$ are $\mathbb{F}_{q}$-linearly independent.
\end{proposition}

\begin{proof}
  Let $h_1, h_{2}, \ldots, h_n$ be the columns of $H$, assume for the sake of contradiction that there exists distinct indices $i_1, i_2, \ldots, i_{d-1}$ such that $h_{i_{1}}, h_{i_{2}}, \ldots, h_{i_{d-1}}$ are $\mathbb{F}_{q}$-linearly dependent.
  We will now construct a $c \in \mathcal{C} \setminus \left\{0\right\}$ with $\wt(c) \leq d - 1$. We start by setting $c_{k} = 0$ for all $k \not \in \left\{i_1, i_2 \ldots, i_{d-1}\right\}$. Now as $h_{i_{1}}, h_{i_{2}}, \ldots, h_{i_{d-1}}$ are $\mathbb{F}_{q}$-linearly dependent, there exists $c_{i_{1}}, c_{i_{2}}, \ldots, c_{i_{d-1}} \in \mathbb{F}_{q}$, not all $0$, such that
  \begin{equation*}
    \sum^{n}_{j = 1} c_{j}h_{j} = \sum^{d - 1}_{j = 1} c_{i_{j}}h_{i_j} = 0 \implies c \in \mathcal{C}.
  \end{equation*}
  Which is a contradiction since $\wt(c) \leq d - 1$.
\end{proof}
\begin{remark}\label{rem:minimum_distance_coresponds_to_minimum_number_of_linearly_dependent_columns}
Another perhaps more natural connection between the minimum distance of a $[n, k]_{q}$ code $\mathcal{C}$ and one of its  parity check matrices $H$, is that the minimum distance of $\mathcal{C}$ corresponds to the minimum number of linearly dependent columns of $H$. This can be seen easily as the minimum distance of $\mathcal{C}$ corresponds to the minimum weight of $\mathcal{C}$.
\end{remark}
Using Proposition \ref{prop:d-1_distinct_columns_are_linearly_independent} we can now state and prove our first upper bound on the parameters of a $[n, k, d]_{q}$ code.

\begin{corollary}[Singleton Bound]\label{cor:singleton_bound}
  Let $\mathcal{C}$ be a $[n, k, d]_q$ code then $d-1 \leq n - k$.
\end{corollary}

\begin{proof}
  Let $H$ be a parity check matrix of $\mathcal{C}$, then $d - 1 \leq \rank(H)$ as any collection of $d-1$ columns of $H$ are $\mathbb{F}_{q}$-linearly independent, confer Proposition \ref{prop:d-1_distinct_columns_are_linearly_independent}. The rest follows from Lemma \ref{lem:H_has_rank_n-k} as $\rank(H) = n -k$.
\end{proof}

\begin{remark}\label{rem:alternative_formulation_of_singleton_bound}
  Rearranging the Singleton bound to obtain $d + k \leq n + 1$ the trade-off between minimum distance and dimension of a code, becomes even more self-evident, as such we can either wish to find a code with a high minimum distance $d$ or a high dimension $k$, relative to the length of the code, but not both!
\end{remark}

\begin{definition}\label{def:MDS_code}
  If a $[n, k, d]_{q}$ code $\mathcal{C}$ meets the singleton bound, that is $d - 1 = n - k$, it is called a \textit{maximum distance seperable (MDS)} code.
\end{definition}

We have actually already seen a MDS code in Example \ref{exmp:repetition_code}. We will now introduce a more interesting example of a MDS code, based on elements of \cite{alg_geom_codes}[Subsection 1.2.1] as well as \cite{notes_on_alg_geom_codes}[Chapter 2].

\begin{example}\label{exmp:rs_codes}[Reed-Solomon Codes]
  Let $q$ be a prime power, and fix $n, k \in \N$ such that $1 \leq k \leq n \leq q$. Since $n \leq q$ we can chose $P_{1}, P_{2}, \ldots, P_{n} \in \F_{q}$ all distinct and define the set $\mathcal{P} := \{P_{1}, P_{2}, \ldots, P_{n}\}$. Finally, we define the vector space
  \begin{equation*}
    L_{k} := \{F \in \F_{q}[X] \;\vert\; \deg(F) \leq k - 1\}.
  \end{equation*}
  Please note that $0 \in L_{k}$, as we use the convention that $\deg(0) = 0$, confer remark \ref{rem:deg_0}.
  Then the evaluation map:
  \begin{align*}
    \ev\colon L_{k} \to \F_{q}^{n}, \quad F \mapsto (F(P_{1}), F(P_{2}), \ldots, F(P_{n}))
  \end{align*}
  is injective, since $\ev$ is a linear map and $F \in \F_{q}[X] \backslash \{0\}$ can have at most $\deg(F)$ roots, so $\ev(F) = 0$ if and only if $F \equiv 0$. The linear code $\mathcal{C} := \ev(L_{k})$, is called the Reed-Solomon code of degree $k$, we will show that $\mathcal{C}$ is a $[n, k, n - k + 1]_{q}$ code.
  Since $1, X, \ldots, X^{k - 1}$ form a basis of $L_{k}$, we see that $\C$ has dimension $k$, and that
  \begin{equation*}
    G = \begin{bmatrix}
          1 & 1 & \cdots & 1\\
          P_{1} & P_{2} & \cdots & P_{n} \\
          \vdots & \vdots & \ddots & \vdots \\
          P_{1}^{k - 1} & P_{2}^{k - 1} & \cdots & P_{n}^{k -  1}
    \end{bmatrix}
  \end{equation*}
  is a generator matrix for $\C$. \\
  Let $F \in L_{k}$ and $c := (F(P_1), F(P_2), \ldots, F(P_{n}))$ then $F$ vanishes at $n - \wt(c)$ of the points, therefore we have $n - \wt(c) \leq \deg(F) \leq k - 1$.
  In the special case $\wt(c) = d$ and $d$ is the minimum distance of $\mathcal{C}$, we see that $n - d \leq k - 1$ from rearranging this inequality we obtain $n - k \leq d - 1$. Combining this with the Singleton bound (Corollary \ref{cor:singleton_bound}) we obtain $n - k = d - 1$, and thus $\mathcal{C}$ is a MDS code, with minimum distance $n - k + 1$.
\end{example}
% TODO: This is of cause a nice property to have, however these codes aren't the solution to all of our problems as we require $n \leq q$, so their lengths are limited.
Later on we will show that Reed-Solomon codes are a special case of what will be our main object of study, namely algebraic geometry codes.

\subsection{The Assymptotic Gilbert-Varshamov Bound}
The following subsection is based on \cite{huffman}[Section 2.1, 2.8 and 2.10] and \cite{notes_on_alg_geom_codes}[Chapter 5]. We will start by proving a few technical results before moving onto to the Gilbert bound. Following this we introduce some new notation before reaching the main result of this subsection, namely the asymptotic Gilbert-Varshamov bound.

\begin{lemma}\label{lem:number_of_elements_in_sphere}
  Let $x \in \mathbb{F}_q^n$, then $\abs{\overline{B_{r}}(x)} = V_{q}(n, r)$ where $V_{q}(n, r) = \sum_{i=0}^r \binom{n}{i} (q - 1)^{i}$.
\end{lemma}
\begin{proof}
  Let $S_i(x) = \left\{x' \in \mathbb{F}_q^n \middle| \d(x, x') = i\right\}$ for $i = 0, 1, \ldots, r$. Then $S_0(x), S_1(x), \ldots, S_{r}(x)$ are all finite and pairwise disjoint and $\overline{B_r}(x) = \bigcup_{i = 0}^r S_i(x)$. Combining these facts we see that:
  \begin{equation}\label{eq:cardinality_of_disjoint_sets}
    \abs{\overline{B_r}(x)} = \sum_{i = 0}^r \abs{S_i(x)}.
  \end{equation}
  Additionally we have that $\abs{S_i(x)} = \binom{n}{i}(q - 1)^i$, since $x' \in S_{i}$ differ from $x$ in exactly $i$ out of $n$ entries. The result follows immediately by combining this with Equation \eqref{eq:cardinality_of_disjoint_sets}.
\end{proof}

Moving forward we follow the convention of \cite{huffman} and let $B_q(n, d)$ be the maximum number of codewords in a linear code of length $n$ and minimum distance at least $d$.

% We will now state the following Theorem 2.1.7 from \cite{huffman} without proof.

% TODO: Look up the proof before the exam.

\begin{theorem}\label{thm:covering_radius}
  If $\mathcal{C}$ is linear code over $\mathbb{F}_{q}$ of length $n$, with minimum distance $d$ and $B_q(n, d)$ codewords then:
  \begin{equation*}
    \mathbb{F}_q^n = \bigcup_{c \in \mathcal{C}} \overline{B_{d - 1}}(c)
  \end{equation*}
\end{theorem}
The proof of is due to \cite{covering_distance_of_linear_code}.
\begin{proof}
  If there exists an $x \in \mathbb{F}_{q}^{n} \not \in \bigcup_{c \in \mathcal{C}} \overline{B_{d - 1}}(c)$. Then $\mathcal{C}' := \mathcal{C} + \Span\{x\}$ is another linear code, such that $\abs{\mathcal{C}'} > \abs{\mathcal{C}}$. We will show that $\mathcal{C}'$ has minimum distance $d$ which in turn contradicts that $\mathcal{C}$ had $B_{q}(n, d)$ codewords. \\
  Assume for the sake of contradiction, that $\mathcal{C}'$ has minimum distance less than $d$, then there exists a $c' \in \mathcal{C}'$ such that $\wt(c') \leq d - 1$. Since $c' \in \mathcal{C}'$ we may write $c' = c + ax$ for some $c\in \mathcal{C}$ and $a \in \mathbb{F}_{q}$. Furthermore, we have that $a \not= 0$, as $\wt(c') \leq d - 1$ implies that $c' \not\in \mathcal{C}$. In addition, we have that $\wt(b y) = \wt(y)$ for all $b \in \mathbb{F}_q$ and $y\in \mathbb{F}_q^{n}$, since $\mathbb{F}_q$ is a domain. Combining the earlier observations we get that:
   \begin{equation*}
     d - 1 \geq \wt(c') = \wt(-a^{-1}c') = \wt(-a^{-1}c - x) = \d(-a^{-1}c, x).
   \end{equation*}
   However this is a contradiction since this means that $x \in \overline{B_{d - 1}}(-a^{-1}c)$ after all.
 \end{proof}

\newpage
We are now able to state and prove the Gilbert bound which we will use to prove the asymptotic Gilbert-Varshamov bound.
\begin{corollary}[Gilbert bound]\label{cor:gilbert_bound}
  Suppose $V_{q}(n, r)$ is defined as in Lemma \ref{lem:number_of_elements_in_sphere} then:
  \begin{equation*}
    B_q(n, d) \geq \frac{q^{n}}{V_{q}(n, d - 1)}
  \end{equation*}
\end{corollary}

\begin{proof}
  Let $\mathcal{C}$ be a linear code of length $n$, with minimum distance $d$ and $B_q(n, d)$ codewords, then $\mathbb{F}_q^n = \bigcup_{c \in \mathcal{C}} \overline{B_{d - 1}}(c)$ by Theorem \ref{thm:covering_radius}, combining this with Lemma \ref{lem:number_of_elements_in_sphere}, we get:
  \begin{equation*}
    q^{n} = \abs{\mathbb{F}_q^n} \leq \sum_{c \in \mathcal{C}} \abs{\overline{B_{d-1}}(c)} = B_q(n,d) V_q(n, d - 1). \qedhere
  \end{equation*}
\end{proof}

Continuing we define the following function which will be used in the asymptotic Gilbert-Varshamov bound.
\begin{definition}
  Let $q \in \mathbb{N}$ such that $q \geq 2$, then the function $H_q: [0, (q - 1)/q] \to \mathbb{R}$ defined as $H_q(0) = 0$ and $H_q(x) = x \log_{q}(q-1) - x \log_{q}(x) - (1-x)\log_q(1 - x)$ otherwise is called the \textit{$q$-ary entropy function}.
\end{definition}
We will now state Lemma 2.10.3, from \cite{huffman}, but we will omit its computation heavy proof.
\begin{lemma}\label{lem:entropy_limit}
  Let $\delta \in [0, (q - 1) / q]$ where $q \in \mathbb{N}$ such that $q \geq 2$, then:
  \begin{equation*}
    \lim_{n \to \infty} \frac{1}{n} \log_q(V_q(n, \floor{\delta n})) = H_q(\delta).
  \end{equation*}
\end{lemma}

Moving on we introduce two invariants which can be used to gauge the quality of a code.
\begin{definition}
Let $\mathcal{C}$ be a $[n, k, d]_q$ code, then the invariants $R := k / n$ and $\delta := d / n$ is called the \textit{transmission rate} and \textit{relative distance} of $\mathcal{C}$ respectively.
\end{definition}
Following the definition a natural question arise, namely why would one consider the invariants $R$ and $\delta$ instead of using $n, k$ and $d$ directly, when gauging the quality of a $[n,k,d]_q$ code?
To answer this consider the repetition code of length $n$, as seen in Example \ref{exmp:repetition_code}, when $n$ increases then so will the minimum distance $d$ as $d = n$, however $\delta$ will stay constant.
In addition to this we are still only able to encode a single symbol hence we are sacrificing transmission rate. Hence, $R$ and $\delta$ better reflect the properties of a code, especially for large $n$.

\begin{remark}\label{rem:upper_bound_of_sum}
  Let $\mathcal{C}$ be a $[n, k, d]_{q}$ code with transmission rate $R$ and relative minimum distance $\delta$, then $d + k \leq n + 1$ by Remark \ref{rem:alternative_formulation_of_singleton_bound}, dividing by $n$ we get that
  \begin{equation*}
    \delta + R \leq 1 + \frac{1}{n}.
  \end{equation*}
  Thus a long code, with good parameters have $\delta + R$ close to $1$.
\end{remark}

\newpage
Remember that $B_q(n, d)$ is the maximum number of codewords of a linear code $\mathcal{C}$, with length $n$ and minimum distance at least $d$. Since $\mathcal{C}$ is a subspace of $\mathbb{F}_q^{n}$, we have that $\dim(\mathcal{C}) = \log_q(B_q(n, d))$. Thus, it makes sense to define what we shall refer to as the \textit{maximal rate}:
\begin{equation*}
  R^{*}(n, d) = \frac{\log_{q}(B_{q}(n, d))}{n}.
\end{equation*}
We will now investigate the behavior of $R^{*}(n, d)$ as $n \to \infty$, for this we define, the function $\alpha_q^{lin}: [0, 1] \to [0, 1]$ as a function of a relative distance $\delta$, specifically $\delta \mapsto \underset{n \to \infty}{\lim \sup} \;R^{*}(n, \delta n)$. Finally, we can state and prove the asymptotic Gilbert-Varshamov\footnote{The Varshamov bound is another bound for the parameters of linear codes, which can be used to prove the theorem as well.} bound.
\begin{theorem}[Asymptotic Gilbert-Varshamov bound]\label{thm:gilbert_varshamov}
  Let $0 \leq \delta \leq (q - 1)/q$, then:
  \begin{equation*}
  \alpha_q^{lin}(\delta) \geq 1 - H_q(\delta).
  \end{equation*}
\end{theorem}
\begin{proof}
  First note that $B_q(n, \delta n) = B_q(n, \ceil{\delta n})$. Since $B_{q}(n, d)$ is the maximum number of codewords of a code of length $n$ with minimum distance at least $d$ and the minimum distance of a linear code is always a natural number.
  % NOTE: Hvis du undre dig over hvor uligheden kommer fra så test med $\delta n \in \mathbb{Z}$.
  Thus, we have
  \begin{align*}
\alpha_q^{lin}(\delta) = \underset{n \to \infty}{\lim \sup} \; \frac{\log_q(B_q(n, \ceil{\delta n}))}{n} &\geq \underset{n \to \infty}{\lim \sup} \; \frac{\log_{q}\left(\dfrac{q^{n}}{V_{q}(n, \ceil{\delta n} - 1)}\right)}{n}\\ &\geq \underset{n \to \infty}{\lim \sup} \; 1 - \frac{\log_{q}({V_{q}(n, \floor{\delta n})}}{n}
  \end{align*}
  where the first inequality follows from the Gilbert bound (Corollary \ref{cor:gilbert_bound}) and the last inequality follows since $\ceil{\delta n} - 1 \leq \floor{\delta n}$ implies $\frac{q^{n}}{V_{q}(n, \ceil{\delta n} - 1)} \geq \frac{q^{n}}{V_{q}(n, \floor{\delta n})}$. The rest follows by applying Lemma \ref{lem:entropy_limit}, as
  \begin{equation*}
  \lim_{n \to \infty}\frac{\log_{q}({V_{q}(n, \floor{\delta n})}}{n} = H_q(\delta) \implies \underset{n \to \infty}{\lim \sup} \;\frac{\log_{q}({V_{q}(n, \floor{\delta n})}}{n} = H_q(\delta). \qedhere
  \end{equation*}
\end{proof}




%In this chapter we study generic decoding algorithms (meaning they work for arbitrary codes), we study these algorithms as the provide generic attacks on the McElice public key cryptography scheme.
%We will exclusively be working nearest neighbor decoding, meaning for a received message $y = c + e \in \mathbb{F}_q^{n}$, where $e \in \mathbb{F}_q^n$ is a random error vector and $c$ is a codeword in a $[n, k]_{q}$ code $\mathcal{C}$. We then estimate $c$ as $\hat{c} := \underset{c \in \mathcal{C}}{\arg \min} (d(c, y))$, equivalently we can estimate the error $e$ as $\hat{e} := \underset{e \in \mathbb{F}_q^n \\ y - e \in \mathcal{C}}{\arg \min} \wt(e)$, we will refer to both of these as nearest neighbor estimates.

%If we can find \textcolor{blue}{efficient} algorithm for solving the McElice problem, see Problem \ref{prob:McElice}, then we would have broken the McElice PKCS, hence we will focus on the cases where we know a generator matrix of the code.
