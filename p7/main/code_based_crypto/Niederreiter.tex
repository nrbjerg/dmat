\section{The Niederreiter Public Key Cryptosystem}

Suppose $\mathcal{C}$ is a $[n, k, d]_q$ code with parity check matrix $H$ and an efficient $t$-error correcting decoding algorithm $dec_{\mathcal{C}}$. Furthermore let $S \in GL_{(n - k) \times (n - k)}(\mathbb{F}_q)$ and $P \in \mathbb{F}_q^{n \times n}$ be a random permutation matrix. The Neiderreiter PCKS (based on $\mathcal{C}$), which allows for encryption of a plain message $e \in \mathbb{F}_q^n$ with $\wt(e) = t$, is described below in Algorithm \ref{alg:Neiderreiter}.

\begin{algorithm}
\caption{The Neiderreiter PKCS}\label{alg:Neiderreiter}
\begin{algorithmic}
  \State \textbf{private key} $(Dec_{\mathcal{C}}, S, P)$
  \State \textbf{public key} $(H' = S \cdot H \cdot P, t)$
  \\
  \Procedure{Neiderreiter Encryption} {$e$: plain message with $\wt(e) = t$}
    \State \Return $H'e$
  \EndProcedure \\
  \Procedure{Neiderreiter Decryption} {$y$: received ciphertext}
    \State Find $z \in \mathbb{F}_q^n$ such that $Hz = S^{-1}y$ using linear algebra.
    \State $c \gets Dec_{\mathcal{C}}(z)$
    \State \Return $c P^{-1}$
  \EndProcedure
\end{algorithmic}
\end{algorithm}

Finding a $z \in \mathbb{F}_q^{n}$ such that $Hz = S^{-1}y$ is simply a matter of solving a linear system. Additionally since $S^{-1}y = S^{-1}SHPe = HPe$ we see that $c := Dec_{\mathcal{C}}(z)$ is the closest codeword to $HPe$ so $e = z - c P^{-1}$.

\subsection{The Equivalence Between the Neiderreiter and the McElice PCKS}%

Consider the $[n, k, d]_{q}$ code $\mathcal{C}$ with generator matrix $G$ and parity check matrix $H$. We demonstrate that the McElice and Neither cryptosystems based on $\mathcal{C}$ have an equivalent level of security. That is if a there exists an efficient attack on the McElice PCKS (based on $\mathcal{C}$), then there exists an efficient attack on the Neither PCKS (based on $\mathcal{C}$).

Hence we let $G'$ and $H'$ be the public keys of the McElice and Neither PCKS respectively.

Suppose we have a message $m \in \mathbb{F}_q^k$, we encrypt the message a obtain $y \in \mathbb{F}_q^{n}$, using Algorithm \ref{alg:McElice}. That is:
\begin{equation*}
  y = m^T G' + e
\end{equation*}
where $e \in \mathbb{F}_q^n$ with $\wt(e) = t$. Given $G'$ we may obtain a parity check matrix $H'$ for the code generated confer Theorem \ref{thm:from_generator_matrix_to_parity_check_matrix}. Multiplying by $(H')^T$ we get:
\begin{equation}\label{eq:Neiderreiter_and_McElice}
  y(H')^T = mG'(H')^T + e (H')^T = e (H')^T
\end{equation}
where the last equality follows as $G'(H')^T = 0$. Additionally since $y$ and $H'$ are public, the righthand side of Equation \eqref{eq:Neiderreiter_and_McElice} can easily be computed. Furthermore since $\wt(e) = t$, we see that we may compute the error $e$ efficiently provided we have an efficient attack on the Neiderreiter PCKS. Hence an efficient attack on the Neiderreiter PCKS would lead to an efficient attack on the corresponding McElice PCKS with very little overhead.

Conversely assume that we have a message $e \in \mathbb{F}_q^{n}$ such that $\wt(e) = t$. If we encrypt the message to obtain $y \in \mathbb{F}_q^{(n - k)}$, using Algorithm \ref{alg:Neiderreiter}. That is:
\begin{equation*}
  y = H'e
\end{equation*}
Again we may obtain generator matrix $G'$ of the code $\Null(H')$, since the null space of a matrix is invariant under row operations, thus we may apply Theorem \ref{thm:from_generator_matrix_to_parity_check_matrix}. Using basic linear algebra one my find a vector $z \in \mathbb{F}_q^n$ with $\wt(z) \geq t$ such that $y = H'z$ after all $d$ is the minimum number of linearly independent columns of $H'$ and $t \leq \floor{\frac{d - 1}{2}}$. Hence:
\begin{equation*}
  y = H'z \text{ and } z = yG' + e
\end{equation*}
Hence $e$ could be extracted efficiently provided that there exists an efficient algorithm for breaking the McElice PCKS.


\section{Advantages and Drawbacks of McElice and Neiderreiter}%
In this final subsection we will briefly discuss some of the advantages and drawbacks of using each system. Both compared to each other and compared to traditional public key cryptography systems. The results are summarized in Table \ref{tab:pros_and_cons}. The encryption procedure of the McElice PCKS is very efficient additionally there exists no well known quantum algorithm for breaking the McElice PCKS. The primary con of the McElice PCKS is the large key size, for example the original proposal by McElice, used a $[1024, 524]_2$ classical Goppa code, see Definition \ref{def:classical}. Due to the dimensions of the code the public key was roughly $524 \cdot 1024 = 536576$ bits or about $67.1$ KB \textcolor{blue}{while yielding a security level of about $65$ bits (remember to do the actual calculation). Meaning an attacker would have to perform $2^{65}$ operations to break the encryption.}. In comparison using RSA a public key size of only $3072$ bits is sufficient to yield a security level of $128$ bits, refer \cite{nist_recomendations_for_key_management}[Table 2].

For this reason much research has been done with the objective of lowering the public key size. However even though many of these proposals have succeeded in lowering the public key size, they often come with security issues.

The Neiderreiter PCKS has similar drawbacks as the McElice PCKS, in that the key size is very large compared to traditional public key cryptography systems, like RSA. However it has an additional drawback compared to the McElice PCKS, namely that the message has to have weight $t$ while having length $n$. Where as the only restriction on the message in the McElice PCKS is that it should have length $n$. On the otherhand one of the advantages of using the Neiderreiter PCKS is that it offers a smaller key size. Since the parity check matrix $H'$ may be published in systematic form, that is $H' = [A | I_{(n - k) \times (n - k)}]$, while keeping the security level the same, we prove this below in Proposition \ref{prop:H_can_be_in_standard_form}. We will however first need a small lemma:

\begin{lemma}\label{lem:equal_syndrome_and_equal_weight}
  Let $H$ be parity check matrix for the $[n, k, d]_q$ code $\mathcal{C}$, $t \leq \floor{\frac{d - 1}{2}}$ and $x, x' \in \mathbb{F}_q^n$. Then $\wt(x) = \wt(x') = t$ and $S_H(x) = S_H(x')$ implies $x = x'$.
\end{lemma}
\begin{proof}
  By Lemma \ref{lem:syndrome_is_the_same_iff_they_are_in_the_same_coset} we have $x = x' + c$ for some $c \in \mathcal{C}$. Thus $c = x - x'$ which implies:
  \begin{equation*}
    \wt(c) = \wt(x - x') \leq 2t < d
  \end{equation*}
  since $\wt(x) = \wt(x') = t$. Thus $c$ must equal zero.
\end{proof}

\begin{proposition}\label{prop:H_can_be_in_standard_form}
  Let $H$ be a parity check matrix for the $[n, k, d]_q$ code $\mathcal{C}$. Furthermore let $H' := UH$ be a systematic parity check matrix of $\mathcal{C}$, then any attack able to break the scheme using $H'$ is able to break a scheme using $H$.
\end{proposition}
\begin{proof}
  Let $W_t = \left\{x \in \mathbb{F}_q^n | \wt(x) = t\right\}$ with $t \leq \floor{\frac{d - 1}{2}}$. Then $S_H$ and $S_{H'}$, restricted to $W_{t}$, are injective by Lemma \ref{lem:equal_syndrome_and_equal_weight}. Assume that $\phi$ breaks the system using $H'$, that is $x = \phi(y) = S_{H'}^{-1}(y)$ for all $y \in S_{H'}(W_t)$. \\
  Next suppose $y \in S_{H}(W_t)$, then $y = S_{H}(x) = Hx$ for some $x \in W_t$. Hence:
  \begin{equation*}
    Uy = U S_H(x) = U H x = S_{H'}(x) \in S_{H'}(W_t)
  \end{equation*}
   and $\phi(Uy) = x$. That is $\phi$ can be used to break the system using the parity check matrix $H$, we note that $U$ can be obtained by performing elementary row operations on $H$.
\end{proof}

Since $H$ is allowed to be in standard form, may simply publish the first $k$ columns of $H$ assuming $H$ is in systematic form, allowing for a much smaller public key size at least when $k$ is relatively large compared to $n$. As an example consider McElices original proposal which used a $[1024, 524]_2$ classical Goppa code, meaning the public key of the equivalent neither system, could be published using $(1024-524) \cdot 524 = 26200$ or about $32.8$ KB, which is about half the size of the public key in the equivalent McElice PCKS. Finally the contents of the discussion is summarized in Table \ref{tab:pros_and_cons}.
\begin{table}[H]
    \centering
    \begin{tabular} {||c|c|c||}
        \hline
        \textbf{System} & \textbf{Advantages} & \textbf{Drawbacks} \\
        \hline
        McElice & \makecell{Fast encryption\\No known efficient quantum attacks}& Large key size (versus e.g. RSA) \\
        \hline

        Neiderreiter & \makecell{Smaller key size (versus McElice) \\ No known efficient quantum attacks} & \makecell{Large key size (versus e.g. RSA)\\ Messages must be of weight $t$}\\
        \hline
    \end{tabular}
    \caption{Advantages and drawbacks of the McElice and Neiderreiter PCKS.}
    \label{tab:pros_and_cons}
\end{table}
