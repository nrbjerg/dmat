\section{Information Sets}\label{sec:information_set_decoding}
The following section will be based on \cite{information_set_decoding}[Sections 3-6], \cite{lee_brickell} and \cite{notes_on_code_based_cryptography}[Chapter 3]. We start by introducing some basic notation. Let $I \subseteq \left\{1, \ldots, n\right\}$ be non-empty. We will let $G_I$ denote the matrix which consists of the columns of $G$ such that the $i$th column is in $G_I$ if and only if $i \in I$ and we will similarly let $y_I$ denote the restriction of $y \in \mathbb{F}_q^n$ to the coordinates indexed by $I$. Following the convention of \cite{notes_on_code_based_cryptography} we let $\support(y) = \left\{i \in \left\{1, \ldots, n\right\} | y_i \neq 0 \right\}$.

\begin{proposition}\label{prop:multiplying_by_information_matrix_generates_the_same_code}
  Let $G \in \mathbb{F}_q^{k \times n}$ be the generator matrix of some code $\mathcal{C}$, and $A \in GL_k(\mathbb{F}_q)$, then $A^{-1}G$ is a generator matrix of $\mathcal{C}$ as well.
\end{proposition}
\begin{proof}
  Suppose $c \in \mathcal{C}$ then $c = m^TG$ for some $m \in \mathbb{F}_q^{k}$, in addition we may solve the equation $x^TA^{-1} = m^{T}$ for $x \in \mathbb{F}^k_q$. Hence $c = x^TA^{-1}G$. We also note that since $A^{-1}$ is invertible, $\rank(A^{-1}G) = \rank(G) = k$, hence the $k$ rows of $A^{-1}G$ are $\mathbb{F}_q$-linearly independent.
\end{proof}

\begin{definition}
  Let $\mathcal{C}$ be a $[n, k]_q$ code with generator matrix $G \in \mathbb{F}_q^{k \times n}$ and $I \subseteq \left\{1, \ldots, n\right\}$ such that $\abs{I} = k$. If $G_I$ is invertible, then the $I$-indexed entries of $mG^{-1}_I G$ is called \textit{information symbols} and $I$ an \textit{information set}.
\end{definition}

\begin{example}\label{exmp:information_sets_of_hamming}
  Consider the $[7, 4, 3]_2$ hamming code $\mathcal{H}_{3}$. That is the code with parity check matrix:
  \begin{equation*}
    H = \begin{bmatrix}
          1 & 1 & 1 & 0 & 1 & 0 & 0 \\
          1 & 1 & 0 & 1 & 0 & 1 & 0 \\
          1 & 0 & 1 & 1 & 0 & 0 & 1
        \end{bmatrix}
  \end{equation*}
  By Theorem \ref{thm:from_generator_matrix_to_parity_check_matrix}.
  \begin{equation*}
    G = \begin{bmatrix}
1 & 0 & 0 & 0 & 1 & 1 & 1 \\
0 & 1 & 0 & 0 & 1 & 1 & 0 \\
0 & 0 & 1 & 0 & 1 & 0 & 1 \\
0 & 0 & 0 & 1 & 0 & 1 & 1 \\
        \end{bmatrix}
  \end{equation*}
  is a parity check matrix for $\mathcal{H}_{3}$. Then obviously $I = \left\{1, 2, 3, 4\right\}$ is an information set since $G_{I}$ is invertible. A less obvious information set would be $J = \left\{1, 3, 6, 7\right\}$ since:
  \begin{equation*}
    G_{J} = \begin{bmatrix}
                1 & 0 & 1 & 1 \\
                0 & 0 & 1 & 0 \\
                0 & 1 & 0 & 1 \\
                0 & 0 & 1 & 1 \\
              \end{bmatrix} \sim \begin{bmatrix}
                                   1 & 0 & 1 & 1 \\
                                   0 & 1 & 0 & 1 \\
                                   0 & 0 & 1 & 0 \\
                                   0 & 0 & 1 & 1
                                 \end{bmatrix} \sim I_{4 \times 4} \qedhere
  \end{equation*}
\end{example}

\begin{remark}
If $G$ is a generator matrix of $\mathcal{C}$, then so is $G^{-1}G$ and vice versa. This follows as the map $\mathbb{F}_q^{k} \ni x \mapsto x^{T} G_I^{-1} \in \mathbb{F}_q^n$ is bijective as $G_I^{-1} \in GL_n(\mathbb{F}_q)$. In addition $(G_{I}^{-1}G)_{I} = I_{k \times k}$.
\end{remark}

%\begin{definition}
%  Let $G \in \mathbb{F}_q^{k \times n}$ be the generator matrix of some code $\mathcal{C}$. The generator matrix $G$ is said to be in \textit{systematic form} if $G = \begin{bmatrix} I_k & Q \end{bmatrix}$, where $Q \in \mathbb{F}_q^{k \times (n - k)}$. Furthermore if $G$ is in systematic form and $m \in \mathbb{F}_q^k$, then the first $k$ symbols of $m^{T}G$ is called \textit{information symbols}.
%\end{definition}
The information symbols contains the information captured by $m$, and the $n-k$ symbols are the redundancy used for error correcting.

The basic idea behind information-set decoding (ISD), given a recived vector $y$ is to find an information-set $I$ such that the entries $y_i$ with $i \in I$ are error free. If $y_I$ is error free, we may obtain the we may obtain the original message as $y_IG^{-1}_I$, as $\rank(G_I) = k$. Hence we will need algorithms for computing information sets given a generator matrix $G \in \mathbb{F}_q^{k \times n}$
\subsection{Algorithms for Computing Information Sets}
We start by introducing perhaps the simplest way to generate information sets in Algorithm \ref{alg:information_set}. The idea behind the algorithm is stems from the following lemma:
\begin{lemma}\label{lem:new_information_sets}
  Let $G \in \mathbb{F}_q^{k \times n}$ be the generator matrix of code $\mathcal{C}$, and $G'$ be $G$ reduced to row echelon form. Let $I'$ be the tuple consisting of the indicies of the $k$ pivot columns of $G$ sorted with respect to the canonical order $\leq$ on $\mathbb{N}$, then any set of the form $\left\{i_1, i_2 \ldots, i_{k}\right\}$ with $i_j \in \mathbb{N}$ and $G'_{i_j,j} \neq 0$ such that $I'_j \leq i_j < I'_{j + 1}$ for all $j < k$ and $I'_k \leq i_k \leq n$.
\end{lemma}
\begin{proof}
  Suppose $\left\{i_1, i_2 \ldots, i_{k}\right\}$ is such a set, then there exists a permutation matrix such that $P \in \mathbb{F}_q^{n \times n}$ such that the pivot columns in $GP$ has indicies $i_1, i_2 \ldots, i_{k}$, since $G'_{i_{j}, j} \neq 0$. Hence the columns $g_{i_1}, g_{i_2} \ldots, g_{i_{k}}$ are $\mathbb{F}_q$-linearly independent and thus $\{i_1, i_2 \ldots, i_{n}\}$ also form an information-set.
\end{proof}

The notion of a information set will become pivotal in the algorithms to come, hence we will need a efficient way to compute these information-sets, the procedures in Algorithm \ref{alg:information_set} provides a way to do exactly this, the correctness follows immediately by Lemma \ref{lem:new_information_sets}.
\begin{algorithm}[H]
\caption{Construction of information sets using initial row reduction}\label{alg:information_set}
\begin{algorithmic}
  \Procedure{RR IS Generator}{$G$: a generator matrix of $[n,k]_q$ code, $r$: boolean value}
  \State $G' \gets G$ reduced to row echelon form
  \State $I' \gets$ The tuple consisting of the indicies of the $k$ pivot columns in $G'$
  \State $J \gets \left\{\left\{i_1, i_2 \ldots, i_{k}\right\} \subseteq \left\{1, 2, \ldots, n\right\}| G'_{i_{j},j} \neq 0, I'_{j} \leq i_{j} < I_{j + 1} \forall j < k, I'_{k} \leq i_k \leq n \right\}$
  \If{$r$}
   \Loop
     \State \textbf{yield} $I \in J$ \Comment{Chosen uniformly, with replacement}
   \EndLoop
  \Else
  \For{$I \in J$}
     \State \textbf{yield} $I$ \Comment{Chosen uniformly, without replacement}
  \EndFor
  \EndIf
  \EndProcedure
\end{algorithmic}
\end{algorithm}
\begin{remark}\label{rem:information_set_alg}
  Gaussian reduction, which is used compute $G'$ has a time complexity of $O(nk\min\{n, k\}) = O(nk^{2})$. In comparison the computational cost the other parts (looking up the pivot columns and constructing new information-sets given this information-set) of Algorithm \ref{alg:information_set}, are negitable. Hence the bulk of the computation time will be spent when computing $G'$. This means that we may allow our algorithms to use multiple different information sets, with very little additional overhead.
\end{remark}
Next we highlight a general weakness of Algorithm \ref{alg:information_set}, namely that we aren't guaranteed to get every information set.
\begin{example}\label{exmp:first_IS_algorithm}
  Consider the generator matrix $G$ of the $[7, 4, 3]_2$ hamming code $\mathcal{H}_3$ as described in Example \ref{exmp:information_sets_of_hamming}. Then using the same notation as in Algorithm \ref{alg:information_set} we see that $G' = G$. Hence $I' = (1, 2, 3, 4)$. Next we find that
  \begin{equation*}
    J = \left\{\left\{1, 2, 3, 4\right\}, \left\{1, 2, 3, 6\right\}, \left\{1, 2, 3, 7\right\}\right\}
  \end{equation*}
  Since $G_{5, 4} = 0$. However $J$ does not contain all of the information sets of $\mathcal{H}_3$, In particular $J$ does not contain $\left\{1, 3, 6, 7\right\}$ which is also an information set, confer Example \ref{exmp:information_sets_of_hamming}.
\end{example}

Perhaps one way to combat this disadvantage could be to permute the columns of $G$ via some permutation $\sigma \in S_n$, by letting:
\begin{equation*}
  P_{\sigma} = \begin{bmatrix} p_{ij} \end{bmatrix} \text{ with } p_{ij} = \begin{cases} 1 & \text{ if }  \sigma(i) = j \\ 0 & \text{ otherwise } \end{cases}
\end{equation*}
and apply the algorithm to $GP_{\sigma}$, obtaining the set $J_{\sigma}$ containing some of the information sets of $GP_{\sigma}$. Then given $I_{\sigma} \in J_{\sigma}$ we may compute an information set for $G$ via $\sigma^{-1}(I_{\sigma})$. But we digress, instead we will consider picking a subset $I$ of $\left\{1, 2, \ldots, n\right\}$ at random and checking if it forms an information set.

\begin{algorithm}[H]
\caption{Constructing Information Sets using Gaussian Elimination}\label{alg:information_set_gauss}
\begin{algorithmic}
  \Procedure{Gaussian IS Generator}{$G$: a generator matrix of $[n,k]_q$ code,
    \newline\phantom{\textbf{procedure} \textsc{Gaussian IS Generator}(}$r$: boolean value}
  \For{$I \subseteq \left\{1, 2, \ldots, n\right\}$} \Comment{Chosen uniformly, with replacement if $r$ is true.}
    \If{$G_I$ is invertible}
    \State \textbf{yield} $I$
    \EndIf
  \EndFor
  \EndProcedure
\end{algorithmic}
\end{algorithm}
\begin{remark}\label{rem:is_gauss}
  One of the simplest ways to check if $G_I$ is invertible in Algorithm \ref{alg:information_set_gauss}, is to see if it is row equivalent with a upper triangular matrix (with non-zero entries in the diagonal). This can be done using Gaussian elimination.
\end{remark}

The next algorithm is an improvement of Algorithm \ref{alg:information_set_gauss}, the main idea is to gradually construct $I$ by adding a new index at each iteration, which allows us to perform an early abort if one of the columns of $G_I$ is linearly dependent on the rest.

\begin{algorithm}[H]
\caption{Construction of information sets with early abort}\label{alg:information_set_early_abort}
\begin{algorithmic}
  \Procedure{IS Generator}{$G$: a generator matrix of $[n,k]_q$ code}
  \Loop
    \State $I \gets \left\{i\right\}$ with $i \in \left\{1, 2, \ldots, n\right\}$ \Comment{Picked uniformly}
    \While{$\abs{I} \neq k$}
       \State $J \gets \left\{j \in \left\{1, 2, \ldots, n\right\} \middle| \rank\left(\begin{bmatrix} G_I & G_{*, j} \end{bmatrix}\right) = \abs{I} + 1\right\}$
       \If{$\abs{J} \neq 0$}
          \State $j \in J$ \Comment{Picked uniformly}
          \State $I \gets I \cup \left\{j\right\}$
       \Else
         \State \textbf{break}
       \EndIf
    \EndWhile
    \If{$\abs{I} = k$}
       \State \textbf{yield} $I$
    \EndIf
  \EndLoop
  \EndProcedure
\end{algorithmic}
\end{algorithm}
Each iteration of the while loops adds (if possible) a $j \in \left\{1, 2, \ldots, n\right\}$ to $I$ such that the columns of $G_I$ and $G_{*, j}$ are linearly independent, hence if the while-loop terminates with $\abs{I} = k$, then we have found an information set. The other case is that the while-loop terminates but $\abs{I} \neq k$, in that case we performed an early abort, since we couldn't find a $j \in \left\{1, 2, \ldots, n\right\}$ such that $\rank\left(\begin{bmatrix} G_I & G_{*, j} \end{bmatrix}\right) = \abs{I} + 1$.

\subsection{Information Set Decoding}
Finally we can state our first information-set decoding algorithm.
\begin{algorithm}[H]
\caption{Plain information-set decoding}\label{alg:plain_ISD}
\begin{algorithmic}
  \Procedure{Plain ISD}{$G$: a generator matrix of $[n,k]_q$ code $\mathcal{C}$, $y$: recived word, $t$: number of errors}
  \For{$I \in \Call{IS Generator}{G, false}$}
  \State $m' \gets y_IG_I^{-1}$
  \If{$\wt(y - m'G) \leq t$}
  \State \Return $m'$
  \EndIf
  \EndFor
  \EndProcedure
\end{algorithmic}
\end{algorithm}

\begin{proof}[Proof of the correctness of Algorithm \ref{alg:plain_ISD}]
  Suppose $y = m^TG + e^T$, where $m \in \mathbb{F}_q^{k}$ and $e \in \mathbb{F}_q^n$ such that $\wt(e) = t$ with  $t \leq \floor{\frac{d - 1}{2}}$, where $d$ is the minimum distance of the code generated by $G$. Assuming that the coordinates of $y_I$ are error free, then $m' := y_IG_I^{-1}$ equals $m$ and hence $\wt(y - m'G) = t$, and we return $m$. Otherwise if $y_I$ isn't error free, then $m' \neq m$, and hence $\wt(mG - m'G) \geq 2t$ as $\mathcal{C}$ has a minimum distance of at least $2t$, combining this with the fact that $y$ differs from $mG$ in exactly $t$ positions we see if $\wt(y - m'G) \leq t$, then we know that $m'$ equals $m$.
\end{proof}
Next we will briefly discuss the expected work factor of Algorithm \ref{alg:plain_ISD}. We will assume that the indicies non zero indicies of the error vector are distributed uniformly. The probability of choosing an information-set, which is error free, is $\binom{n - t}{k} \binom{n}{k}^{-1}$, after all we have $n - t$ error free positions and $\binom{n}{k}$ choices of $k$ indicies. In addition the basic matrix inversion algorithm has a time total work factor of about $\alpha k^3$ for some small $\alpha$. Hence the total expected work factor is about:
\begin{equation*}
  W = \alpha k^3 \binom{n}{k} \binom{n - t}{k}^{-1}
\end{equation*}
The $\binom{n}{k} \binom{n - t}{k}^{-1}$ comes from the fact, that we are computing the expected value of a stochastic variable which follows a geometric distribution with success probability of $\binom{n - t}{k} \binom{n}{k}^{-1}$.

 \subsection{Lee-Brickell's Algorithm}
 The next algorithm, due to Lee and Brickell is a natural generalization of plain information set decoding, which allows for a number of errors $p$ in the recived word $y$, more specifically given an information set $I$ we allow $y_I$ to have $p$ corrupted entries.

 \begin{algorithm}[H]
 \caption{Lee-Brickell's algorithm for information set decoding}\label{alg:lee-brickell}
 \begin{algorithmic}
   \Procedure{Lee-Brickell ISD}{$G$: a generator matrix of $[n,k]_q$ code, $y$: recived word,
     \newline\phantom{\textbf{procedure} \textsc{LEE-Brickell ISD}}$t$: number of errors, $p$: an integer such that $0 \leq p \leq t$}
  \For{$I \in \Call{IS Generator}{G, false}$}
  \State $y' \gets y - y_I G_I^{-1}G$
  \For{$J = \left\{j_1, j_2 \ldots, j_{p}\right\} \subseteq I$ such that $\abs{J}=p$}
  \For{$m \in (\mathbb{F}_q^*)^{p}$}
    \State $e \gets y' - \sum_{i=1}^p m_i g_{*, j_i}$
    \If{$\wt(e) = t$}
    \State \Return $e$
    \EndIf
  \EndFor
  \EndFor
  \EndFor
  \EndProcedure
 \end{algorithmic}
 \end{algorithm}
 Since we have $\binom{k}{p}$ subsets $J$ of $I$ such that $\abs{J} = p$, we should keep the parameter $p$ relatively small compared to $k$.  \textcolor{blue}{what about the binary case}

 When we pick $J \subseteq I$ such that $\abs{J} = p$ and check if $\wt(y' - \sum_{i=1}^p m_ig_{j_{i}}) = t$ for all $m \in (\mathbb{F}_q^{*})^{p}$, we are checking if there is exactly $p$ symbols in $y_I$ which are corrupted. This gives rise to a natural question, mainly: ``Why not simply check the condition for all $m \in \mathbb{F}_q^{p}$, instead of $m \in (\mathbb{F}_q^{*})^{p}$?'' As this would allow for decoding $y$ as long as $y_I$ has $p$ or fewer errors. It is a matter of minimizing the expected work, if we chose $m \in \mathbb{F}_q^{p}$ the inner loop in Algorithm \ref{alg:lee-brickell} would have to perform $q^p$ iterations. On the otherhand picking $m \in (\mathbb{F}_q^{*})^{p}$ the loop only have to perform $(q - 1)^p$ iterations. Additionally as we noted in Remark \ref{rem:information_set_alg}, the overhead of the process of finding an information set, after we have computed the first information set, is very low.

For a given information set $I$ the probability for the algorithm successfully decoding $y$ is:
\begin{equation*}
  \mathbb{P}(\abs{I \cap \support(e)} = p) = \binom{n - k}{t - p}\binom{k}{p}\binom{n}{t}^{-1}
\end{equation*}
Since we must have $t - p$ errors in the symbols of $y_{\left\{1, \ldots, n\right\} \setminus I}$ and $\binom{n}{t}$ possible error vectors. Additionally the work performed iteration, given that it is unsuccessful is about:
\begin{equation*}
  W_I = \alpha k^3 \binom{k}{p} (q - 1)^{p}
\end{equation*}
Since we have to compute $G_I^{-1}$, and $G_I^{-1}G$ have to chose $J \subseteq I$ such that $\abs{J} = p$ and for each $J$ we compute $y' - \sum^p_{i = 1} m_i g_{*, j_i}$, with $m \in (\mathbb{F}_q^{*})^{p}$. Combining these facts we get that the expected work factor of Lee-Brickell's algorithm is about:
\begin{equation*}
  W = \alpha k^{3} \binom{k}{p} (q - 1)^p\dfrac{\binom{n}{t}}{\binom{n - k}{t - p}\binom{k}{p}} = \alpha k^3 (q - 1)^{p} \dfrac{\binom{n}{t}}{\binom{n - k}{t - p}}
\end{equation*}
again, since we are computing the expected value of a geometric distribution.

\subsubsection{Sterns Algorithm}
During this subsection we fix a $[n, k]_q$ code $\mathcal{C}$ furthermore for the sake of simplicity we assume that $k$ is even. Let $G$ be a generator matrix of $\mathcal{C}$. The main idea of Stern's algorithm is to split $G$ into two sub matrices each with $p$ rows, and check if the linear combinations of said rows, overlap at certain indices. If we find a match, then we the weight of the remaining (non-overlapping) part. If it has weight $t$ then we have found error vector.
\begin{algorithm}[H]
\caption{Stern's algorithm for information set decoding}\label{alg:stern}
\begin{algorithmic}
  \Procedure{Stern ISD}{$G$: a generator matrix of $\mathcal{C}$, $\ell$: an integer with $0 \leq \ell \leq n - k$,
    \newline\phantom{\textbf{procedure} \textsc{Stern ISD}(}$p$ an integer with $0 \leq p \leq t$, $y$: received word with a maximum \newline\phantom{\textbf{procedure} \textsc{Stern ISD}(}of $t$ errors}
  \Loop
  \State $I \in \Call{IS Generator}{G, true}$ \Comment{Chosen with replacement}
  \State $y' \gets y - y_I G_I^{-1}G$
  \State Choose $X \subseteq I$ uniformly such that $\abs{X} = k / 2$
  \State $Y \gets I \setminus X$
  \State Choose $Z \subseteq \left\{1, 2, \ldots, n\right\} \setminus I$ such that $\abs{Z} = \ell$
  \For{$A = \left\{a_1, a_2, \ldots, a_{p}\right\} \subseteq X$, $B = \left\{b_1, b_2, \ldots, b_{p}\right\} \subseteq Y$ with $ \abs{A} = \abs{B} = p$}
  \State $(\mathcal{V}_A, \mathcal{V}_{B}) \gets \left(\left\{y - \sum_{i = 1}^p m_i g_{*, a_i} | m \in (\mathbb{F}_q^{*})^p\right\}, \left\{y - \sum_{i = 1}^p m_i g_{*, b_i} | m \in (\mathbb{F}_q^{*})^p\right\}\right)$
  \For{$a \in \mathcal{V}_A, b \in \mathcal{V}_B$ with $a_i = b_i$ for all $i \in Z$}
    \State $e \gets a - b$
    \If {$\wt(e) = t$}
    \State \Return $y - e$
    \EndIf
    \EndFor
  \EndFor
  \EndLoop
  \EndProcedure
\end{algorithmic}
\end{algorithm}
\begin{remark}
  Contrary to the approach in Algorithms \ref{alg:plain_ISD} and \ref{alg:lee-brickell}. The information sets $I$, in Algorithm \ref{alg:stern}, are chosen with replacement, since $X, Y \subseteq I$ are choosen uniformly, hence we cannot simply iterate through each information set once. (Since we may not pick the ``correct'' sets $X$ and $Y$ for a given iteration)
\end{remark}

We will not discuss the expected work factor of Sterns algorithm, instead we refer to \cite{information_set_decoding}[Sections 4 and 5].

%\textcolor{red}{\textbf{TODO}} succes prob and expected work
%
%Let $I$ be a information set chosen uniformly at random and $e \in \mathbb{F}_q^n$ with $\wt(e) = t$. Then
%\begin{equation*}
%  \mathbb{P}(\wt(e_I) = 2p) = \frac{\binom{k}{2p}\binom{n - k}{t - 2p}}{\binom{n}{t}}
%\end{equation*}
%Since we will have $2p$ errors in the entries specified by $I$ and $t - 2p$ in the rest. If $X,Y, Z$ are chosen as described in Algorithm \ref{alg:stern}, then
%\begin{equation*}
%  \mathbb{P}(\wt(e_X) = \wt(e_Y) = p | \wt(e_I) = 2p) = \frac{\binom{k / 2}{p}^{2}}{\binom{k}{2p}}
%\end{equation*}
%Additionally:
%\begin{equation*}
%  \mathbb{P}(\wt(e_Z) = 0 | \wt(e_{I^c}) = t - 2p) = \frac{\binom{n - k - (t - 2p)}{\ell}}{\binom{n - k}{\ell}}
%\end{equation*}
%That is the conditionally probability of having $t - 2p$ errors outside the infromation set while avoiding the positions indexed by $Z$. Since \textcolor{blue}{bla bla bla}, we see that the succes probability in a single iteration is:
%\begin{align*}
%  \mathbb{P}(\textit{succes}) &= \mathbb{P}(\wt(e_I) = 2p) \mathbb{P}(\wt(e_X)=\wt(e_Y) = p | \wt(e_I) = 2p) \mathbb{P}(\wt(E_Z) = 0 | e_{I^c} = t - 2p) \\
%  &= \frac{\binom{k}{2p}\binom{n - k}{t - 2p}}{\binom{n}{t}} \frac{\binom{k / 2}{p}^{2}}{\binom{k}{2p}} \frac{\binom{n - k - (t - 2p)}{\ell}}{\binom{n - k}{\ell}} \\
%  &= \frac{\binom{n - k}{t - 2p}\binom{k / 2}{p}^{2}\binom{n - k - (t - 2p)}{\ell}}{\binom{n}{t}\binom{n - k}{\ell}}
%\end{align*}

