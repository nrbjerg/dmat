\chapter{Generic Decoding Algorithms}
In this chapter we study generic decoding algorithms (meaning they work for arbitrary codes), we study these algorithms as the provide generic attacks on the McElice public key cryptography scheme.
We will exclusively be working nearest neighbor decoding, meaning for a received message $y = c + e \in \mathbb{F}_q^{n}$, where $e \in \mathbb{F}_q^n$ is a random error vector and $c$ is a codeword in a $[n, k]_{q}$ code $\mathcal{C}$. We then estimate $c$ as $\hat{c} := \underset{c \in \mathcal{C}}{\arg \min} (d(c, y))$, equivalently we can estimate the error $e$ as $\hat{e} := \underset{e \in \mathbb{F}_q^n \\ y - e \in \mathcal{C}}{\arg \min} \wt(e)$, we will refer to both of these as nearest neighbor estimates.

%If we can find \textcolor{blue}{efficient} algorithm for solving the McElice problem, see Problem \ref{prob:McElice}, then we would have broken the McElice PKCS, hence we will focus on the cases where we know a generator matrix of the code.
\section{Syndrome Decoding} \label{sec:syndrome_decoding}
The following section is based on \cite{huffman}[Subsection 1.11.2]. We start by introducing a way to partition $\mathbb{F}_q^n$ using a linear code.
\begin{definition}
  Let $\mathcal{C}$ be a $[n, k]_q$ code and let $y \in \mathbb{F}_q^n$, then the \textit{coset} of $\mathcal{C}$ determined by $y$ is defined as:
  \begin{equation*}
    \mathcal{C} + y := \left\{c + y \middle | c \in \mathcal{C}\right\} =: y + \mathcal{C}
  \end{equation*}
\end{definition}
We note that these cosets are nothing more, than the cosets found in group theory, as $(\mathcal{C}, +)$ forms a subgroup of the abelian group $(\mathbb{F}_q^n, +)$. Hence the results of the following proposition should come as no suprice, never the less, we will provide a proof.
\begin{proposition}\label{prop:basic_properties_of_cosets}
  Let $\mathcal{C}$ be a $[n, k]_q$ code, then the following holds for all $y, y' \in \mathbb{F}_q^{n}$:
  \begin{enumerate}
    \item If $y' \in \mathcal{C} + y$ then $\mathcal{C} + y = \mathcal{C} + y'$.\label{prop:basic_properties_of_coset1}
    \item If $\mathcal{C} + y \neq \mathcal{C} + y'$ then $(\mathcal{C} + y) \cap (\mathcal{C} + y') = \emptyset$. \label{prop:basic_properties_of_coset2}
    \item There are $q^{n - k}$ disjoint cosets of $\mathcal{C}$. \label{prop:basic_properties_of_coset3}
  \end{enumerate}
\end{proposition}
\begin{proof}
  We start with Assertion \ref{prop:basic_properties_of_coset1}:
  If $y' \in \mathcal{C} + y$, then $y' = c + y$ for some $c \in \mathcal{C}$, hence $y' + (-c) = y$. Now since $\mathcal{C}$ is a vector space we see that
  \begin{equation*}
   \underset{=y}{\underbrace{y' + (-c)}} + c' \in \mathcal{C} + y \text{ for all } c' \in \mathcal{C}
  \end{equation*}
  so $y' + \mathcal{C} \subseteq y + \mathcal{C}$. We may apply an similar argument to get the other inclusion, thus $\mathcal{C} + y = \mathcal{C} + y'$.

  Continuing with Assertion \ref{prop:basic_properties_of_coset2}: Assume for the sake of contradiction that $\mathcal{C} + y \neq \mathcal{C} + y'$ and $(\mathcal{C} + y) \cap (\mathcal{C} + y') \neq \emptyset$, then pick $x \in (\mathcal{C} + y) \cap (\mathcal{C} + y')$. By Assertion \ref{prop:basic_properties_of_coset1} we have $\mathcal{C} + y = \mathcal{C}  + x = \mathcal{C} + y'$ which is clearly a contradiction.

  Finally Assertion \ref{prop:basic_properties_of_coset3} follows by the Lagrange's index theorem as:
  \begin{equation*}
    q^{n} = \abs{\mathbb{F}_q^n} = \abs{\mathbb{F}_q^{n} / \mathcal{C}}\abs{\mathcal{C}} = \abs{\mathbb{F}_q^n / \mathcal{C}} q^{k}
  \end{equation*}
  implies that $\abs{\mathbb{F}_q^n / \mathcal{C}} = q^{n - k}$. The fact that the cosets are disjoint follows directly from Assertion \ref{prop:basic_properties_of_coset2}.
\end{proof}

\begin{definition}
  Let $\mathcal{C}$ be a $[n, k]_{q}$ code with parity check matrix $H \in \mathbb{F}_q^{(n - k) \times n}$, the syndrome map:
  \begin{equation*}
    S_H: \mathbb{F}_q^n \ni y \mapsto H y \in \mathbb{F}_q^{(n - k)}
  \end{equation*}
  Additionally for any $y \in \mathbb{F}_q^n$ we say that $S_H(y)$ is the \textit{syndrome} associated with $y$.
\end{definition}

\begin{remark}\label{rem:basic_properties}
  Clearly $S_H(y) = 0$ if and only if $y \in \mathcal{C}$, after all this is exactly the definition of a parity check matrix. In addition we have $S_H(y + y') = S_H(y) + S_H(y')$ since matrix vector multiplication is distributive.
\end{remark}

\begin{lemma}\label{lem:syndrome_is_the_same_iff_they_are_in_the_same_coset}
  Let $\mathcal{C}$ be a $[n, k]_{q}$ code with parity check matrix $H$ and $y, y' \in \mathbb{F}_q^{n}$, then $S_H(y)=S_H(y')$ if and only if $y$ and $y'$ are in the same coset of $\mathcal{C}$.
\end{lemma}

\begin{proof}
  Firstly we note that for all $y, y' \in \mathbb{F}_q^n$ there exists a $x \in \mathbb{F}_q^n$ such that $y = y' + x$ and hence $S_H(y) = S_H(y') + S_H(x)$ per Remark \ref{rem:basic_properties}. Now if $S_H(y) = S_H(y')$ we must have $S_H(x) = 0$ meaning $x \in \mathcal{C}$, again by Remark \ref{rem:basic_properties}. The other implication follows by a similar argument as $y$ and $y'$ being in the same coset, implies that there exists a codeword $c \in \mathcal{C}$ such that $y = y' + c$ and hence $S_H(y) = S_H(y') + S_H(c) = S_H(y')$ by Remark \ref{rem:basic_properties}.
\end{proof}

\begin{definition}\label{def:syndrome_lookup_table}
  If $\mathcal{C}$ is a $[n, k]_q$ code, with parity check matrix $H \in \mathbb{F}_q^{(n - k) \times k}$, then we define the \textit{syndrome lookup table} (SLT) $S^{*}_H: \mathbb{F}_q^n \to \mathbb{F}_q^n$ as
  \begin{equation*}
    S_{H}^{*}(y) = \underset{e \in y + \mathcal{C}}{\arg \min} \; \wt(e)
  \end{equation*}
  Furthermore the vector $S_H^{*}(y)$ is called the \textit{coset leader} of $y + \mathcal{C}$.
\end{definition}

Next we introduce the syndrome decoding algorithm, which requires will require such a syndrome lookup table, alternatively we could when decoding a received word $y$ go through the elements of $y + \mathcal{C}$ to find the coset leader, however this is impractical and we instead use a syndrome lookup table to avoid unecessary computation. Hence we will first introduce a procedure for constructing such a lookup table.

\begin{algorithm}[H]
\caption{Syndrome Lookup Table Construction and Syndrome Decoding}\label{alg:syndrome_decoding}
\begin{algorithmic}
  \Procedure{SLT Construction} {$\mathcal{C}$: a $[n,k]_q$ code, $H$: parity check matrix of $\mathcal{C}$}
  \State $S_H^{*} \gets \emptyset$ \Comment{We will view this as a mapping.}
  \For{$i \in \left\{1, 2, \ldots, n \right\}$}
  \For{$x \in \mathbb{F}_q^{n}$ with $\wt(x) = i$}
    \If {$S_{H}(x)$ is not already in the codomain of $S_H^{*}$}
      \State $S^{*}_H \gets S^{*}_H \cup \left\{(S_H(x), x)\right\}$
      \If {$\abs{S_{H}^{*}} = q^{n - k}$}
        \State \Return $S_H^{*}$
      \EndIf
    \EndIf
  \EndFor
  \EndFor
  \EndProcedure
  \newline

  \Procedure{Syndrome Decoding} {$y$: received word, $H$: a parity check matrix
  \newline\phantom{\textbf{procedure} \textsc{Syndrome Decoding}(}$S_H^{*}$: a syndrome lookup table}
    \State $\hat{e} \gets S_H^{*}(S_H(y))$
    \State \Return $y - \hat{e}$
  \EndProcedure
\end{algorithmic}
\end{algorithm}
\begin{remark}
The syndrome lookup table $S^*_H$ constructed in Algorithm \ref{alg:syndrome_decoding}, differs from the syndrome lookup table as defined in Definition \ref{def:syndrome_lookup_table}, in that it takes the syndrome of $y$ and returns the coset leader. However the function $S_H^{*} \circ S_H$ do confine to Definition \ref{def:syndrome_lookup_table}.
\end{remark}
\begin{remark}
  Since $\hat{e}$ is the coset leader of $y + \mathcal{C}$, we know that $y - \hat{e}$ is our nearest neighbor estimate for $c$, since $\wt(\hat{e})$ is minimal.
\end{remark}
We also note that since every coset contains a finite number of words, we may simply iterate through them to find the one which minimizes the hamming weight. By now the diligent reader, might start to question how does all this relate to solving the McEliece problem, since we simply know a generator matrix? This might seem like a problem, however it is pretty trivial to solve, we start by proving the following theorem:

\begin{theorem}\label{thm:from_generator_matrix_to_parity_check_matrix}
  Let $\mathcal{C}$ be a $[n, k]_q$ code and $A \in \mathbb{F}_q^{(n - k) \times k}$. Then $G = \begin{bmatrix} I_k & A \end{bmatrix}$ is a generator matrix for $\mathcal{C}$ if and only if $H = \begin{bmatrix}-A^T & I_{n - k}\end{bmatrix}$ is a parity check matrix for $\mathcal{C}$.
%Let $\mathcal{C}$ be a $[n, k]_q$ code. If $\mathcal{C}$ has a generator matrix $G$ which is in standard form, that is $G = \begin{bmatrix}
%I_k & A
%                                                                                                                   \end{bmatrix}$ for some $A \in \mathbb{F}_q^{(n  - k) \times k}$, then $H = \begin{bmatrix}
%-A^T & I_{n - k}
%                                                                                                                                                                                           \end{bmatrix}$ is a parity check matrix for $\mathcal{C}$.
\end{theorem}
\begin{proof}
  We start by noting that $HG^T = -A^T + A^T = 0_{k \times k}$ hence all rows of $G$ are in $\Null(H)$. Hence if $G$ is a generator matrix of $\mathcal{C}$, then $H$ is a parity check matrix of $\mathcal{C}$. On the otherhand if $H$ is a parity check matrix for $\mathcal{C}$, then $G$ is a generator matrix for $\mathcal{C}$ since $\dim_{\mathbb{F}_q}(\Null(H)) = n - \rank(H) = n - (n - k) = k$ and $\rank(G) = k$. Hence the rows of $G$ must form a $\mathbb{F}_q$-basis of $\Null(H) = \mathcal{C}$.
\end{proof}

Hence if we receive a generator matrix for a $[n, k]_q$ code $C$, we can simply apply Gaussian elimination algorithm to get a generator matrix $G$ which is in reduced echelon form, note that $G$ isn't neccecarily standard form, but by multiplying our generator matrix $G$ by a permutation matrix $P$, we may obtain a generator matrix $G'$, which is in standard form, since $G$ has $k$ pivots as $\dim(\mathcal{C}) = k$. We note that $G'$ is not nessecarily a generator matrix for $\mathcal{C}$ but rather a generator matrix for a code $\mathcal{C}'$ permutation equvilent to $\mathcal{C}$. Applying Theorem \ref{thm:from_generator_matrix_to_parity_check_matrix} we obtain a parity check matrix $H'$ for $\mathcal{C}'$, which we transform into a parity check matrix $H$ of $\mathcal{C}$ by setting $H = P^{-1}H'$

Finally we consider the time and space complexity of Algorithm \ref{alg:syndrome_decoding}. We will consider the two procedures described seperately, since we only need to create one syndrome lookup table once, to decode
a given code.

Since $\textsc{SLT Construction}$ loops over each coset of $\mathcal{C}$, of which there is $q^{n - k}$, by \ref{prop:basic_properties_of_cosets}\ref{prop:basic_properties_of_coset3}. To find the coset leader we can assume that we simply iterate over the words in $y + \mathcal{C}$, of which there is $q^k$, hence we see that $\textsc{SLT Construction}$ has a time complexity of $O(q^{n})$. In addition since each syndrome and coset leader pair needs to be stored, we see that $\textsc{SLT Construction}$ has a space complexity of $O(q^{n - k})$.

Continuing with the $\textsc{Syndrome Decoding}$ procedure, we once again get a space complexity of $O(q^{n - k})$, since we have to store $S_H^{*}$, while the time complexity depends on the underlying datastructure chosen to represent $S_H^{*}$, however it will be atleast $O((n - k)n)$ since we have to compute $S_H(y) = Hy$ and $H$ is a $(n - k) \times n$ matrix

\chapter{Division of multivariate polynomials}
The following notes are based on Chapter 5 and appendix A, of ``Concrete Abstract Algebra From Numbers to Gröbener bases''.
\section{Relations}
\begin{definition}
Let $S$ be a set and let $R \subseteq S \times S$ then $R$ is called a \textit{relation} on $S$ and we write $x R y$ to mean $(x,y) \in R$.
\end{definition}

\begin{definition}
These relations can have certain properties, for instance $R$ is called \textit{reflective} if $x R x$, \textit{symmetric} if $x R y \implies y R x$, \textit{antisymmetric} if $x R y \wedge y R x \implies x = y$ and \textit{transitive} if $x R y \wedge y R z \implies x R z$ for every $x, y, z \in S$
\end{definition}

\begin{definition}
If $R$ is reflective, symmetric and transitive, then $R$ is called an \textit{equivilence relation}.
On the other hand if $R$ is reflective, antisymmetric and transitive, then $R$ is called a \textit{partial ordering}
\end{definition}

\subsection{Partial Orderings}
\begin{definition}
A partial ordering $R$ on $S$ is called \textit{total ordering} if $x \leq y$ or $y \leq x$ for every $x,y \in S$.
If every non-empty $M \subseteq S$ as a \textit{minimum element} $m \in M$, meaning $m \leq x$ for all $x \in M$, then $\leq$ is called a \textit{well ordering} on $S$
\end{definition}

\section{Orderings}
\begin{definition}\label{def:term_ordering}
A total ordering $\leq$ on $\mathbb{N}^{n}$ is called a \textit{term ordering} if:
1. $0 \leq v$.
2. $v_1 \leq v_2 \implies v_1 + v \leq v_2 + v$.
For all $v, v_1,v_2 \in \mathbb{N}^{n}$.
\end{definition}
\begin{example}\label{exmp:lexigraphic_orderings}
 The \textit{lexicographic ordering} $\leq_{lex}$ on $\mathbb{N}^{n}$ is defined by $v \leq_{lex} w$ if there exists $j \in \mathbb{N}$ such that $v_i = w_i$ for all $i \leq j$ and $v_j < w_j$ or $j = n$. \footnote{The lexicographic ordering can be thought of the ``alphabetic'' ordering of the tuples of natural numbers.} \\

The \textit{graded lexicographic ordering} $\leq_{glex}$ is defined by $v \leq_{glex} w$ if $\sum_{i = 1}^n v_i \leq \sum_{i = 1}^n w_{i}$ in the case of equality we also require that $v \leq_{lex} w$.
\end{example}
\subsection{Dicksons Lemma}
\begin{lemma}\label{lem:dicksons}[Dickons]
Let $S \subseteq \mathbb{N}^{n}$. Then there exists a finite set of vectors $v_1, v_2, \ldots v_m \in S$ such that
\begin{equation*}
    S \subseteq \bigcup_{i = 1}^m v_i + \mathbb{N}^{n}
\end{equation*}
\end{lemma}

\begin{proof}
We use strong induction, if $n = 1$, then pick $v_1 = \inf S$, then clearly $S \subseteq v_1 + \mathbb{N}$.

Let $\pi: \mathbb{N}^n \to \mathbb{N}^{n - 1}$ denote the map $(x_1, x_2 \ldots, x_{n}) \mapsto (x_2 \ldots, x_{n})$. Using our hypothesis we see that there exists $v_1, v_2, \ldots v_m \in \pi(S) \subseteq S$ such that $\pi(S) \subseteq \bigcup_{i = 1}^m v_i + \mathbb{N}^{n - 1}$ (since $\pi(S) \subseteq \mathbb{N}^{n - 1}$)

However it is not always the case that $S \subseteq \bigcup_{i = 1}^m v_i + \mathbb{N}^{n}$, after all $v_1, v_2, \ldots v_m$ wheren't constructed with the first coordinates in mind. Hence let
\begin{equation*}
 M = \max\{(v_{1})_{1}, (v_2)_1, \ldots, (v_m)_1\}
\end{equation*}
and $S_i = \left\{s \in S \middle| s_1 = i \right\}$ as well as $S \leq M = \left\{s \in S \middle| s_1 \leq M\right\}$. Then $S = \bigcup_{k = 0}^{M - 1} S_{k} \cup S_{\leq M}$, now since $S_{\geq M} \subseteq \bigcup_{i = 1}^m v_i + \mathbb{N}^{n}$, and $S_j$ can be identified with $\mathbb{N}^{n - 1}$ (The first coordinate of each element is fixed.) the result follows from our hypothesis
\end{proof}

\begin{corollary}\label{cor:every_term_ordering_is_a_well_ordering}
 Every term ordering $\leq$ on $\mathbb{N}^n$ is a well ordering
\end{corollary}
\begin{proof}
  Let $S \subseteq \mathbb{N}^{n}$ be a non-empty subset, then by Dicksons Lemma there are finitely many elements $v_1, v_2, \ldots v_m \in S$ such that $S \subseteq \bigcup_{i = 1}^{m} v_i + \mathbb{N}^{n}$.
  Now if $v \in v_i + \mathbb{N}^{n}$, then $v = v_i + w$ for some $w \in \mathbb{N}^{n}$ which implies $v - v_i \in \mathbb{N}^{n}$ hence $v = (v - v_i) + v_{i} \geq v_{i}$ by Definition \ref{def:term_ordering}, this means that the smallest element in $S$ is the smallest element in $v_1, v_2 \ldots, v_{m}$.
\end{proof}

\begin{definition}
  Let $f = \sum^{m}_{i = 1} a_i X^{v_{i}} \in \mathbb{F}[X_1, X_2 \ldots, X_{n}]$ then the \textit{leading term} of $f$ with respect to the term ordering $\leq$ is denoted as $LT_{\leq}(f) = a_{j}X^{v_{j}}$ where $v_j \leq v_{i}$ for all $i$. We also often write $aX^v_{1} \leq aX^{v_{2}}$ if $v_1 \leq v_{2}$.
\end{definition}
If $R$ is a domain then $LT_{\leq}(fg) = LT_{\leq}(f)LT_{\leq}(g)$ for all $f, g \in R[X_1, X_2 \ldots, X_{n}]$.

\section{The Division Algorithm}%
\label{sec:division algorithm for multivariate polynomials.}

\begin{proposition}\label{prop:division_algorithm}
  Let $R$ be a domain, $\leq$ a term ordering and $f \in R[X_1, X_2 \ldots, X_{n}] \setminus \left\{0\right\}$. Suppose that $f_1, f_2 \ldots, f_{m} \in R[X_1, X_2 \ldots, X_{n}] \setminus \left\{0\right\}$, then there exists $a_1, a_2 \ldots, a_{m} , r \in R[X_1, X_2 \ldots, X_{n}]$ such that
  \begin{equation*}
    f = \sum_{i = 1}^m a_i f_i + r
  \end{equation*}
  where $r = 0$ or none of the terms in $r$ is divisible by $LT_{\leq}(f_{i})$. Furthermore $LT_{\leq}(a_if_i) \leq LT_{\leq}(f)$ if $a_i \neq 0$.
\end{proposition}
Here is the algorithm for computing $a_1, a_2 \ldots, a_{m}, r$:
\begin{enumerate}
  \item Let $a_1 := a_2 := a_m := r := 0$ and $s := f$ giving: $f \stackrel{(*)}{=} \sum_{i = 1}^m a_i f_i + (r + s)$ (The main idea is that this expression, should stay constant during the algorithm.)
  \item We now iterate. If $s = 0$ we are done with the algorithm otherwise we perform the following steps
  \begin{enumerate}
    \item If $LT_{\leq}(f_{i}) | LT_{\leq} (s)$ for some $i$, then pick the smallest of these $i$'s and let:
\begin{align*}
  s := s - \frac{LT_{\leq}(s)}{LT_{\leq}(f_{i})} f_{i}, \quad
  a_{i} := a_{i} + \frac{LT_{\leq}(f_{i})}{LT_{\leq}(f_{i})}
\end{align*}
    Notice that $(*)$ still holds.
    \item If $LT_{\leq}(s)$ is not divisible by any $LT_{\leq}(f_{i})$, we set $r := r + LT_{\leq}(s)$ and $s := s - LT_{\leq}(s)$, again notice that $(*)$ still holds.
  \end{enumerate}
\end{enumerate}
We will leave out the proof of the correctness of this algorithm.
\section{Gröbner bases}%
\label{sec:gröbner}
The main idea is that we want to have a set, where the remainder of the division algorithm does not depend on the term ordering.

\begin{definition}
  Let $f_1, f_2 \ldots, f_{m} \in \K[X_1, X_2 \ldots, X_{n}] \setminus \left\{0\right\}$, then the set $F := \left\{f_1, f_2 \ldots, f_{m}\right\}$ is called a \textit{Gröbner basis for an ideal} $I \subseteq \K[X_1, X_2 \ldots, X_{n}]$ with respect to the term ordering $\leq$ if $F \subseteq I$ and for every $f \in I \setminus \left\{0\right\}$, we have $LT_{\leq}(f_{i}) | LT_{\leq}(f)$ for some $i = 1, \ldots, m$. Finally $F$ is called a \textit{Gröbner basis} with respect to the term ordering $\leq$ if it is a Gröbner basis of $\gen{f_1, f_2 \ldots, f_{m}}$.
\end{definition}

\begin{proposition}\label{prop:}
  Let $\left\{f_1, f_2 \ldots, f_{m}\right\}$ be a Gröbner basis with respect to the term ordering $\leq$. Then for $I = \gen{f_1, f_2 \ldots, f_{m}}$ we have $f \in I$ if and only if $f$ divided by $f_1, f_2 \ldots, f_{m}$ has remainder $0$.
\end{proposition}


