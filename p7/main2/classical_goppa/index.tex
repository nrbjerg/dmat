\chapter{Classical Goppa Codes}
Next we introduce classical Goppa codes. The original proposal of McElice used these codes in its construction. In addition they remain as one of the few codes proposed for with no publicly known structural attacks, atleast if they are chosen correctly.

\begin{theorem}\label{thm:inverse}
  Let $f \in \mathbb{F}_q[X]$ and $\alpha \in \mathbb{F}_q$ such that $f(\alpha) \neq 0$, then the inverse of $(X - \alpha_i)$ exists in the quotient ring $\frac{\mathbb{F}_{q}[X]}{\gen{f}}$, and may be computed as $-\left(\frac{f(X) - f(\alpha_{i})}{X - \alpha_{i}}\right)f(\alpha_{i})^{-1}$.
\end{theorem}

\begin{proof}
  Follows from the fact that:
  \begin{align*}
-\left(\frac{f(X) - f(\alpha_{i})}{X - \alpha_{i}}\right)f(\alpha_{i})^{-1}(X - \alpha_i) &= - f(\alpha_{i})^{-1} (f(X) - f(\alpha_{i})) \\ &= -f(\alpha_{i})^{-1} f(X) + 1 \equiv 1 \mod f
  \end{align*}
  Which is equivalent with $-\left(\frac{f(X) - f(\alpha_{i})}{X - \alpha_{i}}\right)f(\alpha_{i})^{-1}(X - \alpha_i) = 1 \in \frac{\mathbb{F}_{q}[X]}{\gen{f}}$.
\end{proof}

\begin{definition}\label{def:classical}
  Let $x \in \mathbb{F}_q^n$ and $f \in \mathbb{F}_q[X]$ such that $f(x_i) \neq 0$ for $i = 1, 2, \ldots, n$ and $\mathbb{F}_{q_0} \subseteq \mathbb{F}_q$ be a subfield, then the \textit{classical Goppa code} associated with $(x, f, \mathbb{F}_{q_0})$ is defined as:
  \begin{align*}
    \Gamma(x, f, \mathbb{F}_{q_0}) :&= \left\{c \in \mathbb{F}^n_{q_0} \middle| \sum^n_{i = 1} \frac{c_{i}}{X - x_{i}} \equiv 0 \mod f\right\}
     \\&= \left\{c \in \mathbb{F}^n_{q_0} \middle| \sum^n_{i = 1} \frac{c_{i}}{X - x_{i}} = 0 \in \frac{\mathbb{F}_q[X]}{\gen{f}}\right\}
  \end{align*}
  If $f$ is irreducible, then $\Gamma(x, f, \mathbb{F}_{q_0})$ is also called irreducible.
\end{definition}
From Theorem \ref{thm:inverse} and Definition \ref{def:classical} it follows that $c = (c_1, c_2, \ldots, c_{n}) \in \Gamma(x, f, \mathbb{F}_{q_0})$ if and only if
\begin{equation}\label{eq:condtion_on_coef}
  -\sum_{i = 1}^n c_i \frac{1}{f(x_i)} \frac{f(X) - f(x_i)}{X - x_i} = 0 \in \frac{\mathbb{F}_{q}[X]}{\gen{f}}
\end{equation}
additionally if $f = \sum_{j = 0}^{\deg(f)} a_j X^j$, then:
\begin{align*}
  -\frac{1}{f(x_i)}\frac{f(X) - f(x_i)}{X - x_{i}} &= -\frac{1}{f(x_{i})} \frac{\sum_{j=0}^{\deg(f)} a_{j}(X^j - x_i^j)}{X - x_i} \\ &\stackrel{(a)}= -\frac{1}{f(x_i)} \sum^{\deg(f)}_{j = 1} a_j \sum^{j - 1}_{k = 0} X^k x_i^{j - 1 - k}\\
  &\stackrel{(b)}= -\frac{1}{f(x_{i})} \sum^{\deg(f) - 1}_{k = 0} X^k \left(\sum^{\deg(f)}_{j = k + 1} a_j x_i^{j - 1 - k}\right)
\end{align*}
where $(a)$ follows by polynomial division and $(b)$ from interchanging the sums. Thus:
\begin{equation}\label{eq:yields_pcm}
  \sum_{i = 1}^n \frac{c_i}{X - x_i} = -\sum^n_{i = 1} \frac{c_{i}}{f(x_{i})} \sum_{k = 0}^{\deg(f) - 1} X^k \sum_{j = k + 1}^{\deg(f)} a_j x_i^{j - 1 - k}
\end{equation}
Equation \eqref{eq:yields_pcm} must equal $0$ since $c \in \Gamma(x, f, \mathbb{F}_{q_0})$ which is the case if and only if the coefficients of each $X^k$ is zero, which yields the following proposition:
\begin{proposition}\label{prop:pcm_classical_goppa}
  Let $x \in \mathbb{F}_q^n$ and $f \in \mathbb{F}_q[X]$ with $l := \deg(f)$, then the classical gopppa code $\Gamma(x, f, \mathbb{F}_{q_0})$ has parity check matrix:
  \begin{equation*}
    H = \begin{bmatrix}
          f(x_{1})^{-1} a_{l} & f(x_{2})^{-1} a_{l} & \cdots & f(x_{n})^{-1} a_{l} \\
          f(x_{1})^{-1} (a_{l} + a_{l - 1}x_1) & f(x_{2})^{-1} (a_{l} + a_{l - 1}x_2) & \cdots & f(x_{n})^{-1} (a_{l} + a_{l - 1}x_n) \\
          \vdots & \vdots & \ddots & \vdots \\
          f(x_{1})^{-1} \sum_{j = 1}^l a_j x_1^{j -  1} & f(x_{2})^{-1} \sum_{j = 1}^l a_j x_2^{j -  1} & \cdots & f(x_{n})^{-1} \sum_{j = 1}^l a_j x_n^{j -  1} &
        \end{bmatrix}
  \end{equation*}
  over $\mathbb{F}_q$.
\end{proposition}
The following Corollary uses some elementary results from \textcolor{blue}{Galois} theory, namely that if $\mathbb{F}_{q_0}$ is a subfield of the finite field $\mathbb{F}_q$, then there exists a $\mathbb{F}_{q_0}$-basis of $\mathbb{F}_q$. We will denote the length of such a basis by $[\mathbb{F}_q : \mathbb{F}_{q_0}]$ an call it the \textit{degree} of the field extension $\mathbb{F}_q / \mathbb{F}_{q_0}$. \textcolor{blue}{Could be proven in an appendix.}
\begin{corollary}
  Let $\Gamma(x, f, \mathbb{F}_{q_0})$ be a classical goppa code with $x \in \mathbb{F}_q$ and $f \in \mathbb{F}_q[X]$ then:
  \begin{equation*}
    \dim_{\mathbb{F}_{q_0}}(\Gamma(x, f, \mathbb{F}_{q_0})) \geq n - m \deg(f)
  \end{equation*}
  where $m = [\mathbb{F}_q : \mathbb{F}_{q_0}]$.
\end{corollary}
\begin{proof}
  Fixing a $\mathbb{F}_{q_0}$-basis of $\mathbb{F}_q$ say $b_1, b_2, \ldots, b_{m} \in \mathbb{F}_q$, each entry in $H$ can be viewed as a vector in $\mathbb{F}_{q_0}^m$, by the vector space isomorphism:
  \begin{equation*}
  \mathbb{F}_q \ni x =  x_1 b_1 + x_2 b_2 + \cdots + x_m b_m \mapsto \begin{bmatrix} x_1 & x_2 & \cdots & x_{m} \end{bmatrix}^{T} \in \mathbb{F}_{q_0}^m
  \end{equation*}
  Thus $H$ from Proposition \ref{prop:pcm_classical_goppa} can be viewed as a $\mathbb{F}_{q_0}$-matrix $H'$ with dimensions $m \deg(f) \times n$. Since $H'$ has $mt$ rows, we see that $\rank(H') \leq m \deg(f)$, hence
  \begin{equation*}
    \dim_{\mathbb{F}_{q_0}}(\Gamma(x, f, \mathbb{F}_{q_0})) = \dim_{\mathbb{F}_{q_0}}(\Null(H')) = n - \rank(H') \geq n - m \deg(f) \qedhere
  \end{equation*}
\end{proof}

Finally we reach the main theorem of this chapter.

\begin{theorem}\label{thm:classical_goppa_is_AG}
  Let $\Gamma(x, f, \mathbb{F}_q)$ be a classical Goppa code, $P_i = (x_i : 1)$ for $i = 1, 2, \ldots, n$ and $D = \sum_{i = 1}^n P_i$. Then:
  \begin{equation}\label{eq:goppa_is_eval}
    \Gamma(x, f, \mathbb{F}_q) = \mathcal{C}_{\Omega}(\mathbb{P}^1, D, (f)_{0} - P_{\infty}) = \mathcal{C}_L(\mathbb{P}^1, D, (\omega) + D - (f_0) + P_{\infty})
  \end{equation}
  with $\omega = \frac{dh}{h}$ where $h(x) := \prod^n_{i = 1} (x - x_{i})$.
\end{theorem}
\begin{proof}
  The last equality in Equation \eqref{eq:goppa_is_eval} follows by Theorem \ref{thm:res_is_eval}. Hence it is sufficient to show the first equality. We start by showing that $\Gamma(x, f, \mathbb{F}_q) \subseteq \mathcal{C}_{\Omega}(\mathbb{P}^1, D, (f)_0 - P_{\infty})$. Hence let $c \in \Gamma(x, f, \mathbb{F}_q)$ and
  \begin{equation*}
    \omega_c := \left(\sum_{i=1}^n \frac{c_{i}}{x - x_{i}} \right) dx
  \end{equation*}
  by the definition of $\Gamma(x, f, \mathbb{F}_q)$ we see that $v_{P}(\omega_c) \geq 0$ for all $P \in \support((f)_{0})$. In addition $v_{P_i}(\omega_c) = -1$ and $\omega_c$ is regular at any other point $\mathbb{P}^1 \setminus \left\{P_\infty\right\}$. Finally we compute $v_{P_\infty}(\omega_c)$ by setting $x = 1/u$, we see that:
  \begin{equation*}
    \omega_c = \sum_{i=1}^n  \frac{-c_{i} \frac{du}{u^{2}}}{\frac{1}{u} - x_i} = - \sum^n_{i = 1} \frac{c_{i} du}{u(1 - x_{i}u)}
  \end{equation*}
  thus $v_{P_\infty}(\omega_{c}) \geq -1$ and hence $\omega_c \in \Omega((f)_0 - P_{\infty} - D)$ meaning $\Gamma(x, f, \mathbb{F}_q) \subseteq \mathcal{C}_{\Omega}(\mathbb{P}^1, D, (f)_0 - P_{\infty})$.
  \textcolor{red}{\textbf{TODO}} Conversely
\end{proof}
Since Theorem \ref{thm:classical_goppa_is_AG} shows that classical Goppa codes are in fact AG codes it makes sense to speak about their designed minimum distance. This leads us to the following corollary:
\begin{corollary}\label{cor:f_splitting_yields_design_distance}
  If $f \in \mathbb{F}_q[X]$ splits into linear factors over $\mathbb{F}_q$, then:
\begin{equation*}
  d^{*}\left(\Gamma(f, x, \mathbb{F}_q)\right) = \deg(f) + 1
\end{equation*}
\end{corollary}
\begin{proof}
  Using Theorem \ref{thm:classical_goppa_is_AG} we see that:
  \begin{align*}
    d^{*}(\Gamma(f, x, \mathbb{F}_q)) &=d^{*}(\mathcal{C}_L(\mathbb{P}^1, D, (\omega) + D - (f)_0 + P_{\infty}))\\
                                      &= n - \deg((\omega) + D - (f)_0 + P_{\infty}) \\
                                      &\stackrel{(a)}= n - \left(\deg((\omega)) + \deg(D) - \deg((f)_{0}) + \deg(P_{\infty})\right) \\
    &\stackrel{(b)}= n - \left(- 2 + n - \deg(f) + 1\right) \\
    &= \deg(f) + 1
  \end{align*}
  Where $(a)$ follows since $\deg\colon Div(\mathbb{P}^1) \to \mathbb{N}$ is a homomorphism, and Equality $(b)$, follows by $\deg((f)_0) = \deg(f)$ since $f$ factors into linear polynomials and $\deg((\omega)) = -2$ by Proposition \ref{prop:degree_of_canonical_divisor}, with $g = 0$ since $\mathcal{X} = \mathbb{P}^{1}$.
\end{proof}


\section{Subfield Subcodes}%
The following section is concerned with what happends if we only consider the codewords, of a linear code over $\mathbb{F}_q$, with entries in a subfield of $\mathbb{F}_q$. We start by motivating the concept by the following proposition:
\begin{proposition}\label{prop:classical_goppa_is_subfield_subcodes}
  Let $x \in \mathbb{F}_q^n$ and $f \in \mathbb{F}_q[X]$, such that $f(x_i) \neq 0$ for $i = 1, 2, \ldots, n$.
  Suppose $\mathbb{F}_{q_0}$ is a subfield of $\mathbb{F}_q$ then:
  \begin{equation*}
    \Gamma(x, f, \mathbb{F}_{q_0}) = \Gamma(x, f, \mathbb{F}_q) \cap \mathbb{F}_{q_0}^n
  \end{equation*}
\end{proposition}
\begin{proof}
  Follows straight from the definition, namely:
  \begin{align*}
  \Gamma(x, f, \mathbb{F}_{q_0}) &= \left\{c \in \mathbb{F}_{q_0}^n \middle| \sum_{i = 1}^n \frac{c_{i}}{X - x_{i}} \equiv 0 \mod f\right\}\\ &=\left\{c \in \mathbb{F}_{q}^n \middle| \sum_{i = 1}^n \frac{c_{i}}{X - x_{i}} \equiv 0 \mod f\right\} \cap \mathbb{F}_{q_0}^{n} = \Gamma(x, f, \mathbb{F}_q) \cap \mathbb{F}_{q_0}^n\qedhere
  \end{align*}
\end{proof}

We generalize the concept described in Proposition \ref{prop:classical_goppa_is_subfield_subcodes}, with the following definition.
\begin{definition}
  Let $\mathcal{C}$ be a $\mathbb{F}_q$ linear code and $\mathbb{F}_{q_0}$ be a subfield of $\mathbb{F}_q$, then the $\mathbb{F}_{q_0}$ linear code:
  \begin{equation*}
  \mathcal{C}\vert_{\mathbb{F}_{q_0}} := \mathcal{C} \cap \mathbb{F}_{q_0}^{n}
  \end{equation*}
  is called a \textit{subfield subcode} of $\mathcal{C}$.
\end{definition}
Let $c_{1}, c_{2} \in \mathcal{C} \vert_{\mathbb{F}_{q_0}}$, then $c_{1} + c_{2} \in \mathcal{C} \vert_{\mathbb{F}_{q_0}}$, since they are both in $\mathcal{C}$ and $\mathbb{F}_{q_0}^n$. Additionally $c \in \mathcal{C} \vert_{\mathbb{F}_{q_{0}}}$ implies that $kc \in \mathcal{C} \vert_{\mathbb{F}_{q_0}}$ for all  $k \in \mathbb{F}_{q_0}$, since $kc \in \mathcal{C}$ and $kc \in \mathbb{F}_{q_0}^{n}$. Hence $\mathcal{C} \vert_{\mathbb{F}_{q_0}}$ is a $\mathbb{F}_{q_{0}}$ linear code.

\begin{proposition}\label{prop:mininum_distance_of_subfield_subcode}
  If $\mathcal{C}$ is a $\mathbb{F}_q$ linear code and $\mathbb{F}_{q_0}$ is a subfield of $\mathbb{F}_q$, then $d(\mathcal{C}) \leq d(\mathcal{C} \vert_{\mathbb{F}_{q_0}})$.
\end{proposition}
\begin{proof}
  We have $\displaystyle d(\mathcal{C}) = \min_{c \in \mathcal{C}} \wt(c) \leq \min_{c \in \mathcal{C} \vert_{\mathbb{F}_{q_0}}} \wt(c) = d(\mathcal{C} \vert_{\mathbb{F}_{q_0}})$ since $\mathcal{C} \vert_{\mathbb{F}_{q_0}} \subseteq \mathcal{C}$.
\end{proof}
If $\mathcal{C}$ is a $[n, k]_q$ code we generally do not have $d(\mathcal{C}) = d(\mathcal{C} \vert_{\mathbb{F}_{q_0}})$, where $\mathbb{F}_{q_0}$ is a subfield of $\mathbb{F}_q$. This is illustrated with the following example.

\begin{example}\label{exmp:reduction_to_repr}
  We once again consider the finite field $\mathbb{F}_{4}$.  Consider the $[3, 2]_{4}$ linear code $\mathcal{C}$ with generator matrix:
  \begin{equation*}
    G = \begin{bmatrix}
          \alpha & 0 & 1 + \alpha \\
          1 & 1 & 1
        \end{bmatrix}
  \end{equation*}
  We will show that $\mathcal{C} \vert_{\mathbb{F}_2}$ is nothing but the $[3, 1]_2$ repetition code. Hence we assume that $c \in \mathcal{C} \vert_{\mathbb{F}_2}$, then
  \begin{equation}\label{eq:condition_for_c}
    c = x_1 G_{*, 1} + x_2 G_{*, 2}
  \end{equation}
  for some $x \in \mathbb{F}_4$. Notice that since $c_2 \in \mathbb{F}_2$ we must have $x_2 \in \mathbb{F}_2$, as $G_{0, 1} = 0$ and $G_{0, 2} = 1$. We will show that we must have $x_1 = 0$. Combining the $x_1 G_{*, 1} = \begin{bmatrix}
                      x_1 \alpha & 0 & x_1 + \alpha x_{1}
                   \end{bmatrix}$
  with Equation \ref{eq:condition_for_c} we get that we must have:
  \begin{equation*}
    x_1 + x_2 \in \mathbb{F}_2 \text{ and }  (x_1 + \alpha x_1) + x_2 \in \mathbb{F}_2
  \end{equation*}
  The condition that $x_1 + x_2 \in \mathbb{F}_2$ yields that $x_1 \in \mathbb{F}_2$, since if $x_2 = 1$, then $1 + x_1 \in \mathbb{F}_2$ if and only if $x_1 \in \mathbb{F}_2$ as $1 + \alpha \not \in \mathbb{F}_2$ and $1 + (1 + \alpha) = \alpha \not \in \mathbb{F}_2$. Additionally the condition that $(x_1 + \alpha x_1) + x_2 \in \mathbb{F}_2$ implies that $x_1 = 0$, since $x_{1} = 1$ gives $1 + \alpha + x_2 \in \mathbb{F}_2$ which would imply that $x_2 \in \left\{\alpha, 1 + \alpha \right\}$. Hence $\begin{bmatrix} 1 &  1 & 1 \end{bmatrix}$ is a generator matrix for  $\mathcal{C} \vert_{\mathbb{F}_2}$. Meaning $\mathcal{C} \vert_{\mathbb{F}_2}$ is nothing but the $[3, 1]_2$ repetition code. Hence $d(\mathcal{C} \vert_{\mathbb{F}_2}) = 3$ however $d(\mathcal{C}) = \min\{\wt(c) \vert c \in \mathcal{C} \setminus \left\{0\right\}\} \leq 2$ since $\wt(G_{*, 1}) = 2$.
\end{example}

\begin{remark}
  If we have a $t$-error correcting decoder for $\mathcal{C}$, then we may apply it on $\mathcal{C} \vert_{\mathbb{F}_{q_0}}$, however we might have $d(\mathcal{C} \vert_{\mathbb{F}_{q_0}}) > d(\mathcal{C})$, confer Example \ref{exmp:reduction_to_repr}. Hence there might exist a decoding algorithm with a higher error correcting radius.
\end{remark}

\begin{theorem}\label{thm:min_distance_of_classical_goppa}
  Let $x \in \mathbb{F}_q^{n}$ and $f \in \mathbb{F}_q[X]$ such that $f(x_i) \neq 0$ for all $i = 1, 2, \ldots, n$. Let $\mathbb{F}_{q_0}$ be a subfield of $\mathbb{F}_q$ then:
  \begin{enumerate}
     \item The minimum distance of $\Gamma(x, f, \mathbb{F}_{q_0})$ atleast $\deg(f) + 1$.\label{thm:min_distance_of_classical_goppa1}
     \item Any word $y = c + e$ where $c \in \Gamma(x, f, \mathbb{F}_{q_0})$ and $e \in \mathbb{F}_q^n$ with $\wt(e) \leq \floor{\frac{\deg(f) + 1}{2}}$ can a decoded using either one of Algorithms \ref{alg:basic_decoding_algorithm} and \ref{alg:ECP}.\label{thm:min_distance_of_classical_goppa2}
  \end{enumerate}
\end{theorem}
\begin{proof}
  We start by showing Assertion \ref{thm:min_distance_of_classical_goppa1}. Note that since $\overline{\mathbb{F}}_q = \bigcup_{k = 1}^{\infty} \mathbb{F}_{q^k}$ there exists an $N \in \mathbb{N}$ such that $f$ splits into irreducible factors over $\mathbb{F}_{q^N}$, hence $d\left(\Gamma(x, f, \mathbb{F}_{q^N})\right) = \deg(f) + 1$ by Corollary \ref{cor:f_splitting_yields_design_distance}. The rest follows by Proposition \ref{prop:mininum_distance_of_subfield_subcode}. \\
  Assertion \ref{thm:min_distance_of_classical_goppa2} follows since $\Gamma(x, f, \mathbb{F}_{q_0})$ is a subfield subcode of $\Gamma(x, f, \mathbb{F}_{q^N})$. Furthermore $\Gamma(x, f, \mathbb{F}_{q^N})$ has $\floor{\frac{\deg(f) + 1}{2}}$-error decoding algorithms by Theorems \ref{thm:basic_decoding_algorithm_works} and \ref{thm:error_correcting_pair_for_AG_codes}.
\end{proof}


%Even though the mimimum distance of a
%\begin{example}\label{exmp:sub_field_subcode_of_rs}
%  Let $\mathcal{C}$ be a $[4, 3]_{4}$ Reed Solomon code, then $\mathcal{C} \vert_{\mathbb{F}_2}$ is a linear code over $\mathbb{F}_2$ of length $4$, in particular $\mathcal{C}$ is not a Reed Solomon code. However by combining Proposition \ref{prop:mininum_distance_of_subfield_subcode} and the Singleton bound we see that:
%  \begin{equation*}
%    d(\mathcal{C}) - 1 \leq d(\mathcal{C}\vert_{\mathbb{F}_{2}}) - 1 \leq 4 - \dim(\mathcal{C}\vert_{\mathbb{F}_2})
%  \end{equation*}
%  Now since $\mathcal{C}$ is an MDS code, we see  that $d(\mathcal{C}) = 4 - 3 + 1 = 2$, hence $\dim_{\mathbb{F}_2}(\mathcal{C} \vert_{\mathbb{F}_2}) \leq 3$
%\end{example}

