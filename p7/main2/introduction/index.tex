 \chapter{Introduction}
 Most of the widely used public key cryptosystems are build upon problems in abstract algebra and number theory, that seem hard to solve on classical computers, notable examples include the following problems:
 \begin{problem}[Integer Factorization]\label{prob:int_fac}
   Given $n \in \mathbb{N}$, compute the prime factors of $n$.
 \end{problem}
 \begin{problem}[Discrete Logarithm]\label{prob:disc_log}
   Let $n \in \mathbb{N}$ and $a$ be an element of a group $(G, \circ)$, then given $a^{n} = \underset{n \text{ times} }{\underbrace{a \circ a \circ \cdots \circ a}}$, compute $n$.
 \end{problem}
 Problem \ref{prob:int_fac} underpin the security of the well know RSA cryptostyem, see \cite{alg_lauritzen}[Section 1.9], while Problem \ref{prob:disc_log} underpin more recent cryptosystem, such as the Diffie-Hellman key exchange system\footnote{This scheme actually relies on the fact that given $g^{n}$ and $g^{m}$ it seems to be computationally infeasible to compute $g^{nm}$, however if there exists a method for solving Problem \ref{prob:disc_log}, then this problem is solved trivially.}, see \cite{n_t_and_c}[Section 4.3]. However algorithms for solving these problems, in polynomial time, on a sufficiently powerfull quantum computer have are already well known. The previously mentioned algorithms where initially described by Peter Williston Shor in 1996, see \cite{shor}. Instead the goal of this project will be to study a public key cryptosystem based on the following problem:

 \begin{problem}[General Decoding Problem]\label{prob:general_decoding}
   Let $\mathcal{C} \subseteq \mathbb{F}_q^{n}$ be a linear code, let $t \in \mathbb{N}$ such that $t \leq n$ and $y \in \mathbb{F}_q^n$. Decide if there exists a codeword $c \in \mathcal{C}$, such that $d(c, y) \leq t$
 \end{problem}
 The general decoding problem, has been proved to be $\mathbf{NP}$-complete, see \cite{general_decoding_problem_is_np}, for the case where $q = 2$. For a brief and informal introduction to the topic of $\mathbf{NP}$-completeness we refer the reader to Appendix \ref{app:np_completeness}.

 The public key cryptosystem in question is named after Robert J. McElice, who originally proposed it in 1978.

 The following introduction of the McElice public key cryptosystem (abreviated McElice PKCS), is based upon \cite{r6}. Before we can introduce the McElice PKCS, we will need to introduce some basic concepts from error correcting codes.

\begin{definition}
  Let $\mathcal{C}_1$ and $\mathcal{C}_2$ be $[n, k]_{q}$ codes, with generator matrices $G_1$ and $G_2$ respectively then:
  \begin{enumerate}
    \item If there exists a permutation matrix $P \in  \mathbb{F}_{q}^{k \times k}$ such that $G_1 = G_{2}P$, then $\mathcal{C}_1$ and $\mathcal{C}_{2}$ are called \textit{permutation equivalent}.
    \item The codes $\mathcal{C}_1$ and $\mathcal{C}_2$ are called \textit{equivalent}, denoted $\mathcal{C}_1 \sim \mathcal{C}_{2}$, if there exists a matrix $S \in GL_n(\mathbb{F}_q)$ and permutation matrix $P \in \mathbb{F}_q^{n \times n}$ such that $G_1 = S G_2P$.
  \end{enumerate}
%  then if there exists a permutation matrix $P \in \mathbb{F}_q^{k \times k}$ and a non-singular matrix $S \in \mathbb{F}_q^{n \times n}$ such that $G_1 = SG_2P$, then $\mathcal{C}_1$ and $\mathcal{C}_2$ are called \textit{equivalent}, written $\mathcal{C}_1 \sim \mathcal{C}_{2}$. Additionally if there exits a permutation matrix $Q \in \mathbb{F}_{q}^{k \times k}$ such that $G_1 = QG_2$, then $\mathcal{C}_1$ is called \textit{permutation equivalent} to $\mathcal{C}_{2}$.
  %, then $\mathcal{C}_1$ and $\mathcal{C}_2$ are called \textit{permutation equivalent} if there exists a permutation matrix $P \in \mathbb{F}_q^{k \times k}$ such that $G_1 = G_{2}P$ alternatively if there exists a non-singular matrix $S \in \mathbb{F}_q^{n \times n}$ such that  $G_1 = SG_2P$ then $\mathcal{C}_1$ and $\mathcal{C}_2$ are called \textbf{equivalent}.
\end{definition}
\begin{remark}
  The relation $\sim$, is indeed an equivalence relation:
  \begin{enumerate}
\item If $G$ is a generator matrix, then $G = G$ so $\sim$ is reflective.
\item Additionally if $G_1 = SG_2P$  then $S^{-1}G_1P^{-1} = G_2$ so $\sim$ is symmetric.
\item Finally, to show that $\sim$ is transitive, let $G_1 = S_1G_2P_1$ and $G_2 = S_2G_{3}P_2$. Then $G_1 = (S_1S_2)G_3(P_2P_1)$.
  \end{enumerate}
\end{remark}

Two equivalent codes, clearly share the same dimension, after all their generator matricies have the same number of rows. However it is not immediately obvious that they share the same minimum distance.
\begin{proposition}\label{prop:equvilant_codes_share_parameters}
  Let $\mathcal{C}_1$ and $\mathcal{C}_2$ be equivalent $[n, k]_q$ codes, with generator matricies $G_1$ and $G_2$ respectively, such that $G_1 = SG_2P$, for some $S \in GL_n(\mathbb{F}_q)$ and permutation matrix $P \in \mathbb{F}_q^{k \times k}$: Then:
  \begin{enumerate}
    \item If $H_2 \in \mathbb{F}_q^{n \times k}$ is a parity check matrix of $\mathcal{C}_2$ then the matrix $H_2 P$ is a parity check matrix of $\mathcal{C}_{1}$. \label{prop:equvilant_codes_share_parameters1}
    \item Additionally $d(\mathcal{C}_1) = d(\mathcal{C}_2)$. \label{prop:equvilant_codes_share_parameters2}
  \end{enumerate}
\end{proposition}
\begin{proof}
  We start by proving Assertion \ref{prop:equvilant_codes_share_parameters1}. We do this by showing that $(H_2P)G_1^T = 0_{k \times k}$ meaning $\mathcal{C}_1 \subseteq \Null(H_2P)$ and that $\dim_{\mathbb{F}_q}(\Null(H_2P)) = k$.
  The fact that $\dim_{\mathbb{F}_q}(\Null(H_2P)) = k$ follows from the fact that $P$ is a permutation matrix and hence:
  \begin{equation*}
    \dim_{\mathbb{F}_q}(\Null(H_2P)) = \dim_{\mathbb{F}_q}(\Null(H_{2})) \dim_{\mathbb{F}_q} (\mathcal{C_{2}}) = k
  \end{equation*}
  Next using the fact that $G_1 = SG_2P$ we see that:
  \begin{equation*}
    H_2PG_1^{T} = H_2P(P^TG_2^TS^T) \overset{(a)}{=} H_2G_2^TS^T \overset{(b)}{=} 0_{n \times n}
    % NOTE: har fjernet en (S^T)^{-1}
  \end{equation*}
  where equality $(a)$ follows as $P$ is a permutation matrix and hence orthogonal, and equality $(b)$ as $H_2G_{2}^{T} = 0_{n \times n}$.

  Continuing Assertion \ref{prop:equvilant_codes_share_parameters2} follows by combining Assertion \ref{prop:equvilant_codes_share_parameters1} with the fact that the minimum distance of a code with parity check matrix $H$, corresponds with the minimum number of linearly dependent columns of $H$. Finally we conclude the proof by noting that $H_2P$ has the same columns as $H_2$.
  %We note that $\mathcal{C}_1 \sim \mathcal{C}_2$ implies that there exists a $S \in GL_{n}(\mathbb{F}_{q})$ and a permutation matrix $P \in \mathbb{F}_q^{k \times k}$ such that $G_1 = SG_{2}P$. Hence as $S$ and $P$ are non-singular we know that $rank(G_1) = rank(G_{2})$.
  \label{}
\end{proof}
\begin{definition}\label{def:decoder}
Let $\mathcal{C}$ be a $[n, k, d]_{q}$ code and let $t \leq \floor{\frac{d - 1}{2}}$. A \textit{$t$-error correcting decoder} for $\mathcal{C}$ is a mapping $Dec_{\mathcal{C}}: \mathbb{F}_q^n \to \mathcal{C} \cup \left\{?\right\}$ which satisfies the condition that $Dec_{\mathcal{C}}(y) = c$ whenever $y = c + e$, with $c \in \mathcal{C}$ and $e \in \mathbb{F}_q^n$ such that $\wt(e) \leq t$, and $Dec_{\mathcal{C}}(y) = \; ?$ otherwise. An algorithm which implements a $t$-error correcting decoder for $\mathcal{C}$ is called a \textit{$t$-error correcting decoding algorithm} for $\mathcal{C}$.
\end{definition}
\begin{remark}
  We often simply refer to the $t$-error correcting decoder $Dec_{\mathcal{C}}$ as a \textit{decoder} and any algorithm which implements a decoder for $\mathcal{C}$ as an \textit{decoding algorithm} for $\mathcal{C}$.
\end{remark}

%Let $\mathcal{C}$ be a $[n, k, d]_q$ code with generator matrix $G \in \mathbb{F}_q^{n \times k}$, such that we have an efficient algorithm $dec_{\mathcal{C}}$ for decoding codewords up to $t$ errors. Furthermore we let $S \in \mathbb{F}_q^{k \times k}$ be a random non-singular matrix and $P \in \mathbb{F}_q^{n \times n}$ be a random permutation matrix.
%The secret key will be the tripe $(Dec_{\mathcal{C}}, S, P)$ and the public key will be $G' = S \cdot G \cdot P$. To encrypt a $k$-length message $m$, we will compute $y = m G' + e$, such that $e \in \mathbb{F}_q^{n}$ is a random vector with $\wt(e) \leq t$. The decryption of the message is slightly more complicated, the algorithm is given in Algorithm \ref{alg:McElice_decrypt}.

We have now covered all of the necessary tools needed to introduce the McElice PKCS:
Let $\mathcal{C}$ be an $[n, k, d]_q$ code with generator matrix $G$ and an efficient $t$-error correcting decoding algorithm $Dec_{\mathcal{C}}$, furthermore let $S \in GL_{k \times k}(\mathbb{F}_q)$ and $P \in \mathbb{F}_q^{n \times n}$ be a random permutation matrix. An overview of the McElice PKCS is given below in Algorithm \ref{alg:McElice}.
\begin{algorithm}
\caption{McElice PKC}\label{alg:McElice}
\begin{algorithmic}
  \State \textbf{private key} $(Dec_{\mathcal{C}}, S, P)$
  \State \textbf{public key} $(G' = S \cdot G \cdot P, t)$
  \\
  \Procedure{McElice Encryption} {$m$: plain message}
    \State $c \gets m^{T} G'$
    \State Choose $e \in \mathbb{F}_q^n$ uniformly, such that $\wt(e) = t$.
    \State \Return $c + e$
  \EndProcedure \\
  \Procedure{McElice Decryption} {$y$: received chipher-text}
    \State $y' \gets y P^{-1}$
    \State $m' \gets Dec_{\mathcal{C}}(y')$
    \State \Return $m' S^{-1}$
  \EndProcedure
\end{algorithmic}
\end{algorithm}

\begin{proposition}\label{prop:decryption_algorithm_produces_correct_message}
  Given some $y = m^{T}G' + e$, with $e \in \mathbb{F}_q^{n}$ such that $\wt(e) = t$, the decryption algorithm described in Algorithm \ref{alg:McElice} yiels the correct message $m^{T}$.
\end{proposition}
\begin{proof}
  Sticking to the notation used in Algorithm \ref{alg:McElice} we have:
  \begin{equation*}
    y' := (m^{T} G' + e) P^{-1} = m^{T}SG + e P^{-1}
  \end{equation*}
  However as $P^{-1} = P^{T}$ is also permutation matrix, we see that $\wt(eP^{-1}) = t$. Hence we may apply our $t$-error correcting decoder $Dec_{\mathcal{C}}$ to $y'$ and get $m' := m^{T}S$ now multiplying $m'$ by $S^{-1}$ we obtain $m^{T}$.
\end{proof}

\begin{remark}
  The matrix $G'$ will be the generator matrix of another $[n, k]_{q}$ code $\mathcal{C}'$. By Proposition \ref{prop:equvilant_codes_share_parameters} $d(\mathcal{C}') = d(\mathcal{C})$. Hence the decoding of $m^T G' + e$ also makes sense. In fact the decoding $m^T G' + e$ is one of the ways to attack the McElice PKCS. We will discuss this in more detail in Sections \ref{sec:syndrome_decoding} and \ref{sec:information_set_decoding}.
\end{remark}

We mentioned earlier that we would study public key cryptosystem which where bases on the general decoding problem, See Problem \ref{prob:general_decoding}. However the McElice PKCS is actually based on the following ``weaker'' problem, in the sense that an efficient algorithm for solving this problem doesn't necessarily yield an efficient solution to Problem \ref{prob:general_decoding}. However if $\mathbf{NP} = \mathbf{P}$, meaning that Problem \ref{prob:general_decoding} could be solved in polynomial time, then the McElice PKCS would be venerable to attack.
\begin{problem}[McElice Problem]\label{prob:McElice}
  Given $(G', t)$ and a ciphertext $y$ find the unique $m \in \mathbb{F}_q^{k}$ such that $\wt(m^{T}G' - c) = t$.
\end{problem}
The main difference between Problem \ref{prob:McElice} and Problem \ref{prob:general_decoding} is that we are provided with a basis of $\mathcal{C}'$ in Problem \ref{prob:McElice}, since we know $G'$. However the idea is to have as little structure of the underlying code revealed by $G'$ as possible, to make the problem closer to Problem \ref{prob:general_decoding}.

In general we have two kinds of attacks on the McElice PKCS
\begin{enumerate}
    \item \textit{Structural Attacks}: Where we try to extract information about the underlying code, using $G'$. If we obtain enough information we may be able to implement our own efficient $t$-error correcting decoding algorithm.
    \item \textit{Generic Attacks}: Where we try to construct efficient $t$-error correcting decoding algorithms given $(G', t)$, without concerning our selves with the underlying structure of $\mathcal{C}$, we will refer to such decoding algorithms as being \textit{generic}.
\end{enumerate}

Two of the main goals of this project is to investigate these attacks. Understanding of structural attacks allows us to gain insights into which families of codes constitutes ``good'' candidates for use in the McElice PKCS. While generic attacks allows us to measure the security of the McElice PKCS constructed on the codes resilient to structural attacks.

 \section{The Niederreiter Public Key Cryptosystem}

Suppose $\mathcal{C}$ is a $[n, k, d]_q$ code with parity check matrix $H$ and an efficient $t$-error correcting decoding algorithm $dec_{\mathcal{C}}$. Furthermore let $S \in GL_{(n - k) \times (n - k)}(\mathbb{F}_q)$ and $P \in \mathbb{F}_q^{n \times n}$ be a random permutation matrix. The Neiderreiter PCKS (based on $\mathcal{C}$), which allows for encryption of a plain message $e \in \mathbb{F}_q^n$ with $\wt(e) = t$, is described below in Algorithm \ref{alg:Neiderreiter}.

\begin{algorithm}
\caption{The Neiderreiter PKCS}\label{alg:Neiderreiter}
\begin{algorithmic}
  \State \textbf{private key} $(Dec_{\mathcal{C}}, S, P)$
  \State \textbf{public key} $(H' = S \cdot H \cdot P, t)$
  \\
  \Procedure{Neiderreiter Encryption} {$e$: plain message with $\wt(e) = t$}
    \State \Return $H'e$
  \EndProcedure \\
  \Procedure{Neiderreiter Decryption} {$y$: received ciphertext}
    \State Find $z \in \mathbb{F}_q^n$ such that $Hz = S^{-1}y$ using linear algebra.
    \State $c \gets Dec_{\mathcal{C}}(z)$
    \State \Return $c P^{-1}$
  \EndProcedure
\end{algorithmic}
\end{algorithm}

Finding a $z \in \mathbb{F}_q^{n}$ such that $Hz = S^{-1}y$ is simply a matter of solving a linear system. Additionally since $S^{-1}y = S^{-1}SHPe = HPe$ we see that $c := Dec_{\mathcal{C}}(z)$ is the closest codeword to $HPe$ so $e = z - c P^{-1}$.

\subsection{The Equivalence Between the Neiderreiter and the McElice PCKS}%

Consider the $[n, k, d]_{q}$ code $\mathcal{C}$ with generator matrix $G$ and parity check matrix $H$. We demonstrate that the McElice and Neither cryptosystems based on $\mathcal{C}$ have an equivalent level of security. That is if a there exists an efficient attack on the McElice PCKS (based on $\mathcal{C}$), then there exists an efficient attack on the Neither PCKS (based on $\mathcal{C}$).

Hence we let $G'$ and $H'$ be the public keys of the McElice and Neither PCKS respectively.

Suppose we have a message $m \in \mathbb{F}_q^k$, we encrypt the message a obtain $y \in \mathbb{F}_q^{n}$, using Algorithm \ref{alg:McElice}. That is:
\begin{equation*}
  y = m^T G' + e
\end{equation*}
where $e \in \mathbb{F}_q^n$ with $\wt(e) = t$. Given $G'$ we may obtain a parity check matrix $H'$ for the code generated confer Theorem \ref{thm:from_generator_matrix_to_parity_check_matrix}. Multiplying by $(H')^T$ we get:
\begin{equation}\label{eq:Neiderreiter_and_McElice}
  y(H')^T = mG'(H')^T + e (H')^T = e (H')^T
\end{equation}
where the last equality follows as $G'(H')^T = 0$. Additionally since $y$ and $H'$ are public, the righthand side of Equation \eqref{eq:Neiderreiter_and_McElice} can easily be computed. Furthermore since $\wt(e) = t$, we see that we may compute the error $e$ efficiently provided we have an efficient attack on the Neiderreiter PCKS. Hence an efficient attack on the Neiderreiter PCKS would lead to an efficient attack on the corresponding McElice PCKS with very little overhead.

Conversely assume that we have a message $e \in \mathbb{F}_q^{n}$ such that $\wt(e) = t$. If we encrypt the message to obtain $y \in \mathbb{F}_q^{(n - k)}$, using Algorithm \ref{alg:Neiderreiter}. That is:
\begin{equation*}
  y = H'e
\end{equation*}
Again we may obtain generator matrix $G'$ of the code $\Null(H')$, since the null space of a matrix is invariant under row operations, thus we may apply Theorem \ref{thm:from_generator_matrix_to_parity_check_matrix}. Using basic linear algebra one my find a vector $z \in \mathbb{F}_q^n$ with $\wt(z) \geq t$ such that $y = H'z$ after all $d$ is the minimum number of linearly independent columns of $H'$ and $t \leq \floor{\frac{d - 1}{2}}$. Hence:
\begin{equation*}
  y = H'z \text{ and } z = yG' + e
\end{equation*}
Hence $e$ could be extracted efficiently provided that there exists an efficient algorithm for breaking the McElice PCKS.


\section{Advantages and Drawbacks of McElice and Neiderreiter}%
In this final subsection we will briefly discuss some of the advantages and drawbacks of using each system. Both compared to each other and compared to traditional public key cryptography systems. The results are summarized in Table \ref{tab:pros_and_cons}. The encryption procedure of the McElice PCKS is very efficient additionally there exists no well known quantum algorithm for breaking the McElice PCKS. The primary con of the McElice PCKS is the large key size, for example the original proposal by McElice, used a $[1024, 524]_2$ classical Goppa code, see Definition \ref{def:classical}. Due to the dimensions of the code the public key was roughly $524 \cdot 1024 = 536576$ bits or about $67.1$ KB \textcolor{blue}{while yielding a security level of about $65$ bits (remember to do the actual calculation). Meaning an attacker would have to perform $2^{65}$ operations to break the encryption.}. In comparison using RSA a public key size of only $3072$ bits is sufficient to yield a security level of $128$ bits, refer \cite{nist_recomendations_for_key_management}[Table 2].

For this reason much research has been done with the objective of lowering the public key size. However even though many of these proposals have succeeded in lowering the public key size, they often come with security issues.

The Neiderreiter PCKS has similar drawbacks as the McElice PCKS, in that the key size is very large compared to traditional public key cryptography systems, like RSA. However it has an additional drawback compared to the McElice PCKS, namely that the message has to have weight $t$ while having length $n$. Where as the only restriction on the message in the McElice PCKS is that it should have length $n$. On the otherhand one of the advantages of using the Neiderreiter PCKS is that it offers a smaller key size. Since the parity check matrix $H'$ may be published in systematic form, that is $H' = [A | I_{(n - k) \times (n - k)}]$, while keeping the security level the same, we prove this below in Proposition \ref{prop:H_can_be_in_standard_form}. We will however first need a small lemma:

\begin{lemma}\label{lem:equal_syndrome_and_equal_weight}
  Let $H$ be parity check matrix for the $[n, k, d]_q$ code $\mathcal{C}$, $t \leq \floor{\frac{d - 1}{2}}$ and $x, x' \in \mathbb{F}_q^n$. Then $\wt(x) = \wt(x') = t$ and $S_H(x) = S_H(x')$ implies $x = x'$.
\end{lemma}
\begin{proof}
  By Lemma \ref{lem:syndrome_is_the_same_iff_they_are_in_the_same_coset} we have $x = x' + c$ for some $c \in \mathcal{C}$. Thus $c = x - x'$ which implies:
  \begin{equation*}
    \wt(c) = \wt(x - x') \leq 2t < d
  \end{equation*}
  since $\wt(x) = \wt(x') = t$. Thus $c$ must equal zero.
\end{proof}

\begin{proposition}\label{prop:H_can_be_in_standard_form}
  Let $H$ be a parity check matrix for the $[n, k, d]_q$ code $\mathcal{C}$. Furthermore let $H' := UH$ be a systematic parity check matrix of $\mathcal{C}$, then any attack able to break the scheme using $H'$ is able to break a scheme using $H$.
\end{proposition}
\begin{proof}
  Let $W_t = \left\{x \in \mathbb{F}_q^n | \wt(x) = t\right\}$ with $t \leq \floor{\frac{d - 1}{2}}$. Then $S_H$ and $S_{H'}$, restricted to $W_{t}$, are injective by Lemma \ref{lem:equal_syndrome_and_equal_weight}. Assume that $\phi$ breaks the system using $H'$, that is $x = \phi(y) = S_{H'}^{-1}(y)$ for all $y \in S_{H'}(W_t)$. \\
  Next suppose $y \in S_{H}(W_t)$, then $y = S_{H}(x) = Hx$ for some $x \in W_t$. Hence:
  \begin{equation*}
    Uy = U S_H(x) = U H x = S_{H'}(x) \in S_{H'}(W_t)
  \end{equation*}
   and $\phi(Uy) = x$. That is $\phi$ can be used to break the system using the parity check matrix $H$, we note that $U$ can be obtained by performing elementary row operations on $H$.
\end{proof}

Since $H$ is allowed to be in standard form, may simply publish the first $k$ columns of $H$ assuming $H$ is in systematic form, allowing for a much smaller public key size at least when $k$ is relatively large compared to $n$. As an example consider McElices original proposal which used a $[1024, 524]_2$ classical Goppa code, meaning the public key of the equivalent neither system, could be published using $(1024-524) \cdot 524 = 26200$ or about $32.8$ KB, which is about half the size of the public key in the equivalent McElice PCKS. Finally the contents of the discussion is summarized in Table \ref{tab:pros_and_cons}.
\begin{table}[H]
    \centering
    \begin{tabular} {||c|c|c||}
        \hline
        \textbf{System} & \textbf{Advantages} & \textbf{Drawbacks} \\
        \hline
        McElice & \makecell{Fast encryption\\No known efficient quantum attacks}& Large key size (versus e.g. RSA) \\
        \hline

        Neiderreiter & \makecell{Smaller key size (versus McElice) \\ No known efficient quantum attacks} & \makecell{Large key size (versus e.g. RSA)\\ Messages must be of weight $t$}\\
        \hline
    \end{tabular}
    \caption{Advantages and drawbacks of the McElice and Neiderreiter PCKS.}
    \label{tab:pros_and_cons}
\end{table}

