\chapter*{Notation and Shorthands}
\textbf{Abstract Algebra}

\begin{table}[H]
    \begin{tabular}{ll}
      $\F_{q}$ & The finite field with $q$ elements. \\
      $R[X_1, X_2, \ldots, X_{n}]$ & The multivariate polynomial ring over $R$. \\
      $X^{(k)}$ & The mononomial $\prod_{i = 1}^{n} X_{i}^{k_{i}}$. \\
      $R^{*}$ & The set of units in $R$. \\
      $\gen{S}$ & The ideal generated by the elements in the set $S$. \\
      $\mathbb{K}/\mathbb{F}$ & A field extension, meaning $\mathbb{F}$ is a subfield of the field $\mathbb{K}$. \\
      $\cF$ & An algebraic closure of the field $\mathbb{F}$. \\
      $Rad(I)$ & The radical of the ideal $I$. \\
      $\ev$ & The evaluation map. \\
    \end{tabular}
\end{table}

\textbf{Algebraic Geometry}
\begin{table}[H]
    \begin{tabular}{ll}
      $\K$ & An arbitrary algebraically closed field. \\
      $\A^{n}, \mathbb{P}^{n}$ & The $n$-dimensional affine and projective space. \\
      $V(S), V_{\mathbb{P}}(S)$ & The affine and projective zero sets a set of polynomials $S$. \\
      $I(X), I_{\mathbb{P}}(X)$ & The affine and projective vanishing ideals a set of points $X$ \\
      $\Zar$ & The Zariski topology. \\
      $F^{*}, F_{*}$ & The homogenisation and dehomogenisation of the polynomial $F$. \\
      $\mathcal{X}$ & An affine or projective variety. \\
      $\K[\mathcal{X}], \K(\mathcal{X})$ & The coordinate ring and function field of the affine variety $\mathcal{X}$. \\
      $\K_{\mathbb{P}}[\mathcal{X}], \K_{\mathbb{P}}(\mathcal{X}), \K[\mathcal{X}]$ & The homogeneous coordinate ring and homogeneous function field and function field of the projective variety $\mathcal{X}$. \\
      $\K(\mathcal{X})$ & The function field over $\mathcal{X}$. \\
      $\mathcal{O}_{P}(\mathcal{X}), \mathfrak{m}_{P}$ & The local ring of $\mathcal{X}$ at $P$, and it's maximal ideal. \\
      $v_P$ & The discrete valuation on $\mathcal{O}_{P}(\mathcal{X})$. \\
      $I(P, \mathcal{X}, \mathcal{Y})$ & The intersection multiplicity of $\mathcal{X}$ and $\mathcal{Y}$ at the point $p$. \\
      $Div(\mathcal{X})$ & The set of divisors on $\mathcal{X}$. \\
      $\support(D)$ & The support of a divisor $D$. \\
      $(f)$ & The principal divisor of $f \in \K[\mathcal{X}]$. \\
      $L(D), \ell(D)$ & A special vector space of divisors and it's dimension. \\
    \end{tabular}
\end{table}

\newpage
\textbf{Coding theory}
\begin{table}[H]
    \begin{tabular}{ll}
      $\C$ & A linear code. \\
      $G$ & A generator matrix. \\
      $H$ & A parity check matrix. \\
      $\d$ & The Hamming metric. \\
      $\wt$ & The Hamming weight. \\
      $\overline{B_{r}}(x)$ & The hamming ball of radius $r$ with center in $x$. \\
      $H_{q}$ & The $q$-ary entropy function. \\
      $\C^{\perp}$ & The dual code of $\mathcal{C}$. \\
      $\delta$ & The relative distance. \\
      $R$ & The relative distance. \\
    \end{tabular}
\end{table}

\textbf{Algebraic Geometry Codes}
\begin{table}[H]
    \begin{tabular}{ll}
      $\C_{D, G}$ & A Goppa Code. \\
      $\psi$ & The extended Frobenius map \\
      $N_{q}(\mathcal{X})$ & The number of $\mathbb{F}_{q}$-rational points on $\mathcal{X}$ \\
      $N_{q}^{*}(g)$ & The maximum number of rational points on a curve of genus $g$ \\
    \end{tabular}
\end{table}
