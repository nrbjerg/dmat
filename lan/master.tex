% master.tex : master file for the project
% ------------------------------------------------------------------------------
% This is the main file in the project, which collects the contents from all the
% input files (text, images, literature databases, etc.).

% The 'book' class is a document class with many flexible options.
% See https://tex.stackexchange.com/a/36989/118167
\documentclass[11pt,a4paper,oneside,openright,english]{article}

% Some top level variables that are used to automatically input title, authors,
% etc. in the title page and front page.

% The preamble, i.e. all the settings and commands that go before actual
% document contents, in this template is handled in the single file aaumath.sty,
% which defines a package that can be loaded here with \usepackage.
%\usepackage{aaumath}
% All contents of the document go between the \begin and \end commands for the
% 'document' environment.
\include{incl/pre/pkgs}
\include{incl/pre/cmds}
\include{incl/pre/conf}

\title{Linear Algebra Notes}
\author{Martin Sig Nørbjerg}
\date{\today}

\begin{document}
\maketitle
\tableofcontents

% The front matter is not counted in the numbered pages and are instead numbered
% with roman numerals. This consists of, for exmp, the front page, title
% page, preface, and table of contents.
% The main matter is were the bulk of your work goes. Pages and headings have
% arabic numbers.

\section{Introduction}
Throughout these notes we let $\mathbb{F}$ denote a field, either $\mathbb{F} = \mathbb{R}$ (the field of real numbers) or $\mathbb{F} = \mathbb{C}$ (the field of complex numbers). However the most of the results of linear algebra generalizes to arbitrary fields.


\section{Linear Systems}
Suppose we have a system of linear equations:
\begin{align*}
  x + 2y - z = 2 \\
  y  + z = 1
\end{align*}
Then this linear system can be represented as a matrix vector multiplication
\begin{equation*}
  \begin{bmatrix} 1 & 2 & -1 \\ 0 & 1 & 1 \end{bmatrix} \begin{bmatrix} x \\ y \\ z \end{bmatrix} = \begin{bmatrix} 2 \\ 1 \end{bmatrix}
\end{equation*}
\subsection{Augmented matricies}


\section{Vector Spaces}

\begin{definition}
  A \textit{vectorspace} $V$ is a
\end{definition}

\section{Eigenvalues and Eigenvectors}
Eigenvalues and Eigenvectors are in my opinion the crown jewel's of linear algebra. They tie together every concept introduced so far.

\begin{definition}
  Let $A \in \mathbb{F}^{n \times n}$, suppose that there exists $\lambda \in \mathbb{F}$ and a non-zero $\mathbf{v} \in \mathbb{F}^{n}$ then $\lambda$ is called an \textit{eigenvalue} and $\mathbf{v}$ and \textit{eigenvector} of $A$ if and only if:
  \begin{equation}\label{eq:eigen_cond}
    A \mathbf{v} = \lambda \mathbf{v}
  \end{equation}
\end{definition}

\begin{remark}
  It is important to note that $\mathbf{v} = \mathbf{0}$ is never an eigenvector, for technical reasons which will become apparent later. However it does obey Equation \eqref{eq:eigen_cond}.
\end{remark}

\begin{example}\label{exmp:}
  Show that $v = \begin{bmatrix} 1 & -1 \end{bmatrix}^{T}$ is an eigenvector of $A = \begin{bmatrix}  \end{bmatrix}$ and find the corresponding eigenvalue $\lambda$.

  \textit{Solution:}
\end{example}
\subsection{Computing Eigenvalues and Eigenvectors}

Let $A \in \mathbb{F}^{n \times n}$, suppose we want to find an
Suppose $\mathbf{v}$ is an eigenvector of $A$ with eigenvalue $\lambda$, that is $A \mathbf{v} = \lambda \mathbf{v}$. Then:
\begin{equation*}
  \mathbf{0} = A \mathbf{v} - \lambda \mathbf{v} = A \mathbf{v} - \lambda I \mathbf{v} = (A - \lambda I) \mathbf{v}
\end{equation*}

If the matrix $A - \lambda I$ is invertible, then:
\begin{equation*}
  (A - \lambda I)^{-1}\mathbf{0} = \mathbf{v}
\end{equation*}
Hence we must have $\mathbf{v} = 0$ by Lemma \ref{lem:x=0_implies_Ax=0}. However for $\mathbb{v}$ to be an eigenvector we must have $\mathbb{v} \neq \mathbf{0}$. Hence $(A - \lambda I)$ cannot be invertible and hence:
\begin{equation*}
  \det(A - \lambda I) = 0
\end{equation*}

This leads us to the following theorem:
\begin{theorem}\label{thm:char_polynomial}
  Let $A \in \mathbb{F}^{n \times n}$ be a matrix, then the \textit{charateristic polynomial} of $A$ is:
  \begin{equation*}
    f(\lambda) = \det(A - \lambda I)
  \end{equation*}
  Furthermore $\lambda' \in \mathbb{F}$ is an eigenvalue of $A$ if and only if $f(\lambda') = 0$.
\end{theorem}

After we have found an eigenvalue $\lambda'$ using Theorem \ref{thm:char_polynomial} we may obtain an eigenvector by solving the system:
\begin{equation*}
  (A - \lambda'I) \mathbb{v} = \mathbb{0}
\end{equation*}
We will get a parametric solution. The basis of the solution space is a basis of the eigenspace associated with $\lambda'$.

\begin{example}\label{exmp:eigen_value_using_char_polynomial}
  Consider the matrix $A = \begin{bmatrix} 2 & 5 \\ -1 & 3 \end{bmatrix}$ find the eigenvalues of $A$ and their associated eigenspaces.

  \textit{Solution:}
\end{example}


\section{Factorizations}


% Input files should be partitioned such that each file corresponds to one
% chapter. The \include command forces a page break and inserts the contents of
% the input file.


% Appendices are included inside an appendices block, which enumerates chapters
% with letters, starting from A, instead of numbers.

% The backmatter is for extra stuff. Headings are not numbered.

% Automatic list of references, based on which references in the literaturery
% database files were referenced throughout the document.

%\bibliographystyle{apalike_new}

\end{document}
