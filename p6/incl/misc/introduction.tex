\chapter*{Introduction\markboth{Introduction}{Introduction}}\addcontentsline{toc}{chapter}{Introduction}
We start by introducing the concept of a linear error correcting code in Chapter \ref{chap:error_correcting_codes}. Our disposition of linear error correcting codes includes a section on bounds on the parameters of an error correcting code. In this section we prove the Singleton bound (Corollary \ref{cor:singleton_bound}) and show that Reed-Solomon codes are MDS codes (Example \ref{exmp:rs_codes}). Afterwards we prove the Gilbert bound (Corollary \ref{cor:gilbert_bound}) and the asymptotic Gilbert-Varshamov bound (Theorem \ref{thm:gilbert_varshamov}).

Chapter \ref{chap:geom} is on the theory of algebraic geometry, which is the study of the geometry of the zero sets of polynomials. The chapter is divided into three sections:
\begin{enumerate}[label=\arabic*.]
  \item The first section is devoted to algebraic preliminaries, most notably the definition of different kinds of multivariate polynomials, algebraically closed fields (Definition \ref{def:alg_closed}) and results on Noetherian rings, including Hilbert's Basis Theorem \ref{thm:hbt}.
  \item In the second section the basics of algebraic geometry is introduced, however our disposition on the topic is quite limited. We introduce affine and projective varieties (Definitions \ref{def:affine_variety} and \ref{def:projective_variety} respectively) and various algebraic objects associated with these varieties.
  \item The third and final section is devoted to the study of algebraic curves, especially the theory of algebraic plane curves. Our focus will be on the study of divisors and the Reimann-Roch Theorem \ref{thm:reimann_roch} which we state without proof.
\end{enumerate}

In Chapter \ref{chap:alg_geom_codes} we introduce Goppa codes, which is a special type of error correcting curve, constructed on an absolutely irreducible regular projective algebraic curve over $\F_{q}$. The parameters of these codes are investigated using the theory of divisors on algebraic curves. We show that Reed-Solomon codes are a subfamily of Goppa codes (Example \ref{exmp:rs_is_goppa}). Finally, we discuss the Tsfasman-Vlădut-Zink bound (Equation \ref{eq:TVZ_bound}), which shows the existence of sequences of Goppa codes which exceed the Gilbert-Varshamov bound.
