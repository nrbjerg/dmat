\documentclass[10pt]{beamer}

\newcommand{\K}{\mathbbm{k}}
\newcommand{\F}{\mathbb{F}}
\newcommand{\cF}{\overline{\F}}

\usetheme[
%%% options passed to the outer theme
%    hidetitle,           % hide the (short) title in the sidebar
%    hideauthor,          % hide the (short) author in the sidebar
%    hideinstitute,       % hide the (short) institute in the bottom of the sidebar
%    shownavsym,          % show the navigation symbols
%    width=2cm,           % width of the sidebar (default is 2 cm)
    hideothersubsections,% hide all subsections but the subsections in the current section
%    hideallsubsections,  % hide all subsections
    left               % right of left position of sidebar (default is right)
%%% options passed to the color theme
%    lightheaderbg,       % use a light header background
  ]{AAUsidebar}

% If you want to change the colors of the various elements in the theme, edit and uncomment the following lines
% Change the bar and sidebar colors:
%\setbeamercolor{AAUsidebar}{fg=red!20,bg=red}
%\setbeamercolor{sidebar}{bg=red!20}
% Change the color of the structural elements:
%\setbeamercolor{structure}{fg=red}
% Change the frame title text color:
%\setbeamercolor{frametitle}{fg=blue}
% Change the normal text color background:
%\setbeamercolor{normal text}{bg=gray!10}
% ... and you can of course change a lot more - see the beamer user manual.

%\documentclass[danish,12pt,a4paper]{report}
\usepackage{wrapfig}
\usepackage{amsmath,amsfonts,amssymb,amsthm}
\usepackage{mathtools}
\usepackage{commath}
%\usepackage{enumitem}
\usepackage{booktabs}
\usepackage{bm}
\usepackage{multicol}
\usepackage{setspace}
\usepackage{thmtools}
\usepackage{kpfonts}
\usepackage{array}
\usepackage[makeroom]{cancel}
\usepackage{times}
\usepackage{multicol}
\DeclareMathOperator{\Mul}{\mathrm{Mul}}
\usepackage{placeins}
\useinnertheme{default}
\usepackage{tabularx}
\beamertemplatenavigationsymbolsempty
\newcommand{\ny}{\vskip 0.5cm}
\newcommand{\bo}{\vskip 2mm}
\newcommand{\lis}{\vskip 3.5mm}
\newcommand{\booglis}{\vskip 5.5mm}
\newcommand{\pizza}{\vskip 6.2mm}
\newcommand{\technodromen}{\-\hspace{0.5mm}}
\newcommand{\mel}{\-\hspace{5mm}}
\newcommand{\ahmed}{\bigskip \bigskip \bigskip \bigskip \bigskip}
\newcommand{\krang}{\vskip 1mm}
\newcommand{\vitas}{\vskip 0.5mm}
\newcommand{\Lim}[1]{\raisebox{0.5ex}{\scalebox{0.8}{$\displaystyle \lim_{#1}\;$}}}
\makeatletter
\newcommand*\bollet{\mathpalette\bollet@{0.72}}
\newcommand*\bollet@[2]{\mathbin{\vcenter{\hbox{\scalebox{#2}{$\m@th#1\bullet$}}}}}
\newcommand{\koege}{\-\hspace{1mm}}
\newcommand{\creme}{\-\hspace{0.25mm}}
\newcommand{\hrniels}{\-\hspace{0.1mm}}
\newcommand{\john}{\koege\technodromen \wedge \koege\technodromen}
\newcommand{\aal}{\technodromen \wedge \technodromen}
\newcommand{\bimp}{\koege \Leftrightarrow \koege}
\newcommand{\storbimp}{\-\hspace{3.5mm} \Leftrightarrow \-\hspace{3.5mm}}
\newcommand{\medf}{\koege \Rightarrow \koege}
\newcommand{\stormedf}{\-\hspace{3.5mm} \Rightarrow \-\hspace{3.5mm}}
\newcommand{\drjan}{\koege | \koege}
\newcommand{\sortesambo}{\technodromen \left|\frac{}{}\right.}
\newcommand{\ost}{^{\ast}}
\newcommand{\worm}{\input{Billeder/Regnorm.tex}}
\newcommand{\ormekongen}{\input{Billeder/Ormekongen.tex}}
\newcommand{\kr}{\text{ kr.}}
\makeatother
\newcommand{\LL}{\mathcal{L}}
\newcommand{\ro}[1]{%
  \xrightarrow{\mathmakebox[\rowidth]{#1}}%
}


% ¤¤ Opsaetning af figur- og tabeltekst ¤¤ %
\usepackage[font=footnotesize,labelfont={it,bf},format=hang]{caption}
\usepackage{graphicx,scalerel}

	% Indsaet figurer nemt med \figur{Stoerrelse}{Fil}{Figurtekst}{Label}
\newcommand{\figur}[4]{
		\begin{figure}[H] \centering
			\includegraphics[width=#1\textwidth]{billeder/#2}
			\caption{#3}
			\label{#4}
		\end{figure}
}

\usepackage{subfig}
\usepackage{etoolbox}

\usepackage{algpascal}
\usepackage{algorithm} 
\usepackage{algpseudocode}
\usepackage{blkarray}

\newcommand\restr[2]{\ensuremath{\left.#1\right|_{#2}}}

\newcommand{\tdist}{\textnormal{t}}
\newcommand{\Fdist}{\textnormal{F}}
\newcommand{\J}{\mathcal{J}}
\renewcommand{\S}{\bm{S}}
\newcommand{\I}{\mathcal{I}}
\newcommand{\R}{\mathbb{R}}
\newcommand{\Z}{\mathbb{Z}}
\newcommand{\A}{\mathcal{A}}
\newcommand{\N}{\mathbb{N}}
\newcommand{\C}{\mathcal{C}}
\newcommand{\M}{\mathcal{M}}
\newcommand{\B}{\mathcal{B}}
\renewcommand{\d}{\mathrm{d}}
\newcommand{\eps}{\varepsilon}
\newcommand{\e}{\mathrm{e}}
\newcommand{\E}{\mathcal{E}}
\newcommand{\tr}{\mathrm{tr }}
\newcommand{\F}{\mathbb{F}}
\renewcommand{\L}{\mathcal{L}}
\renewcommand{\Re}{\mathrm{Re}}
\renewcommand{\Im}{\mathrm{Im}}
\newcommand{\spand}{\mathrm{span}}
\newcommand\isom{\xrightarrow{
   \,\smash{\raisebox{-0.65ex}{\ensuremath{\scriptstyle\sim}}}\,}}
\newcommand\sbullet[1][.5]{\mathbin{\ThisStyle{\vcenter{\hbox{%
  \scalebox{#1}{$\SavedStyle\bullet$}}}}}%
}
\newcommand\scalemath[2]{\scalebox{#1}{\mbox{\ensuremath{\displaystyle #2}}}}

\renewcommand{\vec}[1]{\bm{#1}}
\usepackage{multirow}
\usepackage[normalem]{ulem}
\useunder{\uline}{\ul}{}
\newcommand{\Var}{\textnormal{Var}}
\newcommand{\unif}{\textnormal{unif}}
\newcommand{\geom}{\textnormal{geom}}
\newcommand{\Poi}{\textnormal{Poi}}

\usepackage[utf8]{inputenc}
\usepackage[english]{babel}
\usepackage[T1]{fontenc}
% Or whatever. Note that the encoding and the font should match. If T1
% does not look nice, try deleting the line with the fontenc.
\usepackage{helvet}
\usepackage{tikz}
\DeclareMathOperator*{\argmin}{arg\,min}

% colored hyperlinks
\newcommand{\chref}[2]{%
  \href{#1}{{\usebeamercolor[bg]{AAUsidebar}#2}}%
}
\vspace{1.5cm}
\title[]% optional, use only with long paper titles
{\Large Algebraic Geometry Codes\\
\vspace{0.3cm}}





\subtitle{}  % could also be a conference name

\date{\no}

\author[Martin Sig Nørbjerg] % optional, use only with lots of authors
{Group 5.239A\\
 Martin Sig Nørbjerg
}
% - Give the names in the same order as they appear in the paper.
% - Use the \inst{?} command only if the authors have different
%   affiliation. See the beamer manual for an example

\institute[
%  {\includegraphics[scale=0.2]{aau_segl}}\\ %insert a company, department or university logo
  Institute of Mathematics\\
  Aalborg University
] % optional - is placed in the bottom of the sidebar on every slide
{% is placed on the title page
  Institute of Mathematics\\
  Aalborg University
  
  %there must be an empty line above this line - otherwise some unwanted space is added between the university and the country (I do not know why;( )
}


% specify a logo on the titlepage (you can specify additional logos an include them in 
% institute command below
\pgfdeclareimage[height=1.5cm]{titlepagelogo}{AAUgraphics/aau_logo_new} % placed on the title page
%\pgfdeclareimage[height=1.5cm]{titlepagelogo2}{graphics/aau_logo_new} % placed on the title page
\titlegraphic{% is placed on the bottom of the title page
  \pgfuseimage{titlepagelogo}
%  \hspace{1cm}\pgfuseimage{titlepagelogo2}
}

\usefonttheme{professionalfonts} % using non standard fonts for beamer


\makeatletter
\pretocmd{\section}{\addtocontents{toc}{\protect\addvspace{-5\p@}}}{}{}
\makeatother


\begin{document}
% the titlepage
{\aauwavesbg}
\begin{frame}[plain,noframenumbering]
    \titlepage
\end{frame}

\section{Coding Theory}% 2 min
\label{sec:ct}

\begin{frame}
  \frametitle{Linear Error Correcting Codes}
  \textbf{Definition 1.1 \& 1.5.} Let $\mathcal{C} \subseteq \mathbb{F}_q^n$ be a linear subspace of dimension $k$, then $\mathcal{C}$ is called a $[n, k]_q$ code. Furthermore if $\mathcal{C}$ has minimum distance $d$, then $\mathcal{C}$ is called a $[n, k, d]_{q}$ code. \\
  \begin{itemize}
    \pause
    \item If $\mathcal{C}$ has minimum distance $d$, then it can correct $\lfloor\frac{d-1}{2}\rfloor$ errors but detect $d - 1$ errors.
    %\item Let $\mathcal{C}$ be a $[n, k]_q$ code:
    %   \begin{itemize}
    %      \item If $c_1, c_2, \ldots, c_{k} \in \mathcal{C}$ is a basis of $\mathcal{C}$, then:
    %      \begin{equation*}
    %        G := \begin{bmatrix}
    %           g_1 & g_2 & \cdots & c_{k}
    %             \end{bmatrix}^{T}
    %      \end{equation*}
    %            is called a \textit{generator matrix}.
    %       \item Let $H \in \mathbb{F}_q^{(n - k) \times n}$ such that $\mathcal{C} = \textnormal{null}(H)$, then $H$ is called a \textit{parity check matrix}.
    %  \end{itemize}
  \end{itemize}
\end{frame}

\subsection{Bounds on the Parameters of Codes}
\begin{frame}
  \frametitle{Bounds on the Parameters of Codes.}
  %\textbf{Proposition 1.16.} Let $H$ be a parity check matrix of the $[n, k, d]_q$ code $\mathcal{C}$, then any collection of $d - 1$ columns of $H$ are $\mathbb{F}_q$-linearly independent.\\
  %\pause
  %\textit{Proof:}
  %\begin{itemize}
  %  \item If $h_{i_{1}}, h_{i_2}, \ldots, h_{i_{d-1}}$ are $\mathbb{F}_q$-linearly dependent, then there exists $c_{i_1}, c_{i_2}, \ldots, c_{i_{d - 1}} \in \mathbb{F}_q$ (not all zero) such that:
  %  \begin{equation*}
  %    0 = \sum_{j = 1}^{d-1} c_{i_j} h_{i_j}
  %  \end{equation*}
  %  \pause
  %  \item Letting $c_i = 0$ for $i \not \in \left\{i_1, i_2, \ldots, i_{d-1}\right\}$, we construct a codeword $c \in \mathcal{C}$ such that $wt(c) \leq d -1$.
  %\end{itemize}
  %\\
  %\pause
  \textbf{Corollary 1.18.} Let $\mathcal{C}$ be a $[n, k, d]_{q}$ code, then $d - 1 \leq n - k$. \\
  \pause
  \textit{Proof:} Let $H$ be a parity check matrix of $\mathcal{C}$.
  \begin{itemize}
    \pause
    \item $\dim(\mathcal{C}) = \dim(null(H)) = k \implies rank(H) = n - k$
    \pause
    \item $rank(H) \geq d - 1$, by the Proposition 1.16.
  \end{itemize}
  %\textbf{Theorem 1.29.} Let $q \in \mathbb{N}$ and $\delta \in [0, \frac{q-1}{q}]$ then:
  %\begin{equation*}
  %  \alpha_q^{lin}(\delta) \geq 1 - H_q(\delta)
  %\end{equation*}
\end{frame}

%\begin{frame}
%  \frametitle{Assymptotic Gilbert Varshamov Bound}
%  We let $R^{*}(n, d)$ be the highest:w
%  transmissionrate of a code of length $n$ and minimum distance $d$. \\ \pause In addition we define $\alpha_q^{lin}: [0, 1] \to [0, 1]$ as:
%  \begin{equation*}
%    \alpha_q^{lin}(\delta) = \underset{n \to \infty}{\lim \sup} \; R^{*}(n, \delta n)
%  \end{equation*}
%  \pause
%\end{frame}

\section{Divisors}% 4 min
\label{sec:div}

\begin{frame}
  \frametitle{Divisors}
  \textbf{Definition 2.81 \& 2.82.} Let $\mathcal{X}$ be an aboslutely irreducible regular projective plane curve, over $\mathbb{F}_q$.
  \begin{itemize}
  \pause
  \item A divisor $D$ on $\mathcal{X}$ is a formal sum:
  \begin{equation*}
     D = \sum_{P \in \mathcal{X}} n_P P
  \end{equation*}
  where $n_P = 0$ for all but a finite number of points $P \in \mathcal{X}$.
  \pause
  \item Let $f \in \cF_{q}(\mathcal{X}) \setminus \left\{0\right\}$, then $(f) := \sum_{P \in \mathcal{X}} v_P(f)P$ is called a principal divisor.
  \pause
  \begin{itemize}
    \item $\forall f \in \cF_q(\mathcal{X}) \setminus \left\{0\right\}$ we have $\deg((f)) = 0$ by Proposition 2.83.
  \end{itemize}
  %\pause
  %\item Let $D \in Div(\mathcal{X})$, then $supp(D) := \left\{P \in \mathcal{X}\vert n_P \neq 0\right\}$.
  %\pause
  %\item A $D \in Div(\mathcal{X})$ is called effective if $n_P \geq 0$ for all $P \in \mathcal{X}$.
  \end{itemize}


\end{frame}
\begin{frame}
  \frametitle{The vector space $L(D)$}
  \begin{itemize}

  \end{itemize}
  \textbf{Definition 2.84.} Let $D \in Div(\mathcal{X})$, then we define the vector space $L(D)$ as:
  \begin{equation*}
    L(D) := \left\{f \in \cF_{q}(\mathcal{X}) \setminus \left\{0\right\} \mid (f) + D \text{ is effective}\right\} \cup \left\{0\right\}
  \end{equation*}
  and let $\ell(D) := \dim_{\cF_q}(L(D))$. \\ \; \\

  \pause
  \textbf{Proposition 2.88. (i)} Let $D \in Div(\mathcal{X})$, then $\deg(D) < 0$ implies that $\ell(D) = 0$. \\
  \pause
  \textit{Proof:} For all $f \in \cF_q(\mathcal{X}) \setminus \left\{0\right\}$ we have:
  \begin{equation*}
    \deg((f) + D) = \deg((f)) + \deg(D) = \deg(D) < 0
  \end{equation*}
  since $\deg((f)) = 0$. \pause Meaning $L(D) = \left\{0\right\}$.

\end{frame}

\begin{frame}
  \frametitle{The Riemann-Roch Theorem}
  Let $\mathcal{X}$ be a regular projective plane curve of genus $g$. \\
  \textbf{Theorem 2.91.} Let $D \in Div(\mathcal{X})$, then for all canonical $W \in Div(\mathcal{X})$ we have
  \begin{equation*}
    \ell(D) - \ell(W - D) = \deg(D) - g + 1
  \end{equation*}

  \pause
  \textbf{Corollary 2.92.} If $\deg(D) > 2g - 2$, then $\ell(D) = \deg(D) - g + 1$.
  \pause
  \textit{Proof:}
  \begin{itemize}
    \item $\deg(W - D) < 0$.
  \pause
    \item $\ell(W - D) = 0$ by Proposition 2.88 (i), combining this with Theorem 2.91 yields the result.
  \end{itemize}
\end{frame}

\section{Goppa Codes}% 4 min
\label{sec:goppa}

\begin{frame}
  \frametitle{Goppa Codes}
  \textbf{Definition 3.3.} Let $P_1, P_2, \ldots, P_{n} \in \mathcal{X}$ be $n$ distinct rational points.
  \begin{itemize}
  \pause
    \item Let $D = \sum^n_i P_i$ and $G \in Div(\mathcal{X})$ such that $supp(D) \cap supp(G) = \emptyset$.
  \pause
    \item If $\mathcal{P} := \left(P_1, P_2, \ldots, P_{n}\right)$, then $\mathcal{C}_{D, G} := Ev_{\mathcal{P}}(L(G))$ is called a \textit{Goppa Code}.
  \end{itemize}
\end{frame}

\begin{frame}
  \frametitle{Parameters of Goppa Codes}
  \textbf{Theorem 3.5.} If $\mathcal{C}_{D, G}$ is a $[n, k, d]_q$ code. Then:
  \begin{enumerate}[(i)]
  \pause
    \item $k = \ell(G) - \ell(G - D)$.
  \pause
    \item $d \geq n - \deg(G)$.
  \end{enumerate}
  \pause
  \textit{Proof:}
  \begin{enumerate}[(i)]
    \item Follows from the fact that $Ev_{\mathcal{P}} \restriction_{L(G)}$ is a surjective linear map, from $L(G)$ to $\mathcal{C}_{G, D}$.
      \pause

    \begin{equation*}
      k = \dim_{\mathbb{F}_q}(image(Ev_{\mathcal{P}})) = \ell(G) - \dim_{\mathbb{F}_q}(ker(Ev_{\mathcal{P}}))
    \end{equation*}
    \pause
    But $ker(Ev_{\mathcal{P}}) = L(G - D)$ since:
    \begin{itemize}
    \pause \item $f \in L(G)$ \& $Ev_{\mathcal{P}}(f) = 0 \implies v_{P_i}(f) \geq 1$ and hence $f \in L(G - D)$. As $(f) + G - D$ is effective.
    \pause \item $f \in L(G - D) \implies f(P) = 0 \forall P \in supp(D)$ as $supp(G) \cap supp(D) = \emptyset$. Meaning $f \in ker(Ev_{\mathcal{P}})$.
    \end{itemize}
  %\pause
  %   \item Suppose $c = (f(P_1), f(P_2), \ldots, f(P_{n})) \in \mathcal{C}_{G, D}$.
  %   \begin{itemize}
  %\pause
  %      \item If $wt(c) = d$, then there exists $P_{i_1}, P_{i_2}, \ldots, P_{i_{n - d}} \in supp(D)$ such that $f(P_{i_j}) = 0$.
  %\pause
  %     \item $f \in L \left(G - \sum_{j = 1}^{n - d} P_{i_j}\right)$ since $v_{P_{i_j}}(f) \geq 1$.
  %\pause
  %     \item Since $G - \sum_{j = 1}^{n - d} P_{i_j} + (f)$ is effective we have that:
  %           \begin{equation*}
  %             \underbrace{\deg \left(G - \sum_{j = 1}^{n - d} P_{i_j}\right)}_{= \deg(G) - (n - d)} + \deg((f)) \geq 0
  %           \end{equation*}
  %       \end{itemize}
  \end{enumerate}
\end{frame}

\begin{frame}
  \frametitle{Parameters of Goppa Codes (Continued.)}
  \textbf{Corollary 3.6 (i).} If $\deg(G) < n$, then $k = \ell(G)$. \\
  \pause
  \textit{Proof:} $\deg(G - D) < 0$, the rest follows by Proposition 2.88 (i) and Theorem 3.5 (i). \\ \; \\
  \pause
  \textit{Remark 3.7.}
  \begin{itemize}
    \item $k = \ell(G) \geq \deg(G) - g + 1$ by the Riemann-Roch Theorem 2.91.
  \pause
    \item Combining this with Theorem 3.5 (ii), we see:
  \end{itemize}
      \begin{equation}\label{eq:1}
        d + k \geq \underbrace{(n - \deg(G))}_{\geq d} + \underbrace{(\deg(G) - g + 1)}_{\geq k} = n - g + 1
      \end{equation}
  \begin{itemize}
  \pause
    \item Combinig this with the Singleton Bound we see that $n + 1 \geq d + k \geq n - g + 1$, and that $g = 0$ implies that $\mathcal{C}_{G, D}$ is an MDS code.
  \end{itemize}
\end{frame}

%\begin{frame}
%  \frametitle{The Tsfasman–Vlăduţ–Zink Bound}
%  Equation \eqref{eq:1} yields:
%  \begin{equation*}
%        R + \delta \geq 1 - \frac{g - 1}{n}
%  \end{equation*}
%  \pause
%  Hence we want to maximize the number of rational points and minimize the genus of the underlying curve.\\ \; \\
%  \pause
%  \textbf{Theorem 3.8.} Let $q = p^2$ where $p$ is prime, and $N_q^{*}: \N \to \N$ be the maximum number of $\mathbb{F}_q$-rational points on a absolutely irreducible smooth projective curve over $\mathbb{F}_q$ with genus $\leq g$, then:
%  \begin{equation*}
%    \underset{g \to \infty}{\lim \sup} \; \frac{N_{q}*(g)}{g} \leq \sqrt(q) - 1
%  \end{equation*}
%\end{frame}
%
%
%\begin{frame}
%  \frametitle{The Tsfasman–Vlăduţ–Zink Bound}
%  Combining the previus results we see that, there exists a sequence $\left\{\mathcal{C}_{G_g, D_g}\right\}_{g \in \mathbb{N}}$ such that:
%  \begin{equation}
%    \underset{g \to \infty}{\lim \sup} \; R(\mathcal{C}_{G_g, D_g}) + \delta(\mathcal{C}_{G_g, D_g}) \geq 1 - \frac{1}{\sqrt{q} - 1}
%  \end{equation}
%  This exceeds the Gilbert-Varshamov Bound.
%\end{frame}


%\setbeamerfont{subsubsection in toc}{size=\tiny}
%\begin{frame}{Table of Contents}{}
%\footnotesize{\tableofcontents}
%\end{frame}


\begin{frame}{End}
    \center{Thank you for listening.}
\end{frame}

\end{document}
