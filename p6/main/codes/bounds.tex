\section{Bounds on the Parameters of Linear Codes}\label{sec:bounds}
In this section we will introduce a few select bounds for the parameters of codes, namely the Singleton, Gilbert and asymptotic Gilbert-Varshamov bound. We will start by stating and proving the following lemma, its proof will be based upon the ideas presented in the proof of Theorem 2 found in \cite{the_singleton_bound_and_rs_code}. % FIXME: Find en anden kilde

\begin{proposition}\label{prop:d-1_distinct_columns_are_linearly_independent}
  Let $\mathcal{C}$ be a $[n, k, d]_q$ code, with parity check matrix $H$ then any $d-1$ columns of $H$ are $\mathbb{F}_{q}$-linearly independent.
\end{proposition}

\begin{proof}
  Let $h_1, h_{2}, \ldots, h_n$ be the columns of $H$, assume for the sake of contradiction that there exists distinct indices $i_1, i_2, \ldots, i_{d-1}$ such that $h_{i_{1}}, h_{i_{2}}, \ldots, h_{i_{d-1}}$ are $\mathbb{F}_{q}$-linearly dependent.
  We will now construct a $c \in \mathcal{C} \setminus \left\{0\right\}$ with $\wt(c) \leq d - 1$. We start by setting $c_{k} = 0$ for all $k \not \in \left\{i_1, i_2 \ldots, i_{d-1}\right\}$. Now as $h_{i_{1}}, h_{i_{2}}, \ldots, h_{i_{d-1}}$ are $\mathbb{F}_{q}$-linearly dependent, there exists $c_{i_{1}}, c_{i_{2}}, \ldots, c_{i_{d-1}} \in \mathbb{F}_{q}$, not all $0$, such that
  \begin{equation*}
    \sum^{n}_{j = 1} c_{j}h_{j} = \sum^{d - 1}_{j = 1} c_{i_{j}}h_{i_j} = 0 \implies c \in \mathcal{C}.
  \end{equation*}
  Which is a contradiction since $\wt(c) \leq d - 1$.
\end{proof}
\begin{remark}\label{rem:minimum_distance_coresponds_to_minimum_number_of_linearly_dependent_columns}
Another perhaps more natural connection between the minimum distance of a $[n, k]_{q}$ code $\mathcal{C}$ and one of its  parity check matrices $H$, is that the minimum distance of $\mathcal{C}$ corresponds to the minimum number of linearly dependent columns of $H$. This can be seen easily as the minimum distance of $\mathcal{C}$ corresponds to the minimum weight of $\mathcal{C}$.
\end{remark}
Using Proposition \ref{prop:d-1_distinct_columns_are_linearly_independent} we can now state and prove our first upper bound on the parameters of a $[n, k, d]_{q}$ code.

\begin{corollary}[Singleton Bound]\label{cor:singleton_bound}
  Let $\mathcal{C}$ be a $[n, k, d]_q$ code then $d-1 \leq n - k$.
\end{corollary}

\begin{proof}
  Let $H$ be a parity check matrix of $\mathcal{C}$, then $d - 1 \leq \rank(H)$ as any collection of $d-1$ columns of $H$ are $\mathbb{F}_{q}$-linearly independent, confer Proposition \ref{prop:d-1_distinct_columns_are_linearly_independent}. The rest follows from Lemma \ref{lem:H_has_rank_n-k} as $\rank(H) = n -k$.
\end{proof}

\begin{remark}\label{rem:alternative_formulation_of_singleton_bound}
  Rearranging the Singleton bound to obtain $d + k \leq n + 1$ the trade-off between minimum distance and dimension of a code, becomes even more self-evident, as such we can either wish to find a code with a high minimum distance $d$ or a high dimension $k$, relative to the length of the code, but not both!
\end{remark}

\begin{definition}\label{def:MDS_code}
  If a $[n, k, d]_{q}$ code $\mathcal{C}$ meets the singleton bound, that is $d - 1 = n - k$, it is called a \textit{maximum distance seperable (MDS)} code.
\end{definition}

We have actually already seen a MDS code in Example \ref{exmp:repetition_code}. We will now introduce a more interesting example of a MDS code, based on elements of \cite{alg_geom_codes}[Subsection 1.2.1] as well as \cite{notes_on_alg_geom_codes}[Chapter 2].

\begin{example}\label{exmp:rs_codes}[Reed-Solomon Codes]
  Let $q$ be a prime power, and fix $n, k \in \N$ such that $1 \leq k \leq n \leq q$. Since $n \leq q$ we can chose $P_{1}, P_{2}, \ldots, P_{n} \in \F_{q}$ all distinct and define the set $\mathcal{P} := \{P_{1}, P_{2}, \ldots, P_{n}\}$. Finally, we define the vector space
  \begin{equation*}
    L_{k} := \{F \in \F_{q}[X] \;\vert\; \deg(F) \leq k - 1\}.
  \end{equation*}
  Please note that $0 \in L_{k}$, as we use the convention that $\deg(0) = 0$, confer remark \ref{rem:deg_0}.
  Then the evaluation map:
  \begin{align*}
    \ev\colon L_{k} \to \F_{q}^{n}, \quad F \mapsto (F(P_{1}), F(P_{2}), \ldots, F(P_{n}))
  \end{align*}
  is injective, since $\ev$ is a linear map and $F \in \F_{q}[X] \backslash \{0\}$ can have at most $\deg(F)$ roots, so $\ev(F) = 0$ if and only if $F \equiv 0$. The linear code $\mathcal{C} := \ev(L_{k})$, is called the Reed-Solomon code of degree $k$, we will show that $\mathcal{C}$ is a $[n, k, n - k + 1]_{q}$ code.
  Since $1, X, \ldots, X^{k - 1}$ form a basis of $L_{k}$, we see that $\C$ has dimension $k$, and that
  \begin{equation*}
    G = \begin{bmatrix}
          1 & 1 & \cdots & 1\\
          P_{1} & P_{2} & \cdots & P_{n} \\
          \vdots & \vdots & \ddots & \vdots \\
          P_{1}^{k - 1} & P_{2}^{k - 1} & \cdots & P_{n}^{k -  1}
    \end{bmatrix}
  \end{equation*}
  is a generator matrix for $\C$. \\
  Let $F \in L_{k}$ and $c := (F(P_1), F(P_2), \ldots, F(P_{n}))$ then $F$ vanishes at $n - \wt(c)$ of the points, therefore we have $n - \wt(c) \leq \deg(F) \leq k - 1$.
  In the special case $\wt(c) = d$ and $d$ is the minimum distance of $\mathcal{C}$, we see that $n - d \leq k - 1$ from rearranging this inequality we obtain $n - k \leq d - 1$. Combining this with the Singleton bound (Corollary \ref{cor:singleton_bound}) we obtain $n - k = d - 1$, and thus $\mathcal{C}$ is a MDS code, with minimum distance $n - k + 1$.
\end{example}
% TODO: This is of cause a nice property to have, however these codes aren't the solution to all of our problems as we require $n \leq q$, so their lengths are limited.
Later on we will show that Reed-Solomon codes are a special case of what will be our main object of study, namely algebraic geometry codes.

\subsection{The Assymptotic Gilbert-Varshamov Bound}
The following subsection is based on \cite{huffman}[Section 2.1, 2.8 and 2.10] and \cite{notes_on_alg_geom_codes}[Chapter 5]. We will start by proving a few technical results before moving onto to the Gilbert bound. Following this we introduce some new notation before reaching the main result of this subsection, namely the asymptotic Gilbert-Varshamov bound.

\begin{lemma}\label{lem:number_of_elements_in_sphere}
  Let $x \in \mathbb{F}_q^n$, then $\abs{\overline{B_{r}}(x)} = V_{q}(n, r)$ where $V_{q}(n, r) = \sum_{i=0}^r \binom{n}{i} (q - 1)^{i}$.
\end{lemma}
\begin{proof}
  Let $S_i(x) = \left\{x' \in \mathbb{F}_q^n \middle| \d(x, x') = i\right\}$ for $i = 0, 1, \ldots, r$. Then $S_0(x), S_1(x), \ldots, S_{r}(x)$ are all finite and pairwise disjoint and $\overline{B_r}(x) = \bigcup_{i = 0}^r S_i(x)$. Combining these facts we see that:
  \begin{equation}\label{eq:cardinality_of_disjoint_sets}
    \abs{\overline{B_r}(x)} = \sum_{i = 0}^r \abs{S_i(x)}.
  \end{equation}
  Additionally we have that $\abs{S_i(x)} = \binom{n}{i}(q - 1)^i$, since $x' \in S_{i}$ differ from $x$ in exactly $i$ out of $n$ entries. The result follows immediately by combining this with Equation \eqref{eq:cardinality_of_disjoint_sets}.
\end{proof}

Moving forward we follow the convention of \cite{huffman} and let $B_q(n, d)$ be the maximum number of codewords in a linear code of length $n$ and minimum distance at least $d$.

% We will now state the following Theorem 2.1.7 from \cite{huffman} without proof.

% TODO: Look up the proof before the exam.

\begin{theorem}\label{thm:covering_radius}
  If $\mathcal{C}$ is linear code over $\mathbb{F}_{q}$ of length $n$, with minimum distance $d$ and $B_q(n, d)$ codewords then:
  \begin{equation*}
    \mathbb{F}_q^n = \bigcup_{c \in \mathcal{C}} \overline{B_{d - 1}}(c)
  \end{equation*}
\end{theorem}
The proof of is due to \cite{covering_distance_of_linear_code}.
\begin{proof}
  If there exists an $x \in \mathbb{F}_{q}^{n} \not \in \bigcup_{c \in \mathcal{C}} \overline{B_{d - 1}}(c)$. Then $\mathcal{C}' := \mathcal{C} + \Span\{x\}$ is another linear code, such that $\abs{\mathcal{C}'} > \abs{\mathcal{C}}$. We will show that $\mathcal{C}'$ has minimum distance $d$ which in turn contradicts that $\mathcal{C}$ had $B_{q}(n, d)$ codewords. \\
  Assume for the sake of contradiction, that $\mathcal{C}'$ has minimum distance less than $d$, then there exists a $c' \in \mathcal{C}'$ such that $\wt(c') \leq d - 1$. Since $c' \in \mathcal{C}'$ we may write $c' = c + ax$ for some $c\in \mathcal{C}$ and $a \in \mathbb{F}_{q}$. Furthermore, we have that $a \not= 0$, as $\wt(c') \leq d - 1$ implies that $c' \not\in \mathcal{C}$. In addition, we have that $\wt(b y) = \wt(y)$ for all $b \in \mathbb{F}_q$ and $y\in \mathbb{F}_q^{n}$, since $\mathbb{F}_q$ is a domain. Combining the earlier observations we get that:
   \begin{equation*}
     d - 1 \geq \wt(c') = \wt(-a^{-1}c') = \wt(-a^{-1}c - x) = \d(-a^{-1}c, x).
   \end{equation*}
   However this is a contradiction since this means that $x \in \overline{B_{d - 1}}(-a^{-1}c)$ after all.
 \end{proof}

\newpage
We are now able to state and prove the Gilbert bound which we will use to prove the asymptotic Gilbert-Varshamov bound.
\begin{corollary}[Gilbert bound]\label{cor:gilbert_bound}
  Suppose $V_{q}(n, r)$ is defined as in Lemma \ref{lem:number_of_elements_in_sphere} then:
  \begin{equation*}
    B_q(n, d) \geq \frac{q^{n}}{V_{q}(n, d - 1)}
  \end{equation*}
\end{corollary}

\begin{proof}
  Let $\mathcal{C}$ be a linear code of length $n$, with minimum distance $d$ and $B_q(n, d)$ codewords, then $\mathbb{F}_q^n = \bigcup_{c \in \mathcal{C}} \overline{B_{d - 1}}(c)$ by Theorem \ref{thm:covering_radius}, combining this with Lemma \ref{lem:number_of_elements_in_sphere}, we get:
  \begin{equation*}
    q^{n} = \abs{\mathbb{F}_q^n} \leq \sum_{c \in \mathcal{C}} \abs{\overline{B_{d-1}}(c)} = B_q(n,d) V_q(n, d - 1). \qedhere
  \end{equation*}
\end{proof}

Continuing we define the following function which will be used in the asymptotic Gilbert-Varshamov bound.
\begin{definition}
  Let $q \in \mathbb{N}$ such that $q \geq 2$, then the function $H_q: [0, (q - 1)/q] \to \mathbb{R}$ defined as $H_q(0) = 0$ and $H_q(x) = x \log_{q}(q-1) - x \log_{q}(x) - (1-x)\log_q(1 - x)$ otherwise is called the \textit{$q$-ary entropy function}.
\end{definition}
We will now state Lemma 2.10.3, from \cite{huffman}, but we will omit its computation heavy proof.
\begin{lemma}\label{lem:entropy_limit}
  Let $\delta \in [0, (q - 1) / q]$ where $q \in \mathbb{N}$ such that $q \geq 2$, then:
  \begin{equation*}
    \lim_{n \to \infty} \frac{1}{n} \log_q(V_q(n, \floor{\delta n})) = H_q(\delta).
  \end{equation*}
\end{lemma}

Moving on we introduce two invariants which can be used to gauge the quality of a code.
\begin{definition}
Let $\mathcal{C}$ be a $[n, k, d]_q$ code, then the invariants $R := k / n$ and $\delta := d / n$ is called the \textit{transmission rate} and \textit{relative distance} of $\mathcal{C}$ respectively.
\end{definition}
Following the definition a natural question arise, namely why would one consider the invariants $R$ and $\delta$ instead of using $n, k$ and $d$ directly, when gauging the quality of a $[n,k,d]_q$ code?
To answer this consider the repetition code of length $n$, as seen in Example \ref{exmp:repetition_code}, when $n$ increases then so will the minimum distance $d$ as $d = n$, however $\delta$ will stay constant.
In addition to this we are still only able to encode a single symbol hence we are sacrificing transmission rate. Hence, $R$ and $\delta$ better reflect the properties of a code, especially for large $n$.

\begin{remark}\label{rem:upper_bound_of_sum}
  Let $\mathcal{C}$ be a $[n, k, d]_{q}$ code with transmission rate $R$ and relative minimum distance $\delta$, then $d + k \leq n + 1$ by Remark \ref{rem:alternative_formulation_of_singleton_bound}, dividing by $n$ we get that
  \begin{equation*}
    \delta + R \leq 1 + \frac{1}{n}.
  \end{equation*}
  Thus a long code, with good parameters have $\delta + R$ close to $1$.
\end{remark}

\newpage
Remember that $B_q(n, d)$ is the maximum number of codewords of a linear code $\mathcal{C}$, with length $n$ and minimum distance at least $d$. Since $\mathcal{C}$ is a subspace of $\mathbb{F}_q^{n}$, we have that $\dim(\mathcal{C}) = \log_q(B_q(n, d))$. Thus, it makes sense to define what we shall refer to as the \textit{maximal rate}:
\begin{equation*}
  R^{*}(n, d) = \frac{\log_{q}(B_{q}(n, d))}{n}.
\end{equation*}
We will now investigate the behavior of $R^{*}(n, d)$ as $n \to \infty$, for this we define, the function $\alpha_q^{lin}: [0, 1] \to [0, 1]$ as a function of a relative distance $\delta$, specifically $\delta \mapsto \underset{n \to \infty}{\lim \sup} \;R^{*}(n, \delta n)$. Finally, we can state and prove the asymptotic Gilbert-Varshamov\footnote{The Varshamov bound is another bound for the parameters of linear codes, which can be used to prove the theorem as well.} bound.
\begin{theorem}[Asymptotic Gilbert-Varshamov bound]\label{thm:gilbert_varshamov}
  Let $0 \leq \delta \leq (q - 1)/q$, then:
  \begin{equation*}
  \alpha_q^{lin}(\delta) \geq 1 - H_q(\delta).
  \end{equation*}
\end{theorem}
\begin{proof}
  First note that $B_q(n, \delta n) = B_q(n, \ceil{\delta n})$. Since $B_{q}(n, d)$ is the maximum number of codewords of a code of length $n$ with minimum distance at least $d$ and the minimum distance of a linear code is always a natural number.
  % NOTE: Hvis du undre dig over hvor uligheden kommer fra så test med $\delta n \in \mathbb{Z}$.
  Thus, we have
  \begin{align*}
\alpha_q^{lin}(\delta) = \underset{n \to \infty}{\lim \sup} \; \frac{\log_q(B_q(n, \ceil{\delta n}))}{n} &\geq \underset{n \to \infty}{\lim \sup} \; \frac{\log_{q}\left(\dfrac{q^{n}}{V_{q}(n, \ceil{\delta n} - 1)}\right)}{n}\\ &\geq \underset{n \to \infty}{\lim \sup} \; 1 - \frac{\log_{q}({V_{q}(n, \floor{\delta n})}}{n}
  \end{align*}
  where the first inequality follows from the Gilbert bound (Corollary \ref{cor:gilbert_bound}) and the last inequality follows since $\ceil{\delta n} - 1 \leq \floor{\delta n}$ implies $\frac{q^{n}}{V_{q}(n, \ceil{\delta n} - 1)} \geq \frac{q^{n}}{V_{q}(n, \floor{\delta n})}$. The rest follows by applying Lemma \ref{lem:entropy_limit}, as
  \begin{equation*}
  \lim_{n \to \infty}\frac{\log_{q}({V_{q}(n, \floor{\delta n})}}{n} = H_q(\delta) \implies \underset{n \to \infty}{\lim \sup} \;\frac{\log_{q}({V_{q}(n, \floor{\delta n})}}{n} = H_q(\delta). \qedhere
  \end{equation*}
\end{proof}

