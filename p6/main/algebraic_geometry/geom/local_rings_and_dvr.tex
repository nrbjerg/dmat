\subsection{Local and Discrete Valuation Rings}\label{subsec:local_rings_and_dvr}
The contents of this subsection will be based on \cite{Fulton}[Sections 2.4 and 2.5]. The results presented will apply to both affine and projective varieties, most of the proofs will however only be given in the affine cases, however the results generalize since $\K(\mathcal{X})$ and $\K(\mathcal{X}^{*})$ are isomorphic, if $\mathcal{X}$ is an affine variety.

\begin{definition}
  Let $\mathcal{X}$ be an affine or projective variety and $P \in \mathcal{X}$, then the ring
  \begin{equation*}
    \mathcal{O}_{P}(\mathcal{X}) := \left\{f / g \in \K(\mathcal{X}) \mid f / g \text{ is defined at } P \right\}
  \end{equation*}
  is called \textit{the local ring of} $\mathcal{X}$ \textit{at} $P$, furthermore we define:
  \begin{equation*}
    \mathfrak{m}_{P}(\mathcal{X}) := \left\{f / g \in \mathcal{O}_{P}(\mathcal{X}) \mid f(P) = 0\right\}.
  \end{equation*}
\end{definition}
By identifying $f \in \K[\mathcal{X}]$ with $f / 1 \in \mathcal{O}_P(\mathcal{X})$ we see that $\K[\mathcal{X}]$ is a subring of $\mathcal{O}_{P}(\mathcal{X})$. Additionally, it is straight forward to see that $\mathcal{O}_P(\mathcal{X})$ is a subring of $\K(\mathcal{X})$.

The notion of a local ring $R$ is defined more generally in \cite{lang}[Section 2.4], to be a ring with a unique maximal ideal, in the proposition below we prove that $\mathcal{O}_P(\mathcal{X})$ satisfies this definition.
\begin{proposition}\label{prop:maximal_ideal_of_local_ring}
  Let $\mathcal{X}$ be an affine or projective variety and $P \in \mathcal{X}$, then $\mathfrak{m}_{P}(\mathcal{X})$ is the unique maximal ideal of $\mathcal{O}_P(\mathcal{X})$
\end{proposition}
\begin{proof}
  Suppose $I \subseteq \mathcal{O}_P(\mathcal{X})$ such that $I \not \subseteq \mathfrak{m}_P(\mathcal{X})$, then there exists $f / g \in I$ such that $f(P) \neq 0$, then $f / g$ is a unit as $(f / g)^{-1} = g / f$, is defined at $P$, and hence $(f / g)^{-1}(f / g) = 1 \in I$ which in turn implies that $I = \mathcal{O}_{P}(\mathcal{X})$.
\end{proof}

\begin{lemma}\label{lem:local_ring_is_noetherian_domain}
  Let $\mathcal{X}$ be an affine or projective variety and $P \in \mathcal{X}$, then $\mathcal{O}_{P}(\mathcal{X})$ is a Noetherian domain.
\end{lemma}
\newpage
\begin{proof}
  Let $I \subseteq \mathcal{O}_P(\mathcal{X})$ be an ideal. We will show that $I$ is finitely generated, consider the ideal $I \cap \K[\mathcal{X}]$ of $\K[\mathcal{X}]$, this ring is Noetherian by Proposition \ref{prop:cosets_are_noetherian}, and thus there exists $f_1, f_2, \ldots, f_{m} \in \K[\mathcal{X}]$ which generate $I \cap \K[\mathcal{X}]$. We will show that $f_1, f_2, \ldots, f_{m}$ also generates $I$ in $\mathcal{O}_{P}(\mathcal{X})$. Suppose $\alpha \in I$, then there exists $g, h \in \K(\mathcal{X})$ such that $\alpha = g / h$ and $h(P) \neq 0$, as such we have $h \alpha \in \K[\mathcal{X}] \in I \cap \K[\mathcal{X}]$, now since $I \cap \K[\mathcal{X}]$ was generated by $f_1, f_2, \ldots, f_{m}$ there exists $c_1, c_2, \ldots, c_{m} \in \K[\mathcal{X}]$ such that $h \alpha = \sum_{i = 1}^{m} c_{i} f_{i}$ therefore $\alpha = \sum_{i = 1}^m (c_i / h) f_{i}$. However, since $\alpha$ was chosen arbitrary, we have proven that $\mathcal{O}_{P}(\mathcal{X})$ is a Noetherian domain.
\end{proof}

\begin{definition} \label{def:dvr}
A ring $R$ is called a \textit{discrete valuation ring} if there exists an irreducible element $t \in R$, which we shall call a \textit{uniformizing parameter}, such that every $z \in R \setminus \left\{0\right\}$ can be written uniquely as $z = ut^{m}$, where $u \in R$ is a unit and $m \in \mathbb{N}$ is called the \textit{order} of $z$, which we denote $ord(z)$.
\end{definition}

The following theorem will be especially useful, in our treatment of plane algebraic curves.
\begin{theorem}\label{thm:local_ring_is_a_DVR}
  If $\mathfrak{m}_{P}$ is a principal ideal generated by $t$, then $\mathcal{O}_{P}(\mathcal{X})$ is a discrete valuation ring with uniformizing parameter $t$.
\end{theorem}

\begin{proof}
  Suppose $\gen{t} = \mathfrak{m}_P$ and that $z \in \mathcal{O}_{P}$, we will start by showing uniqueness: Suppose $z = ut^m = v t^l$, where $u, v$ are units and $m, l \in \mathbb{N}$, we can without loss of generality assume that $m \geq l$. This implies that $u  t^{m - l} = v$ however as $t$ generates $\mathfrak{m}_{p}$, it is not a unit, hence we must have $m = l$ and $u = v$.

  As for the existence of a unit $u \in \mathcal{O}_{P}(\mathcal{X})$ and $m \in \mathbb{N}$ such that $z = u t^{m}$: Suppose $z \in \mathcal{O}_{P}(\mathcal{X})$, we may assume that $z$ is not a unit since if $z$ is a unit we may pick $u = z$ and $m = 0$. Now as $z$ is not a unit we have $z \in \mathfrak{m}_{P}$ and hence there exists $z_{1} \in \mathcal{O}_{P}(\mathcal{X})$ such that $z = z_{1} t$, if $z_{1}$ is a unit, then we are finished, otherwise we may similarly write $z_{1} = z_{2} t$, and so on. As such we construct an infinite chain of ideals $\gen{z_{1}} \subseteq \gen{z_{2}} \subseteq \cdots$ since $\mathcal{O}_{P}(\mathcal{X})$ is Noetherian domain, by Lemma \ref{lem:local_ring_is_noetherian_domain}, the chain must have a maximal member, confer Proposition \ref{prop:noetherian_chain_of_ideals_have_maximal_member}. Meaning that there exists $k \in \N$ such that $\gen{z_{k}} = \gen{z_{k +1}}$, which implies $z_{k + 1} = v z_{k}$ for some $v \in \mathcal{O}_{P}(\mathcal{X})$. Combining this with $z_{k} =  z_{k + 1}t$ we see that $z_{k} = v z_{k} t = (vt) z_{k}$, thus $vt = 1$. However, this is a contradiction as $t$ generates $\mathfrak{m}_{P}$ and hence is not a unit.
\end{proof}
