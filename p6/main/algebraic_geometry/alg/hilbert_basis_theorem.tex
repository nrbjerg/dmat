\subsection{Noetherian Rings and Hilbert Basis Theorem}%
In this subsection we will show that if $\mathbb{F}$ is a field, then all ideals of $\mathbb{F}[X_1, X_2, \ldots, X_{n}]$ are finitely generated. This fact will become particularly useful in Subsection \ref{subsec:hilbert_nullstellenzats} and \ref{subsec:local_rings_and_dvr}. The contents will be based on those presented in \cite{Fulton}[Section 1.4].
\begin{definition}
  A ring $R$ is called \textit{Noetherian} if all ideals of $R$ are finitely generated.
  %Let $R$ be a ring. If all ideals of $R$ are finitely generated, then $R$ is called \textbf{Noetherian}.
\end{definition}

Noetherian rings have many nice properties, some of these will be presented and proved in the following results. We start by showing the following fundamental property:

\begin{proposition}\label{prop:noetherian_chain_of_ideals_have_maximal_member}
  Let $R$ be a Noetherian ring, and $I_1 \subseteq I_2 \subseteq \cdots$ be an ascending chain of ideals, then the chain has a maximal member, meaning there exists an $n \in \mathbb{N}$ such that $I_{n} = I_{k}$ for all $k \geq n$.
\end{proposition}
\begin{proof}
  Since $R$ is Noetherian, the ideal $I = \bigcup_{j = 1}^{\infty} I_{j}$, is finitely generated. Suppose $x_1, x_2, \ldots, x_{m} \in I$ generates $I$. Then for each $i \in \left\{1, 2, \ldots, m\right\}$ there exists an $n_{i} \in \mathbb{N}$ such that $x_{i} \in I_{k}$ whenever $k \geq n_{i}$. Setting $n = \max \left\{n_1, n_2, \ldots, n_{m}\right\}$ concludes the proof.
\end{proof}

We now state and prove the main result of this section.
\begin{theorem}[Hilbert Basis Theorem]\label{thm:hbt}
  Let $R$ be a Noetherian ring, then $R[X_1, X_2, \ldots, X_{n}]$ is also a Noetherian ring.
\end{theorem}
\begin{proof}
  The theorem will follow by induction if we can show that if $R$ is a Noetherian ring then $R[X]$ is Noetherian. As such let $I$ be an ideal in $R[X]$, and $J$ be the ideal which consist of the leading coefficients of all polynomial in $I$. However, as $J$ is an ideal in $R$ a finite set of polynomials $\mathcal{F} = \left\{F_1, F_2, \ldots, F_{k}\right\} \subseteq I$ exists, such that the leading coefficients of the polynomials in $\mathcal{F}$ generate $J$. Furthermore, let $M := \max \left\{\deg(F_{1}), \deg(F_{2}), \ldots, \deg(F_{r}) \right\}$ and $J_m$ be the ideal consisting of all leading coefficients of polynomials with degree less than or equal to $m$ for $m \leq M$. By a similar argument there exists a finite set of polynomials $\mathcal{F}_{m} = \left\{F_{m, 1}, F_{m, 2}, \ldots, F_{m, r_{m}}\right\}$ whose leading coefficients generate $J_{m}$.

  Let $I'$ be the ideal generated by the elements in the finite set $\mathcal{F} \cup (\bigcup_{m = 1}^{M} \mathcal{F}_{m})$. We will show that $I' = I$, and hence that $I$ is finitely generated. Assume for the sake of contradiction that $I' \subset I$ then pick $G \in I \setminus I'$ of the lowest degree possible. Then either $\deg(G) > M$ or $\deg(G) \leq M$:
  \begin{enumerate}
    \item If $\deg(G) > M$, then there exists polynomials $H_1, H_2, \ldots, H_{r}$ such that $\sum_{i = 1}^{r} H_{i}F_{i}$ and $G$ have the same leading term, since $F_1, F_2, \ldots, F_{r}$ generated $I$. Now define $G' := G - \sum_{i = 1}^r H_{i}F_{i}$ then $\deg(G') < \deg(G)$ and the hence we have $G' \in I'$, which in turn imply that $G \in I'$, since $I'$ is an ideal.
    \item Else if $\deg(G) = m \leq M$, then we can apply a similar argument this time by setting $G' := G - \sum_{i = 1}^{r_{m}}H_{i} F_{m, i}$.
  \end{enumerate}
  This proves the theorem as $I'$ was finitely generated by a subset of $I$ and we have showed that $I'$ is not smaller than $I$.
\end{proof}
%\begin{proof}
%  Since $R[X_1, X_2, \ldots, X_{n}]$, where defined inductively (confer Definition \ref{def:multivariate_polynomials}) it is sufficent to show that $R[X]$ is Noetherian, when $R$ is Noetherian.
%\end{proof}
The following corollary is rather trivial, since every field is a principal ideal domain. However, it is a fact, that we will use frequently in Section \ref{sec:algebraic_geometry}.
\begin{corollary}\label{cor:pols_noetherian}
  Let $\mathbb{F}$ be a field, then $\mathbb{F}[X_1, X_2, \ldots, X_{n}]$ is a Noetherian ring.
\end{corollary}

The following proposition and its proof are derived from \cite{Fulton}[Problem 1.22]
\begin{proposition}\label{prop:cosets_are_noetherian}
  Let $R$ be a Noetherian ring, and $I \subseteq R$ be an ideal, then $R / I$ is also a Noetherian ring.
\end{proposition}
%Before we can prove the proposition we need the following lemma
%\begin{lemma}\label{lem:natural_one_to_one_correspondence_between_ideals}
%  Let $R$ be a Noetherian ring, and $I \subseteq R$ an ideal, then there is a natural one to one correspondence between the ideals of $R / I$ and the ideals of $R$ which contain $I$
%\end{lemma}
%
%\begin{proof} % Problem 1.22 (Fulton)
%  , we will show the correspondance from ideals in $R$ which contasin $I$ to ideals in $R / I$, and vice versa in two steps, starting with ideals in $R$
%  \begin{enumerate}
%     \item Suppose $J'$ is an ideal in $R / I$, then $\pi^{-1}(J') =: J$ is an ideal in $R$ containing $I$ as: $J$ contains $I$ since $[0] \in J$ and $\pi^{-1}([0]) = I$. Suppose $x, y \in J$, then $[x], [y] \in J'$ and thus $[x + y] \in J'$ we have that $\pi^{-1}([x + y]) = (x + y)I \subseteq J$, which implies that $J$ is closed under addition, for multiplication suppose $x \in J$ and $c \in R$, then $cx \in \pi^{-1}([c][x])$ and $[c][x] \in J'$
%  \end{enumerate}
%
%  \textcolor{red}{\textbf{TODO}}
%\end{proof}

\begin{proof}
  Suppose $J' \subseteq R / I$ is an ideal, consider the natural ring homomorphism $\pi$ defined as $R \ni x \mapsto [x] \in R / I$, then $J := \pi^{-1}(J')$ is an ideal in $R$. This can be seen by letting $x, y \in J$, then $[x], [y] \in J'$ and thus $[x + y] \in J'$ and hence $x + y \in \pi^{-1}([x + y]) \subseteq J$, which implies that $J$ is closed under addition. To see that $J$ is closed under multiplication: Let $c \in R$ and $x \in J$, then $cx \in \pi^{-1}([c][x])$ and $[c][x] \in J'$. Suppose that $x_1, x_2, \ldots, x_{n} \in J$ generates $J$ and let $[y] \in J'$, and $z \in \pi^{-1}([y])$, then there exists $a_1, a_2, \ldots, a_{n} \in R$ such that
  \begin{equation*}
    z  = \sum_{i = 1}^{n} a_{i} x_{i}
  \end{equation*}
  This implies that $[y] = \pi(z) = \pi(\sum_{i = 1}^{n} a_{i} x_{i}) = \sum_{i = 1}^{n} [a_{i}][x_{i}]$ where the last equality follows from the fact that $\pi$ is a ring homomorphism. However, as $[y]$ was chosen arbitrarily, this shows that $R / I$ is a Noetherian ring.
\end{proof}
