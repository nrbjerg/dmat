\section{Groups}%
\label{sec:groups}

A \textit{group} $G$ is a monoid $G$, such that for all $x \in G$ there exists an \textit{inverse element} \footnote{it can be shown that these inverse elements are unique} $x^{-1} \in G$ such that $x x^{-1} = x^{-1} x = e$. Furthermore we write $x^{-n} := (x^{-1})^n$. The group $G$ is called \textit{trivial} if $G = \left\{e\right\}$, in this case we denote $G$ with $1$.

\begin{example}\label{exmp:permuations}
 Let $Perm(S)$ be the set of permutations on $S$, that is the bijective mappings from $S \to S$, then $Perm(S)$ forms a group under the operation $\circ$, defined as $\sigma \circ \tau = \sigma(\tau)$ for all $\sigma, \tau \in Perm(S)$
\end{example}
A group $G$ is said to be \textit{cyclic} if there exists a $g \in G$ such that for all $h \in G$ there exists $n \in \mathbb{N}$ such that $h = g^n$. In this case $g$ is said to be a \textit{cyclic generator} of $G$.
\begin{example}
  The simplest example of a cyclic group, is $\mathbb{Z}$ together with $+$, it is generated by $1$.
\end{example}

Suppose $G_1, G_2, \ldots, G_{n}$ are groups with operations $\cdot_1, \cdot_2, \ldots, \cdot_{n}$, then $G = G_1 \times G_2 \times \cdots \times G_{n}$ forms a group, together with the operation $\cdot: G \times G \to G$, defined by
\begin{equation*}
  (x_1, x_2, \ldots, x_{n}) \cdot (y_1, y_2, \ldots, y_{n}) = (x_1 \cdot_1 y_1, x_2 \cdot_{2} y_{2}, \ldots, x_{n} \cdot_{n} y_{n})
\end{equation*}
this group is called the \textit{direct product group} of $G_1, G_2, \ldots, G_{n}$.

\begin{example}\label{exmp:}
  Consider the group $\mathbb{R}$ together with $+$, then $\mathbb{R}^n$ is a direct group, with respect to addtion (this time of vectors).
\end{example}

Let $G$ be a group, a \textit{subgroup} $H$ is a subset of $G$, that is it self a group.
\begin{proposition}\label{prop:}
  Let $H_1, H_2, \ldots, H_{n}$ be subgroups of $G$, then $\bigcap_{i = 1}^n H_i$ is a subgroup of $G$.
\end{proposition}
\begin{proof}
Suppose $H_1$ and $H_2$ are both subgroups of $G$, then $H_1 \cap H_{2}$ is a subgroup: Since $e \in H_i$, and if $g \in H_i$, then we must have $g^{-1} \in H_{i}$, and since $H_1$ and $H_2$ are both closed under the binary operation, then $H_1 \cap H_2$ will also be (as $g \in H_i \implies g \cdot h \in H_i$ for all $h \in H_i$). The rest now follows by induction.
\end{proof}

Suppose $G$ is a group, then we say that $S \subseteq G$ \textit{generates} $G$ if $g \in G$ can be writen as $\prod^n_{i = 1} x_i$, where every $x_i$ or $x_i^{-1}$ is in $S$. The group $G$ generated by $S$ is the smallest group (with respect to $\subseteq$) that contains $S$. If $S = \left\{x_1, x_2, \ldots, x_{k}\right\}$ is a generator of $G$, then we write
\begin{equation*}
  G = \gen{x_1, x_2, \ldots, x_{k}} = \left\{\prod_{r = 1}^n x_{i_r}^{k_{i_{r}}} \middle| k_{i_r} \in \mathbb{Z}, (i_r)_{r = 1}^n \subseteq (1, 2, \ldots, k)\right\}.
\end{equation*}

\begin{example}\label{exmp:quaternion_group}
  We consider the group generated by the elements $i, j$ such that if $k = ij$ and $m = i^2$, then
  \begin{equation*}
    i^4 = j^4 = k^4 = e, \quad i^2 = j^2 = k^2 = m, \quad ij = mji
  \end{equation*}
  This group will be denoted $\mathbb{H} := \gen{i, j}$ and called the group of \textit{quaternions}.
\end{example}
Let $G, G'$ be groups / monoids, then $f: G \to G'$ is called a \textit{(group / monoid) homeomorphism} between $G$ and $G'$ if $f(xy) = f(x)f(y)$, it can be shown that $f(e) = e'$. A bijective homeomorphism is called a \textit{(group / monoid) isomorphism}. Furthermore a homeomorphism from $G$ to $G$ is called a \textit{endomorphism} and a isomorphism from $G$ to $G$ is called an \textit{automorphism}.

Finally if there exists a isomorphism between $G$ and $G'$, then $G$ and $G'$ is said to be \textit{isomorphic}\footnote{In this case $G$ and $G'$ behave essencially the same (up to a relabeling of the elements).} and we write $G \cong G'$.
If $f: G \to G'$ is a homeomorphism, such that $\tilde{f}: G \to Im(f)$, is a isomophism, then $f$ is called an \textit{embedding}. If $f$ is an injective homeomorphism, we sometimes write $f: G \hookrightarrow G'$.

\begin{proposition}\label{prop:}
  Let $f: G \to G'$ be a homeomorphism, such that $ker(f) = \left\{e\right\}$, then $f$ is injective.
\end{proposition}
\begin{proof}
Let $y, x \in G$ such that $f(x) = f(y)$, then $f(xy^{-1}) = f(x) f(y^{-1}) = e$, hence $x y^{-1} \in ker(f)$, so $x y^{-1} = e$ which implies $y^{-1} = x^{-1} \iff y = x$.
\end{proof}
\begin{remark}
  A injective homeoporhism is an embedding.
\end{remark}

\begin{proposition}\label{prop:}
  Let $G$ be a group and $H, K$ subgroups of $G$, such that $H \cap K = e$, $H K = G$ and $xy = yx$ for all $x \in H, y \in K$. Then $(H \times K) \ni (x, y) \mapsto xy \in G$ is an isomorphism.
\end{proposition}

Let $G$ be a group and $H$ a subgroup, then $aH, a \in G$ is called a \textit{left coset} $H$ in $G$, the \textit{right cosets} are defined similarly. The number of left cosets of $H$ in $G$ is denoted $(G : H)$ and is called the \textit{index}
of $H$ in $G$. The index of the trivial subgroup is called the \textit{order}
of $G$.

\begin{proposition}
  Let $G$ be a group and $H, K$ be subgroups, st $K \subseteq H$, then let $\left\{x_iH\right\}$ be the set of left cosets of $H$ in $G$ and $\left\{y_jK\right\}$ be the set of left cosets of $K$ in $H$, then $\{x_{i}y_{j}K\}$ is the set of cosets of $K$ in $G$
\end{proposition}
\begin{proof}
  Note that $H = \bigcup_i x_iK$ and $G = \bigcup_j y_j H$ (both disjoint unions, as $x \in yH' \implies xH' = yH'$ for all subgroups $H'$)
  Hence $G = \bigcup_j y_{j} \bigcup_i x_i K = \bigcup_{j, i} y_j x_i K$, also a disjoint union since if exists two distinct pair of indicies $(i, j)$ and $(i', j')$ such that $y_jx_iK = y_{j' }x_{i' }K$, then $y_j H = y_{j'} H$ (by multiplying by $H$ on the right), thus $y_j = y_{j'}$ and it follows that $x_i K = x_{i'} K$, and thus $x_i = x_{i'}$.
\end{proof}

The proposition below is a natural consequence.
\begin{proposition}\label{prop:lagrange}
  Let $G$ be a subgroup and $H, K$ be subgroups such that $K \subseteq H$, then $(G : K) = (G : H)(H : K)$. In particular we have $(G : H)(H : 1) = (G : 1)$  in the sense that if two of these indicies, then the third is also finite. If order of $G$ is finite then the order of $H$ divides it.
\end{proposition}

\begin{corollary}
  Every group $G$ of prime order is cyclic.
\end{corollary}
\begin{proof}
Suppose $(G : 1) = p$, let $H$ be a subgroup generated by $a \in G \setminus \left\{e\right\}$, by Proposition \ref{prop:lagrange}, $(G : H)$ divides $p$, however $H$ has atleast two elements, so we must have $\abs{H} = p$. Thus $G$ is cyclic.
\end{proof}
