\chapter{Field Theory}
The following chapter will be based on ``Fields and Galois Theory'' by John M. Howie.
\begin{definition}
  suppose $K, L$ are fields and $\phi: K \to L$ is a monomorphism (an injective homomorphism), then $L$ is called an \textbf{extension field} of $K$, denoted $L : K$.
\end{definition}
Note that since we can identify $K$ with $\phi(K)$, we can regard $K$ as a subfield of $L$ (and $L$ as a vectorspace over $K$). Hence there exists a basis of $L$ over $K$. The cardinality of such a basis is called the \textit{dimension } of $L$, this dimension will be called the \textit{degree of $L$ over $K$}, which will be denoted $[L : K]$
\begin{example}\label{exmp:}
  The degree $[\mathbb{R} : \mathbb{Q}]$ is infinite since $\mathbb{R}$ is uncountable and any finite extension of $\mathbb{Q}$ is countable. In contrast $[\mathbb{C} : \mathbb{R}] = 2$ as $\left\{1, i\right\}$ forms a basis.
\end{example}
\begin{theorem}\label{thm:degree1}
  Let $L : K$, then $L = K$ if and only if $[L : K] = 1$.
\end{theorem}
\begin{proof}
The proof is relatively trivial, so it is omited.
\end{proof}

\begin{theorem}\label{thm:degree_divides}
  Let $M : L$ and $L : K$ then $[M : L][L : K] = [M : K]$.
\end{theorem}
\begin{proof}
  The main idea is to show that if $\left\{a_1, a_2 \ldots, a_{r}\right\}$ is linear independent of $M$ over $L$, and $\left\{b_1, b_2 \ldots, b_{s}\right\}$ is linearly independent of $L$ over $K$, then $\left\{a_ib_j \middle| i = 1, \ldots, r, j = 1, \ldots, s\right\}$
  is linearly independent of $M$ over $K$.
\end{proof}
\begin{corollary}
  Let $K_1, K_2 \ldots, K_{n}$ be fields such that $K_{i + 1} : K$ for all $i = 1, \ldots, n - 1$. Then:
  \begin{equation*}
    [K_n : K_1] = \prod_{i=0}^{n  - 2} [K_{n - i} : K_{n - i - 1}]
  \end{equation*}
\end{corollary}
\begin{exercise}[3.2]
  Let $M : L$ and $L : K$ such that $[M : K] < \infty$ show that $[M : K] = [L : K] \implies M = L$
\end{exercise}
\begin{proof}
  We have $[L : K] [M : L] = [M : K]$ by \autoref{thm:degree_divides}, hence $[M : L] = 1$ since $[M : K] = [L : K]$, by \autoref{thm:degree1}
\end{proof}
\begin{exercise}
  Let $L : K$ such that $[L : K]$ is prime, show that there exists no subfield $E$ of $L$ such that $K \subset E \subset L$.
\end{exercise}
\begin{proof}
Assume for contradiction that such a subfield exists, then $[K : E][E : L] = [L : K]$, however this would mean that $[L : K]$ is composite afterall, since $[K : E] = 1 \iff K = E$ and $[E : L]  = 1 \iff E = L$, which is a contradiction.
\end{proof}
\section{Extensions and Polynomials}%
\begin{definition}
  Let $K : L$ and $S \subseteq L$, let $K(S) = \left\{F \subseteq L \middle| F \text{ is a field and } K \cup S \subseteq F\right\}$, then $K(S)$ is called the \textbf{subfield of $L$ generated over $K$ by $S$}. If $S = \left\{a_1, a_2 \ldots, a_{n}\right\}$ is finite we write $K(S)$ as $K(a_1, a_2 \ldots, a_{n})$.
\end{definition}

\begin{theorem}
The subfield $K(S)$ coincides with the set $E$ of all elements of $L$ that can  be expressed as quotients of finite linear combinations (with coefficients in $K$) of finite products of elements of $S$
\end{theorem}
\begin{remark}
  This is perhaps simply quotients of polynomials?
\end{remark}
When $S = \left\{a\right\}$ we get that $K(a)$ is simply the set of quotients of polynomails in $a$ with coefficients in $K$ and $K(a)$ is called a \textit{simple extension} of $K$.
\begin{theorem}\label{thm:options_for_extensions}
  Let $K : L$  and $a \in L$, then either:
  \begin{enumerate}
     \item $K(a)$ is isomorphic to $K(X)$ the field of rational forms with coefficents in $K$
    \item there exists a unique monic irreducible polynomial $m \in K[X]$  (this is called the \textit{minimal polynomail of $a$}) with the property that for all $f \in K[X]$:
          \begin{enumerate}
\item $f(a) = 0$ if and only if $m | f$.
            \item $K(a) = K[a]$.
                  \item $[K[a]:K] = \deg(m)$.
          \end{enumerate}
  \end{enumerate}
\end{theorem}

\begin{remark}
  If we know that $[K[a] : K] = n$  and we find a monic polynomial $g$ of degree $n$ such that $g(a) = 0$ then $g$ is the minimum polynomial of $a$, the minimum polynomial is unique.
\end{remark}

\begin{definition}
  If $a \in L$ has  a minimum polynomail over $K$, then $a$ is said to be \textbf{algebraic} over $K$ and that $K[a] (=K(a))$ by \autoref{thm:options_for_extensions} is a \textbf{simple algebraic extension} of $K$. A complex number which is algebraic over $\mathbb{Q}$ is called an \textbf{algebraic number}.
  If $K(a)$ is isomorphic to $K(X)$ (the field of rational functions over $K$) we say that $a$ is \textbf{trancendental} over $K$ and $K(a)$ is called a \textbf{simple transcendental extension of $K$}. The number $a \in \mathbb{C}$ which is trancendental over $\mathbb{Q}$ is called a \textbf{trancendental number}.
\end{definition}
We will show in \autoref{thm:algebraic_elements_form_a_subfield} that the set of algebraic numbers forms a subfield of $\mathbb{C}$, this subfield will be denoted by $\mathbb{A}$.

\begin{theorem}
Let $K(a)$ be a simple trancendental extension of $K$. Then $K(a) : K = \infty$.
\end{theorem}
\begin{proof}
The elements $1, a, a^2, \ldots$ are linearly independent over $K$.
\end{proof}

\begin{definition}
$L : K$ is called an \textbf{algebraic extension} if every element of $L$ is algebraic over $K$. Otherwise $L$ is called a \textbf{transcendental extension.}
\end{definition}

\begin{theorem}\label{thm:finite_extensions_are_algebraic}
If $[K : L]$ is finite, then $K : L$ is an algebraic extension.
\end{theorem}
\begin{proof}
Suppose that $a$ is a trancendental element over $K$, then $1, a, a^2, \ldots$ are linearly independent over $K$, so $[K : L] = \infty$ afterall.
\end{proof}

\begin{proposition}\label{prop:algebraic_elements_are_transitive}
  Let $M : L$ and $L : K$ and $a \in M$, then if $a$ is algebraic over $K$, then it is also algebraic over $L$.
\end{proposition}
\begin{proof}
Follows from the fact that: $K[X] \subseteq L[X]$.
\end{proof}

\begin{theorem}\label{thm:algebraic_elements_form_a_subfield}
Let $K : L$ and $\mathcal{A}(L) = \left\{a \in L \middle| a \text{ is algebraic over } K \right\}$, then $\mathcal{A}(L)$ is a subfield of $L$.
\end{theorem}
\begin{proof}
  suppose $a, b \in \mathcal{A}(L)$. Then:
  \begin{equation*}
    a - b \in K(a, b) = (K[a])[b]
  \end{equation*}
  by \autoref{prop:algebraic_elements_are_transitive} $b$ is algebraic over $K[a]$, so both $[K[a] : K]$ and $[(K[a])[b] : K[a]]$ are finite. From \autoref{thm:degree_divides} it follows that $[K(a, b) : K] $ is finite. Hence it follows from \autoref{thm:finite_extensions_are_algebraic} that $a - b$ is algebraic over $K$. A similar argument can be made to show that $a / b \in \mathcal{A}(L)$ for all $a, b (\neq 0) \in \mathcal{A}(L)$.
\end{proof}

\begin{theorem}
The field $\mathbb{A}$ is countable.
\end{theorem}
The proof relises on the arithmetic of infinite cardinal numbers see  \\ (\textit{https://en.wikipedia.org/wiki/Aleph\_number}).
\begin{proof}
Let $|\mathbb{Q}| = \aleph_{0}$, since $\mathbb{Q} \subseteq \mathbb{A}$ we know that $|\mathbb{A}| \geq |\mathbb{Q}| = \aleph_{0}$. Now since the number of monic polynomails of degree $n$ over $\mathbb{Q}$ is $\aleph_0^n = \aleph_{0}$ (can be relized by a process similar to showing that the set of rational numbers are countable.) Each of these polynomials can have at most $n$ roots, hence the number of roots of these monic polynomails are at most $n\aleph_0 = \aleph_0$. So $|\mathbb{A}| \leq \aleph_{0}$.
\end{proof}

\begin{corollary}
  $\mathbb{C} \setminus \mathbb{A} \neq \emptyset$.
\end{corollary}

\begin{proof}
The proof relies on the fact that $\abs{\mathbb{C}} = \abs{\mathbb{R}} = 2^{\aleph_0}$ so $\abs{\mathbb{C} \setminus \mathbb{A}} = 2^{\aleph_0} > 0$.
\end{proof}
