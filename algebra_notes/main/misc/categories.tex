\section{Category Theory}%
\label{sec:label}

A \textit{category} $\mathcal{C}$ consists of a collection of \textit{objects}
$Obj(\mathcal{C})$, for each $A, B \in Obj(\mathcal{C})$, there exists a set $Mor(A, B)$ of \textit{morphisms} (maps between $A$ and $B$), such that for all $A, B, C \in Obj(\mathcal{C})$ there exists a law of composition (ie. map):
\begin{equation*}
  \circ: Mor(B, C) \times Mor(A, B) \to Mor(A, C)
\end{equation*}
that satisifies:
\begin{enumerate}
\item $Mor(A, B) \cap Mor(A', B') = \emptyset$ unless $A = A' \wedge B = B'$, in that case they are equal.
\item for all $A \in Obj(\mathcal{C})$ there exists a morphism $id_A \in Mor(A, A)$,  which acts as a left and right identity for the elements in $Mor(A, B)$ and $Mor(B, A)$, for all $B \in Obj(\mathcal{C})$.
  \item Law of composition is associative meaning if $f \in Mor(A, B), g \in Mor(B, C), h \in Mor(C, D)$, then
        \begin{equation*}
          f \circ (g \circ h) = (f \circ g) \circ h
        \end{equation*}
\end{enumerate}
Every morphism in $\mathcal{C}$, is called an \textit{arrow} and the collection of all arrows is denoted $Ar(\mathcal{C})$.
The morphism $f \in Mor(A, B)$ is called an \textit{isomorphism}
if there exists a $g \in Mor(B, A)$ such that $f \circ g = id_{A}$ and $g \circ f = id_{B}$ if $A = B$, then $f$ is called an \textit{automorphism}, automorphisms of $A$ will be denoted $Aut(A)$, these together with the law of composition forms a group.
.
\begin{example}
  The groups form a category, whose morphisms are the group-homomorphisms.
\end{example}
